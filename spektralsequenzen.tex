\documentclass{cheat-sheet}

\pdfinfo{
  /Title (Zusammenfassung Spektralsequenzen)
  /Author (Tim Baumann)
}

\usepackage{tikz}
%\usetikzlibrary{calc}
%\usetikzlibrary{matrix,shapes,arrows,positioning}

% Kategorientheorie-Makros

% Konzepte
\DeclareMathOperator{\Ob}{Ob} % Objekte (einer Kategorie)
\DeclareMathOperator{\Mor}{Mor} % Morphismenmenge / -klasse
\DeclareMathOperator{\Hom}{Hom} % Homomorphisms
\DeclareMathOperator{\dom}{dom} % Domain
\DeclareMathOperator{\codom}{codom} % Codomain
\newcommand{\op}{\mathrm{op}} % opposite category
\DeclareMathOperator{\Aut}{Aut} % Automorphismengruppe
\newcommand{\ladj}{\dashv} % Links-adjungiert (left-adjoint)
\newcommand{\Lim}{\lim} % Limes
\DeclareMathOperator{\colim}{colim} % Kolimes
\newcommand{\Colim}{\colim} % Kolimes

% Konkrete Kategorien
\newcommand{\SetC}{\mathbf{Set}} % Kategorie der Mengen
\newcommand{\sSet}{\mathbf{sSet}} % Kategorie der simplizialen Mengen
\newcommand{\Top}{\mathbf{Top}} % Kategorie der topologischen Räume
\newcommand{\AbGrp}{\mathbf{Ab}} % Kategorie der abelschen Gruppen
\newcommand{\Grp}{\mathbf{Grp}} % Kategorie der Gruppen
\newcommand{\RMod}{\mathbf{R\text{-}Mod}} % Kategorie der R-Moduln
\newcommand{\Ouv}{\mathbf{Ouv}} % Kategorie der offenene Mengen eines topol. Raumes
\newcommand{\KHaus}{\mathbf{KHaus}} % Kategorie der kompakten Hausdorffräume
\newcommand{\CatC}{\mathbf{Cat}} % Kategorie der kleinen Kategorien
\newcommand{\Vect}{\mathbf{Vect}} % Kategorie der Vektorräume über einem Körper
\newcommand{\VectFin}{\mathbf{Vect}_{\mathrm{fin}}} % Kategorie der endlichen Vektorräume über einem Körper
\newcommand{\kVect}{\text{$k$-$\Vect$}} % Kategorie der k-Vektorräume über einem Körper k
\newcommand{\kVectFin}{\text{$k$-$\VectFin$}} % Kategorie der endlichen k-Vektorräume über einem Körper k
\newcommand{\Mod}{\mathbf{Mod}} % Kategorie der Moduln über einem Ring

% Bezeichnungen für Variablen, die für Kategorien stehen
\newcommand{\Aat}{\mathcal{A}} % Category-A
\newcommand{\Bat}{\mathcal{B}} % Category-B
\newcommand{\Cat}{\mathcal{C}} % Category-C
\newcommand{\Dat}{\mathcal{D}} % Category-D
\newcommand{\Eat}{\mathcal{E}} % Category-E
\newcommand{\Iat}{\mathcal{I}} % Category-I (Indexkategorie)
\newcommand{\Jat}{\mathcal{J}} % Category-J (Indexkategorie)
\newcommand{\Sit}{\mathcal{S}} % Situs-S
 % Kategorientheorie-Makros

%\newcommand{\keS}{k.\,e.\,S.} % kurze exakte Sequenz
%\newcommand{\leS}{l.\,e.\,S.} % lange exakte Sequenz
%\newcommand{\ZM}[1]{\Z_{#1}} % \Z modulo n

% Kleinere Klammern
\delimiterfactor=701


\begin{document}

\maketitle{Spektralsequenzen}

% TODO: Definition Spektralsequenz

Sei $\Aat$ im Folgenden eine abelsche Kategorie.

\begin{defn}
  Eine (homologische) \emph{Spektralsequenz} besteht aus
  \begin{itemize}
    \item Objekten $E^r_{p,q} \in \Ob(\Aat)$ für alle $p, q \in \Z$ und $r \geq 1$,
    \item Morphismen $d^r_{p,q} : E^r_{p,q} \to E^r_{p-r,q+r-1}$ mit $d^r_{p-r,q+r-1} \circ d^r_{p,q} = 0$,
    \item und Isos $\alpha : H_{p,q}(E^r) \!\coloneqq\! \ker(d^r_{p,q}) / \im(d^r_{p+r,q-r+1}) \xrightarrow{\cong} E^{r+1}_{p,q}$.
  \end{itemize}
\end{defn}

% XXX: Morphismus von Spektralsequenzen

\begin{sprech}
  \begin{itemize}
    \item Die Morphismen $d^r_{p,q}$ heißen \emph{Differentiale}.
    \item Die Gesamtheit aller $E^r_{p,q}$ mit $r \in \N$ fest heißt $r$-te \emph{Seite}.
  \end{itemize}
\end{sprech}

\begin{bem}
  Bei einer kohomologischen Spektralsequenz sind die Indizes vertauscht und die Differentiale laufen $d_r^{p,q} : E_r^{p,q} \to E_r^{p+r,q-r+1}$.
\end{bem}

\begin{tikzpicture}[x=10pt,y=10pt]
  \begin{scope}[shift={(0,0)}]
    \foreach \x in {-1,...,4}{
      \foreach \y in {-2,-1,...,3}{
        \node[draw,circle,inner sep=0.5pt,fill] at (\x,\y) {};
      }
    }
    \foreach \x in {-1,...,5}{
      \foreach \y in {-2,-1,...,3}{
        \draw[->,gray] (\x-0.2,\y) -- (\x-0.8,\y);
      }
    }
    \draw[->] (-0.35,-0.35) -- (5.35,-0.35) node[below] {p};
    \draw[->] (-0.35,-0.35) -- (-0.35,4) node[left] {q};
    \node at (5.65,4.5) {$E^1$};
  \end{scope}
  \begin{scope}[shift={(9,0)}]
    \foreach \x in {-1,...,4}{
      \foreach \y in {-2,-1,...,3}{
        \node[draw,circle,inner sep=0.5pt,fill] at (\x,\y) {};
      }
    }
    \foreach \x in {0,...,5}{
      \foreach \y in {-2,-1,...,3}{
        \draw[->,gray] (\x-0.2,\y+0.1) -- (\x-1.8,\y+0.9);
      }
    }
    \draw[->] (-0.35,-0.35) -- (5.35,-0.35) node[below] {p};
    \draw[->] (-0.35,-0.35) -- (-0.35,4) node[left] {q};
    \node at (5.65,4.5) {$E^2$};
  \end{scope}
  \begin{scope}[shift={(18,0)}]
    \foreach \x in {-1,...,4}{
      \foreach \y in {-2,-1,...,3}{
        \node[draw,circle,inner sep=0.5pt,fill] at (\x,\y) {};
      }
    }
    \foreach \x in {1,...,5}{
      \foreach \y in {-2,-1,...,2}{
        \draw[->,gray] (\x-0.2,\y+0.1) -- (\x-2.8,\y+1.9);
      }
    }
    \draw[->] (-0.35,-0.35) -- (5.35,-0.35) node[below] {p};
    \draw[->] (-0.35,-0.35) -- (-0.35,4) node[left] {q};
    \node at (5.65,4.5) {$E^3$};
  \end{scope}
\end{tikzpicture}

% TODO: Zeichnung

% TODO: links/rechts

% TODO: Exakte Paare

% TODO: Serre Spectral Sequence
% TODO: Leray Spectral Sequence
% TODO: May Spectral Sequence
% TODO: Spektralsequenz eines Doppelkomplexes

\end{document}
