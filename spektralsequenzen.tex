\documentclass{cheat-sheet}

\pdfinfo{
  /Title (Zusammenfassung Spektralsequenzen)
  /Author (Tim Baumann)
}

\usepackage{tikz}
\usetikzlibrary{cd}
%\usetikzlibrary{calc}
%\usetikzlibrary{matrix,shapes,arrows,positioning}

% http://tex.stackexchange.com/questions/117732/tikz-and-babel-error
% Es ist schierer Wahnsinn, welche Hacks LaTeX benötigt!
\tikzset{
  every picture/.prefix style={
    execute at begin picture=\shorthandoff{"}
  }
}

% Kategorientheorie-Makros

% Konzepte
\DeclareMathOperator{\Ob}{Ob} % Objekte (einer Kategorie)
\DeclareMathOperator{\Mor}{Mor} % Morphismenmenge / -klasse
\DeclareMathOperator{\Hom}{Hom} % Homomorphisms
\DeclareMathOperator{\dom}{dom} % Domain
\DeclareMathOperator{\codom}{codom} % Codomain
\newcommand{\op}{\mathrm{op}} % opposite category
\DeclareMathOperator{\Aut}{Aut} % Automorphismengruppe
\newcommand{\ladj}{\dashv} % Links-adjungiert (left-adjoint)
\newcommand{\Lim}{\lim} % Limes
\DeclareMathOperator{\colim}{colim} % Kolimes
\newcommand{\Colim}{\colim} % Kolimes

% Konkrete Kategorien
\newcommand{\SetC}{\mathbf{Set}} % Kategorie der Mengen
\newcommand{\sSet}{\mathbf{sSet}} % Kategorie der simplizialen Mengen
\newcommand{\Top}{\mathbf{Top}} % Kategorie der topologischen Räume
\newcommand{\AbGrp}{\mathbf{Ab}} % Kategorie der abelschen Gruppen
\newcommand{\Grp}{\mathbf{Grp}} % Kategorie der Gruppen
\newcommand{\RMod}{\mathbf{R\text{-}Mod}} % Kategorie der R-Moduln
\newcommand{\Ouv}{\mathbf{Ouv}} % Kategorie der offenene Mengen eines topol. Raumes
\newcommand{\KHaus}{\mathbf{KHaus}} % Kategorie der kompakten Hausdorffräume
\newcommand{\CatC}{\mathbf{Cat}} % Kategorie der kleinen Kategorien
\newcommand{\Vect}{\mathbf{Vect}} % Kategorie der Vektorräume über einem Körper
\newcommand{\VectFin}{\mathbf{Vect}_{\mathrm{fin}}} % Kategorie der endlichen Vektorräume über einem Körper
\newcommand{\kVect}{\text{$k$-$\Vect$}} % Kategorie der k-Vektorräume über einem Körper k
\newcommand{\kVectFin}{\text{$k$-$\VectFin$}} % Kategorie der endlichen k-Vektorräume über einem Körper k
\newcommand{\Mod}{\mathbf{Mod}} % Kategorie der Moduln über einem Ring

% Bezeichnungen für Variablen, die für Kategorien stehen
\newcommand{\Aat}{\mathcal{A}} % Category-A
\newcommand{\Bat}{\mathcal{B}} % Category-B
\newcommand{\Cat}{\mathcal{C}} % Category-C
\newcommand{\Dat}{\mathcal{D}} % Category-D
\newcommand{\Eat}{\mathcal{E}} % Category-E
\newcommand{\Iat}{\mathcal{I}} % Category-I (Indexkategorie)
\newcommand{\Jat}{\mathcal{J}} % Category-J (Indexkategorie)
\newcommand{\Sit}{\mathcal{S}} % Situs-S
 % Kategorientheorie-Makros

%\newcommand{\keS}{k.\,e.\,S.} % kurze exakte Sequenz
\newcommand{\leS}{l.\,e.\,S.} % lange exakte Sequenz
%\newcommand{\ZM}[1]{\Z_{#1}} % \Z modulo n

\newenvironment{centertikzcd}
  {\begin{center}\begin{tikzcd}}
  {\end{tikzcd}\end{center}}

% Kleinere Klammern
\delimiterfactor=701


\begin{document}

\maketitle{Spektralsequenzen}

Sei $\Aat$ im Folgenden eine abelsche Kategorie.

\begin{defn}
  Eine (homologische) \emph{Spektralsequenz} (SS) besteht aus
  \begin{itemize}
    \item Objekten $E^r_{p,q} \in \Ob(\Aat)$ für alle $p, q \in \Z$ und $r \geq 1$,
    \item Morphismen $d^r_{p,q} : E^r_{p,q} \to E^r_{p-r,q+r-1}$ mit $d^r_{p-r,q+r-1} \circ d^r_{p,q} = 0$
    \item und Isos $\alpha : H_{p,q}(E^r) \!\coloneqq\! \ker(d^r_{p,q}) / \im(d^r_{p+r,q-r+1}) \xrightarrow{\cong} E^{r+1}_{p,q}$.
  \end{itemize}
\end{defn}

% XXX: Morphismus von Spektralsequenzen

\begin{sprech}
  \begin{itemize}
    \item Die Morphismen $d^r_{p,q}$ heißen \emph{Differentiale}.
    \item Die Gesamtheit $E^r \coloneqq \{ E^r_{p,q} \}_{p,q}$ mit $r \in \N$ fest heißt $r$-te \emph{Seite}.
  \end{itemize}
\end{sprech}

\begin{bem}
  Man stellt Seiten in einem 2-dim Raster dar:
\end{bem}

\begin{tikzpicture}[x=10pt,y=10pt]
  \begin{scope}[shift={(0,0)}]
    \foreach \x in {-1,...,4}{
      \foreach \y in {-2,-1,...,3}{
        \node[draw,circle,inner sep=0.5pt,fill] at (\x,\y) {};
      }
    }
    \foreach \x in {-1,...,5}{
      \foreach \y in {-2,-1,...,3}{
        \draw[->,gray] (\x-0.2,\y) -- (\x-0.8,\y);
      }
    }
    \draw[->] (-0.35,-0.35) -- (5.35,-0.35) node[below] {p};
    \draw[->] (-0.35,-0.35) -- (-0.35,4) node[left] {q};
    \node at (5.65,4.5) {$E^1$};
  \end{scope}
  \begin{scope}[shift={(9,0)}]
    \foreach \x in {-1,...,4}{
      \foreach \y in {-2,-1,...,3}{
        \node[draw,circle,inner sep=0.5pt,fill] at (\x,\y) {};
      }
    }
    \foreach \x in {0,...,5}{
      \foreach \y in {-2,-1,...,3}{
        \draw[->,gray] (\x-0.2,\y+0.1) -- (\x-1.8,\y+0.9);
      }
    }
    \draw[->] (-0.35,-0.35) -- (5.35,-0.35) node[below] {p};
    \draw[->] (-0.35,-0.35) -- (-0.35,4) node[left] {q};
    \node at (5.65,4.5) {$E^2$};
  \end{scope}
  \begin{scope}[shift={(18,0)}]
    \foreach \x in {-1,...,4}{
      \foreach \y in {-2,-1,...,3}{
        \node[draw,circle,inner sep=0.5pt,fill] at (\x,\y) {};
      }
    }
    \foreach \x in {1,...,5}{
      \foreach \y in {-2,-1,...,2}{
        \draw[->,gray] (\x-0.2,\y+0.1) -- (\x-2.8,\y+1.9);
      }
    }
    \draw[->] (-0.35,-0.35) -- (5.35,-0.35) node[below] {p};
    \draw[->] (-0.35,-0.35) -- (-0.35,4) node[left] {q};
    \node at (5.65,4.5) {$E^3$};
  \end{scope}
\end{tikzpicture}

\begin{bem}
  Bei einer kohomologischen Spektralsequenz sind die Indizes vertauscht und die Differentiale laufen $d_r^{p,q} : E_r^{p,q} \to E_r^{p+r,q-r+1}$.
\end{bem}

\begin{defn}
  Eine Spektralsequenz \emph{konvergiert}, falls für alle $p, q \in \Z$ ein $R \in \N$ existiert, sodass für alle $r \geq R$ die Differentiale von und nach $E^r_{p,q}$ null sind und damit $E^\infty_{p,q} \coloneqq E^R_{p,q} \cong E^{R+1}_{p,q} \cong E^{R+2}_{p,q} \ldots$ \\
  Der \emph{Grenzwert} der SS ist die Unendlich-Seite $E^\infty \coloneqq \{ E^\infty_{p,q} \}_{p,q}$.
\end{defn}

\begin{nota}
  $E^r \Rightarrow E^\infty$
\end{nota}

\begin{defn}
  Eine SS \emph{degeneriert} auf Seite $R$, wenn $d^r_{p,q} \!=\! 0$ für alle $r \!\geq\! R$.
\end{defn}

\begin{bem}
  Das entspricht der gleichmäßigen Konv. aus der Analysis.
\end{bem}

\begin{bem}
  Viele Spektralsequenzen leben im ersten Quadranten, \dh{} $E^r_{p,q} = 0$ wenn $p < 0$ oder $q < 0$. Das impliziert, dass für $p$, $q$ fest und $r$ groß alle Differentiale von und nach $E^r_{p,q}$ aus dem ersten Quadranten heraus- oder hineinführen und damit Null sind.
  %Somit konvergieren solche Spektralsequenzen immer.
\end{bem}

\begin{defn}
  Ein \emph{exaktes Pärchen} in $\Aat$ ist gegeben durch Objekte $A, E \in \Ob(\Aat)$ und Morphismen wie folgt
  \begin{centertikzcd}
    A \arrow[rr, "i"] &&
    A \arrow[ld, "j"] \\
    & E \arrow[lu, "k"]
  \end{centertikzcd}
  sodass das Dreieck an jeder Ecke exakt ist.
\end{defn}

\begin{bem}
  Für das Differential $d \coloneqq j \circ k : E \to E$ gilt $d^2 = 0$.
\end{bem}

\begin{defn}
  Sei ein exaktes Pärchen wie oben gegeben. \\
  Dann gibt es ein \emph{abgeleitetes Pärchen}
  \begin{centertikzcd}
    A' \arrow[rr, "i'"] &&
    A' \arrow[ld, "j'"] \\
    & E' \arrow[lu, "k'"]
  \end{centertikzcd}
  mit \quad
  \inlineitem{$E' \coloneqq \ker(d) / \im(d)$,} \quad
  \inlineitem{$A' \coloneqq i(A) \subset A$,} \\
  \inlineitem{$i' \coloneqq i|_{A'}$} \quad
  \inlineitem{$j'(i(a)) \coloneqq [j(a)] \in E'$} \quad
  \inlineitem{$k'([e]) \coloneqq k(e)$}
\end{defn}

% 1.1 im Hatcher
\begin{lem}
  Das abgeleitete Pärchen eines exakten Pärchens ist exakt.
\end{lem}

\begin{bem}
  Man erhält nun aus einem exakten Pärchen eine Spektralsequenz (im nachfolgenden Sinne) durch iteriertes Ableiten.
\end{bem}

\begin{bem}
  Man kann auch die $r$-te Seite als einzelnes Obj. $E^r$ auffassen. Dann ist eine \emph{Spektralsequenz} gegeben durch Objekte $E^r$, $r \geq 1$, Differentiale $d^r : E^r \to E_r$ mit $d^r \circ d^r = 0$ und Isomorphismen $\alpha^r : H(E^r) \coloneqq \ker(d^r) / \im(d^r) \to E^{r+1}$.
\end{bem}

\begin{defn}
  Eine \emph{Filtrierung} eines $A$-Moduls $M$ ist eine aufsteigende Folge $\nldots \subseteq F_p M \subseteq F_{p+1} M \subseteq \nldots$ von Untermodulen von $M$ mit $p \in \Z$, sodass $0 = \cap_p F_p M$ und $M = \cup_p F_p M$.
\end{defn}
% XXX: beschränkte Filtrierungen

\begin{bem}
  Sei $\ldots \subseteq X_p \subseteq X_{p+1} \subseteq \ldots$ eine aufsteigende Filtrierung eines topologischen Raumes $X$. Man kann dann die Homologiegruppen schön übersichtlich in ein Raster schreiben:
  \begin{centertikzcd}[row sep=0.4cm, column sep=0.25cm, font=\scriptsize]
    \Emph{H_{n\!+\!1}(X_p)} \arrow[r, Emph] \arrow[d] & \Emph{H_{n\!+\!1}(X_p, X_{p\!-\!1})} \arrow[r, Emph] \arrow[d] & \Emph{H_n(X_{p\!-\!1})} \arrow[r] \arrow[d, Emph] & H_n(X_{p\!-\!1}, X_{p\!-\!2}) \arrow[r] \arrow[d] & \nldots \\
    H_{n\!+\!1}(X_{p\!+\!1}) \arrow[r] \arrow[d] & H_{n\!+\!1}(X_{p\!+\!1}, X_p) \arrow[r] \arrow[d] & \Emph{H_n(X_p)} \arrow[r, Emph] \arrow[d] & \Emph{H_n(X_p, X_{p\!-\!1})} \arrow[r, Emph] \arrow[d] & \nldots \\
    H_{n\!+\!1}(X_{p\!+\!2}) \arrow[r] & H_{n\!+\!1}(X_{p\!+\!2}, X_{p\!+\!1}) \arrow[r] & H_n(X_{p\!+\!1}) \arrow[r] & H_n(X_{p\!+\!1}, X_{p}) \arrow [r] & \nldots
  \end{centertikzcd}
  Die langen exakten Sequenzen von Raumpaaren liegen treppenstu- fenartig in diesem Raster.
  Man erhält aus den langen Morphismen wie in den $\leS{}$ (rechts, rechts, runter) exaktes Pärchen $(A, E)$ mit $A_{n+1,p} \coloneqq H_n(X_p)$ und $E_{n,p}^1 \coloneqq H_n(X_p, X_{p-1})$.
\end{bem}

\subsection{Die Leray-Serre-Spektralsequenz}

\begin{defn}
  Eine \emph{Serre-Faserung} ist eine stetige Abb. $p : E \to B$, die die Homotopieliftungseigenschaft für alle CW-Komplexe $A$ erfüllt, \dh{}
  für alle $H$, $H_0$ wie unten, sodass das Quadrat kommutiert, gibt es ein diagonales $\tilde{H}$, sodass die Dreiecke kommutieren:
  \begin{centertikzcd}[row sep=0.8cm, column sep=1cm]
    A \arrow[r, "H_0"] \arrow[d, hook, "i_0"] &
    E \arrow[d, "p"] \\
    A \times \I \arrow[r, "H"] \arrow[ur, "\exists\, \tilde{H}", dashed] &
    B
  \end{centertikzcd}
\end{defn}

\begin{lem}
  Die Homotopieliftungseig. ist genau dann für alle CW-Kom- plexe erfüllt, wenn sie für die Kuben $A = \I^n$ erfüllt ist.
\end{lem}

\begin{bem}
  Jeder stetige Weg $\gamma : \I \to B$ in $B$ induziert eine Homoto- pieäquivalenz $\gamma_* : p^{-1}(\gamma(0)) \to p^{-1}(\gamma(1))$ zwischen den Fasern über Anfangs- und Endpunkt.
  Wenn $B$ wegzshgd ist, so sind alle Fasern homotopieäquivalent und man notiert $F \to E \to B$ für die Faserung, wobei $F$ die Faser über einem beliebigen Punkt ist. Man erhält eine Wirkung der Fundamentalgr. $\pi_1(B)$ auf der Homologie $H_k(F)$ durch
  \[ \pi_1(B) \to \Aut(H_k(F)), \quad [(\gamma : \I \to B)] \mapsto (\gamma_* : F \to F)_* \]
\end{bem}

% 1.3 im Hatcher
\begin{thm}
  Sei $F \to E \to B$ eine Serre-Faserung, $B$ wegzshgd und $G$ eine ab. Gruppe. Angenommen, $\pi_1(B)$ wirkt trivial auf $H_*(F; G)$. \\
  Dann gibt es die \emph{(Leray-)Serre-Spektralsequenz} mit
  \[ E^2_{p,q} = H_p(B; H_q(F; G)), \]
  deren Eintrag $E^\infty_{p,n-p}$ der Quotient $F^p_n/F^{p-1}_n$ in einer Filtration
  $0 \subseteq F_n^0 \subseteq \ldots \subseteq F_n^n = H_n(X; G)$ von $H_n(X; G)$ ist.
\end{thm}
% XXX: Verallgemeinerung auf lokale Koeffizienten?

\begin{bem}
  Wenn $G$ ein Vektorraum ist, so folgt $H_n(X; G) \cong \oplus_p E^\infty_{p,n-p}$.
\end{bem}

% 1.14 im Hatcher
\begin{thm}
  Sei $F \to E \to B$ eine Serre-Faserung, $B$ wegzshgd und $G$ eine ab. Gruppe. Angenommen, $\pi_1(B)$ wirkt trivial auf $H^*(F; G)$.
  Dann ex. die \emph{(Leray-)Serre-Spektralsequenz} für Kohomologie mit
  \[ E_2^{p,q} = H^p(B; H^q(F; G)), \]
  deren Eintrag $E_\infty^{p,n-p}$ der Quotient $F_p^n/F_{p+1}^n$ in einer Filtration
  $0 \subseteq F_n^n \subseteq \ldots \subseteq F_0^n = H^n(X; G)$ von $H^n(X; G)$ ist.
\end{thm}

% TODO: Multiplikative Struktur auf der Serre-Spektralsequenz

% TODO: Leray Spectral Sequence
% TODO: May Spectral Sequence
% TODO: Spektralsequenz eines Doppelkomplexes

\end{document}
