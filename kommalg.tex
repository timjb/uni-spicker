\documentclass{cheat-sheet}

\usepackage{stmaryrd} % \mapsfrom

\pdfinfo{
  /Title (Zusammenfassung Kommutative Algebra)
  /Author (Tim Baumann)
}

\DeclareMathOperator{\ggT}{ggT} % größter gemeinsamer Teiler
\DeclareMathOperator{\kgV}{kgV} % kleinstes gemeinsames Vielfaches
\newcommand{\Ring}{\mathbf{Ring}} % Kategorie der Ringe
\DeclareMathOperator{\ann}{ann} % Annulator

% Bezeichnungen für Ideale
\newcommand{\aaa}{\mathfrak{a}}
\newcommand{\bbb}{\mathfrak{b}}
\newcommand{\ccc}{\mathfrak{c}}
\newcommand{\jjj}{\mathfrak{j}}
\newcommand{\ppp}{\mathfrak{p}}
\newcommand{\qqq}{\mathfrak{q}}
\newcommand{\mmm}{\mathfrak{m}}
\newcommand{\nnn}{\mathfrak{n}}

\begin{document}

\maketitle{Zusammenfassung Kommutative Algebra}

% I. Ringe und Ideale
\section{Ringe und Ideale}

% 1. Ringe und Ringhomomorphismen

% 1.1. Ringe

\begin{defn}
  Ein Ring ist ein Tupel $(A, +, \cdot, 0, 1)$ mit einer Menge~$A$, Operationen $+, \cdot : A \times A \to A$ und Elementen $0, 1 \in A$, sodass
  \begin{itemize}
    \item $(A, +, 0)$ eine abelsche Gruppe ist,
    \item $(A, \cdot, 1)$ ein Monoid ist und
    \item die Multiplikation distributiv über die Addition ist, \dh{}
    \[
      x (y + z) = x y + x z
      \enspace \text{und} \enspace
      (y + z) x = y x + z x
      \quad \forall \, x, y, z \in A.
    \]
  \end{itemize}
\end{defn}

\begin{bspe}
  \inlineitem{$\Z$,} \enspace
  \inlineitem{$K[x_1, \ldots, x_n]$,} \enspace
  \inlineitem{\emph{Nullring}: der Ring mit $0 = 1$}
\end{bspe}

% 1.2. Unterringe

\begin{defn}
  Sei $(A, +, \cdot)$ ein Ring.
  Eine Teilmenge $B \subseteq A$ heißt \emph{Unterring}, falls $0, 1 \in B$ und $B$ unter $+$ und $\cdot$ abgeschlossen ist.
\end{defn}

\begin{bspe}
  \inlineitem{$\Z \subset \Q$}, \enspace
  \inlineitem{$K \subset K[X]$}
\end{bspe}

% 1.3. Ringhomomorphismen

\begin{defn}
  Ein \emph{Ringhomomorphismus} $\phi : A \to B$ ist eine Abbildung, welche sowohl ein Gruppenhomomor. $(A, +_A, 0_A) \to (B, +_B, 0_B)$ als auch ein Ringhomomorphismus $(A, \cdot_A, 1_A) \to (B, \cdot_B, 1_B)$ ist.
\end{defn}

\begin{bem}
  Ringe und Ringhomomorphismen bilden eine Kategorie $\Ring$.
\end{bem}

\begin{lem}
  Ein Ringhomomorphismus ist genau dann ein Isomorphismus (in dieser Kategorie), wenn er bijektiv ist.
\end{lem}

\begin{konv}
  Seien $A$ im Folgenden Ringe und $\phi : A \to B$ ein Ringhomomorphismus.
\end{konv}

% 2. Ideale und Quotientenringe

% 2.1. Ideale

\begin{defn}
  Eine Teilmenge $\aaa \subseteq A$ heißt (beidseitiges) \emph{Ideal} von $A$, falls
  \begin{itemize}
    \item $\aaa \subseteq A$ eine Untergruppe ist und
    \item für alle $a \in A$ und $x \in \aaa$ gilt: $ax, xa \in \aaa$.
  \end{itemize}
\end{defn}

\begin{lem}
  Der Schnitt von (beliebig vielen) Idealen ist selbst ein Ideal.
\end{lem}

\begin{defn}
  Sei $M \subseteq A$ eine Teilmenge.
  Das von $M$ \emph{erzeugte Ideal} ist der Schnitt aller Ideale von~$A$, die $M$ umfassen.
\end{defn}

\begin{bem}
  Falls $A$ kommutativ ist, so gilt
  \[
    \text{von $M$ erzeugtes Ideal} = \Set{\sum_{k=1}^n \lambda_k x_k}{n \in \N, \lambda_k \in A, x_k \in M}.
  \]
\end{bem}

\begin{nota}
  $(x_1, \ldots, x_n) \subseteq A$ ist das von $\{ x_1, \ldots, x_n \}$ erzeugte Ideal.
\end{nota}

\begin{bem}
  \begin{minipage}[t]{0.85 \linewidth}
    \begin{itemize}
      \item Das \emph{Nullideal} $(0)$ ist das kleinste Ideal, denn $(0) = \{ 0 \}$.
      \item Das \emph{Einsideal} $(1)$ ist das größte Ideal, denn $(1) = A$.
    \end{itemize}
  \end{minipage}
\end{bem}

% 2.2. Bild und Kern

\begin{prop}
  \begin{itemize}
    \item Sei $\bbb \subseteq B$ ein Ideal.
    Dann ist auch $\phi^{-1}(\bbb) \subseteq A$ ein Ideal.
    \item Sei $A' \subseteq A$ ein Unterring.
    Dann ist auch $\phi(A') \subseteq B$ ein Unterring.
  \end{itemize}
\end{prop}

\begin{defn}
  Das Ideal $\ker \phi \coloneqq \phi^{-1}((0))$ heißt \emph{Kern} von $\phi$.
\end{defn}

\begin{bem}
  $\phi$ ist injektiv $\iff$ $\ker \phi = 0$
\end{bem}

% selbst hinzugefügt
\begin{prop}
  Sei $\phi : A \to B$ surjektiv, $\aaa \subseteq A$ ein Ideal.
  Dann ist auch das Bild $\phi(A) \subseteq B$ ein Ideal.
\end{prop}

% 2.3. Quotientenringe

\begin{prop}
  Sei $\aaa \subseteq A$ ein Ideal.
  Dann gibt es einen Ring $A/\aaa$ und einen Ringhomomor. $\pi : A \to A/\aaa$ mit folgender universeller Eigenschaft:

  \hfill\begin{minipage}{0.95 \linewidth}
    Für jeden Ring $B$ und Ringhomomor. $\psi : A \to B$ mit $\aaa \subseteq \ker \psi$ gibt es genau einen Ringhomomor. $\widetilde{\psi} : A/\aaa \to B$ mit $\psi = \widetilde{\psi} \circ \pi$.
  \end{minipage}
\end{prop}

\begin{konstr}
  Sei durch $x \sim y \coloniff x - y \in \aaa$ eine Äq'relation $\sim$ auf~$A$ definiert.
  Setze $A/\aaa \coloneqq A/{\sim}$ und $\pi(x) \coloneqq [x]$.
  Die Addition und Multiplikation auf~$A$ ind. die Addition bzw. Multiplikation auf $A/\aaa$.
\end{konstr}

\begin{defn}
  $A/\aaa$ heißt \emph{Quotientenring} von $A$ nach $\aaa$.
\end{defn}

\begin{nota}
  Man lässt häufig die Äquivalenzklammern weg, man schreibt also
  "`$x = y$ in $A/\aaa$"' anstatt "`$[x] = [y]$"'.
\end{nota}

\begin{prop}
  Sei $\aaa \subseteq A$ ein Ideal.
  Folgende Korresp. ist bij. und monoton:
  \[
    \begin{array}{r c l}
      \{ \text{ Ideale $\bbb \subseteq A$ mit $\bbb \supseteq \aaa$ } \} & \leftrightarrow & \{ \text{ Ideale $\ccc \subseteq A/\aaa$ } \} \\
      \bbb & \mapsto & \pi(\bbb) \\
      \pi^{-1}(\ccc) & \mapsfrom & \ccc
    \end{array}
  \]
\end{prop}

\begin{prop}[\emph{Homomorphiesatz}]
  Sei $\phi : A \to B$ ein Ringhomomor.
  Dann ist $\underline{\phi} : A / \ker(\phi) \to \im(\phi), \enspace [x] \mapsto \phi(x)$ ein Ringisomorphismus.
\end{prop}

% 3. Nullteiler, nilpotente Elemente und Einheiten

% 3.1. Integritätsbereiche

Im Folgenden seien alle Ringe \emph{kommutativ}, \dh{} $xy = yx$ f.\,a. $x, y$.

\begin{defn}
  Sei $A$ ein kommutativer Ring.
  Ein Element $x \in A$ heißt
  \begin{itemize}
    \item \emph{regulär}, falls $\fa{y \in A} xy = 0 \implies y = 0$.
    \item \emph{Nullteiler}, falls es nicht regulär ist, \dh{} wenn ein $y \in A \setminus \{ 0 \}$ mit $xy = 0$ existiert.
  \end{itemize}
\end{defn}

\begin{defn}
  Ein Ring $A$ heißt \emph{Integritätsbereich}, wenn $0 \in A$ der einzige Nullteiler in $A$ ist.
\end{defn}

\begin{acht}
  Die Null im Nullring ist regulär!
\end{acht}

\begin{bem}
  Ein Ring $A$ ist genau dann ein Integritätsbereich, wenn
  \[
    0 \neq 1 \text{ in $A$}
    \qquad \text{und} \qquad
    \fa{x, y \in } xy = 0 \implies x = 0 \vee y = 0.
  \]
\end{bem}

% weglassen?
\begin{beob}
  Sei $\phi : A \to B$ ein injektiver Ringhomomorphismus.
  Ist $B$ ein Integritätsbereich, so auch $A$.
\end{beob}

\begin{defn}
  Ein Ideal $\aaa \subseteq A$ heißt \emph{Hauptideal}, falls $\aaa = (a)$ für ein $a \in A$. \\
  Ein Ring~$A$ heißt \emph{Hauptidealbereich}, falls jedes Ideal in $A$ ein Hauptideal ist.
\end{defn}

\begin{bspe}
  \inlineitem{$\Z$,} \enspace
  \inlineitem{$K[x]$}
\end{bspe}

\begin{gegenbsp}
  \inlineitem{$K[x_1, \ldots, x_n]$ für $n \geq 2$}
\end{gegenbsp}

% 3.2. Nilpotente Elemente

\begin{defn}
  Ein Element $x \in A$ heißt \emph{nilpotent}, falls $\ex{n \geq 0} x^n = 0$.
\end{defn}

% weglassen?
\begin{beob}
  Ist $A$ ein Integritätsbereich, so ist $0 \in A$ das einzige nilpotente Element in $A$.
\end{beob}

% 3.3. Einheiten

\begin{defn}
  Sei $A$ ein Ring, nicht notwendigerweise kommutativ. \\
  Ein Element $x \in A$ heißt \emph{Einheit}, falls ein $y \in A$ mit $xy = yx = 1$ existiert.
  $A^\times \coloneqq \{ \text{ Einheiten in $A$ } \}$ heißt \emph{Einheitengruppe}. \\
  Der Ring $A$ heißt \emph{Schiefkörper}, falls $0$ die einzige Nicht-Einheit ist. \\
  Falls zusätzlich $A$ kommutativ ist, so heißt $A$ ein \emph{Körper}.
\end{defn}

\begin{beob}
  \begin{itemize}
    \item $x \in A$ ist eine Einheit $\iff$ $(x) = (1)$ $\iff$ $A/(x) = 0$
    \item Einheiten sind regulär.
  \end{itemize}
\end{beob}

% 3.4. Charakterisierung von Körpern

\begin{prop}
  Sei $A$ ein kommutativer Ring.
  Dann sind äquivalent:
  \begin{itemize}
    \item $A$ ist ein Körper.
    \item $A$ besitzt genau zwei Ideale (nämlich $(0)$ und $(1)$).
    \item Ein Ringhomomorphismus $A \to B$ ist genau dann injektiv, wenn~$B$ nicht der Nullring ist.
  \end{itemize}
\end{prop}

% 4. Primideale und maximale Ideale

\begin{defn}
  \begin{itemize}
    \item Ein Ideal $\ppp \subset A$ heißt \emph{Primideal}, falls $1 \not\in \ppp$ und $\fa{a, b \in A} ab \in \ppp \implies a \in \ppp \vee b \in \ppp$.
    \item Ein Ideal $\mmm \subset A$ heißt \emph{maximal}, falls für jedes Ideal $\ppp \subseteq \aaa \subseteq A$ \textit{entweder} $\ppp = \aaa$ oder $\aaa = A$ (nicht beides!) gilt.
  \end{itemize}
\end{defn}

\begin{bspe}
  \begin{itemize}
    \item Jedes Ideal in $\Z$ hat die Form $(m)$ mit $m \in \N$.
    Das Ideal $(m)$ ist genau dann prim, wenn $m=0$ oder $m$ eine Primzahl ist.
    \item Sei $f \in K[x_1, \ldots, x_n]$ ein irred. Polynom.
    Dann ist $(f)$ prim.
\end{itemize}
\end{bspe}

\begin{lem}
  \begin{minipage}[t]{0.85 \linewidth}
    $
      \begin{array}[t]{l c l}
        \text{$\ppp \subseteq A$ ist prim} & \iff & \text{$A/\ppp$ ist ein Integritätsbereich} \\
        \text{$\mmm \subseteq A$ ist maximal} & \iff & \text{$A/\mmm$ ist ein Körper} \\
      \end{array}
    $
  \end{minipage}
\end{lem}

\begin{kor}
  Maximale Ideale sind prim.
\end{kor}

\begin{prop}
  Sei $\aaa \subseteq A$ ein Ideal.
  Folgende Korresp. ist bij. und monoton:
  \[
    \begin{array}{r c l}
      \{ \text{ Primideale $\ppp \subseteq A$ mit $\ppp \supseteq \aaa$ } \} & \leftrightarrow & \{ \text{ Primideale $\qqq \subseteq A/\aaa$ } \} \\
      \ppp & \mapsto & \pi(\ppp) \\
      \pi^{-1}(\qqq) & \mapsfrom & \qqq
    \end{array}
  \]
  Genauso bekommt man eine bijektive, monotone Korrespondenz
  \[
    \begin{array}{r c l}
      \{ \text{ max. Ideale $\mmm \subseteq A$ mit $\mmm \supseteq \aaa$ } \} & \leftrightarrow & \{ \text{ max. Ideale $\nnn \subseteq A/\aaa$ } \}
    \end{array}
  \]
\end{prop}

% 4.2. Maximale Ideale

\begin{prop}
  Ein Ring besitzt genau dann ein maximales Ideal, wenn er nicht der Nullring ist.
\end{prop}

\begin{kor}
  \begin{itemize}
    \item Sei $\aaa \subseteq A$ ein Ideal.
    Dann gibt es genau dann ein maximales Ideal $\ppp \subset A$ mit $\ppp \supseteq \aaa$, wenn $\aaa \neq (1)$.
    \item Ein Element $x \in A$ liegt genau dann in einem maximalen Ideal von~$A$, wenn~$x$ keine Einheit ist.
  \end{itemize}
\end{kor}

% 4.3. Lokale Ringe

\begin{defn}
  Ein \emph{lokaler Ring} ist ein komm. Ring~$A$ mit genau einem max. Ideal $\mmm$.
  Der Körper $F \coloneqq A/\mmm$ heißt \emph{Restklassenkörper} von~$A$.
\end{defn}

\begin{nota}
  Man schreibt "`Sei $(A, \mmm, F)$ ein lokaler Ring."'
\end{nota}

\begin{defn}
  Ein \emph{halblokaler Ring} ist ein kommutativer Ring mit nur endlich vielen maximalen Idealen.
\end{defn}

\begin{lem}
  Sei $\mmm \subset A$ ein Ideal mit $A \setminus \mmm = A^\times$.
  Dann ist $(A, \mmm)$ ein lokaler Ring.
\end{lem}

\begin{prop}
  Sei $\mmm \subset A$ ein maximales Ideal, sodass $1 + x$ für alle $x \in \mmm$ eine Einheit ist.
  Dann ist $A \setminus \mmm = A^\times$, also $(A, \mmm)$ ein lokaler Ring.
\end{prop}

% 5. Das Nil- und das Jacobsonsche Ideal

% 5.1. Das Nilradikal

\begin{prop}
  Die Menge $\nnn \coloneqq \{ \text{ nilpotente Elemente } \} \subseteq A$ ist ein Ideal, das sogenannte \emph{Nilradikal}.
\end{prop}

\begin{bem}
  Der Ring $A/\nnn$ hat außer $0$ keine nilpotenten Elemente.
\end{bem}

\begin{prop}
  Das Nilradikal eines kommutativen Ringes ist der Schnitt aller seiner Primideale.
\end{prop}

% 5.2. Das Jacobsonsche Ideal

\begin{defn}
  Das \emph{Jacobsonsche Ideal} $\jjj \subset A$ ist der Schnitt aller maximalen Ideale von $A$.
\end{defn}

\begin{prop}
  Ein Element $x \!\in\! A$ liegt genau dann im Jacobsonschen Ideal~$\jjj$, wenn $1 - xy$ für alle $y \in A$ eine Einheit ist.
\end{prop}

% 6. Operationen mit Idealen

% 6.1. Summe, Schnitt und Produkt von Idealen

\begin{defn}
  Die \emph{Summe von Idealen} $(\aaa_i)_{i \in I}$ von~$A$ ist das Ideal
  \[
    \sum_{i \in I} \aaa_i \coloneqq \Set{\sum_{k=1}^n x_k}{k \in \N, x_k \in \aaa_{i_k}, i_k \in I}.
  \]
\end{defn}

\begin{bem}
  $\sum_{i \in I} \aaa_i$ ist das kleinste Ideal, das alle $\aaa_i$ umfasst.
\end{bem}

\begin{beob}
  $(x_1) + \ldots + (x_n) = (x_1, \ldots, x_n)$
\end{beob}

\begin{bem}
  Ideale eines Ringes $A$ bilden mit Schnitt und Summe einen vollständigen Verband bezüglich der Inklusionsordnung.
\end{bem}

\begin{defn}
  Das \emph{Produkt zweier Ideale} $\aaa, \bbb \subseteq A$ ist
  \[
    \aaa \bbb \coloneqq \text{von $\Set{ab}{a \in \aaa, b \in \bbb}$ erzeugtes Ideal.}
  \]
\end{defn}

\begin{beob}
  \inlineitem{$\aaa \bbb \subseteq \aaa \cap \bbb$,} \quad
  \inlineitem{$(x_1) \cdot \ldots \cdot (x_n) = (x_1 \cdot \ldots \cdot x_n)$}
\end{beob}

\begin{bsp}
  In $A = \Z$ gilt für $m, n \in \N$ \\
  \inlineitem{$(m) + (n) = (m, n) = (\ggT(m, n))$,} \quad
  \inlineitem{$(m) \cap (n) = (\kgV(m, n))$.}
\end{bsp}

\begin{beob}
  \begin{itemize}
    \item Summe, Schnitt und Produkt von Idealen sind assoziativ.
    \item Summe und Schnitt sind kommutativ. Das Produkt von Idealen ist kommutativ, wenn der Ring kommutativ ist.
    \item Distributivgesetz: \enspace $\aaa (\bbb + \ccc) = \aaa \bbb + \aaa \ccc$
    \item Modularitätsgesetz: \enspace Ist $\aaa \supseteq \bbb$ oder $\aaa \supseteq \ccc$, so folgt
    \[
      \aaa \cap (\bbb + \ccc) = (\aaa \cap \bbb) + (\aaa \cap \ccc).
    \]
  \end{itemize}
\end{beob}

\begin{defn}
  Zwei Ideale $\aaa, \bbb \subseteq A$ heißen \emph{koprim}, falls $\aaa + \bbb = (1)$.
\end{defn}

\begin{bsp}
  In $A = \Z$ gilt: \enspace
  $(m)$, $(n)$ sind koprim $\iff$ $\ggT(m, n) = 1$
\end{bsp}

\begin{prop}
  Seien $\aaa_1, \ldots \aaa_n \subseteq A$ paarweise koprime Ideale.
  Dann gilt
  \[
    \bigcap_{i=1}^n \aaa_i = \prod_{i=1}^n \aaa_i.
  \]
\end{prop}

% 6.2. Direkte Produkte

\begin{defn}
  Das \emph{direkte Produkt} einer Familie $(A_i)_{i \in I}$ von Ringen ist der Ring $\prod_{i \in I} A_i \coloneqq \{ (a_i \in A_i)_{i \in I} \}$ mit kmpnntnwsr Verknüpfung.
\end{defn}

\begin{bem}
  Das direkte Produkt ist das kategorienth. Produkt in $\Ring$.
\end{bem}

\begin{prop}
  Seien $\aaa_1, \ldots \aaa_n \subseteq A$ Ideale.
  Dann ist
  \[
    \phi : A \to \prod_{i=1}^n (A/\aaa_i), \quad x \mapsto ([x], \ldots, [x])
  \]
  genau dann surjektiv, wenn die Ideale $\aaa_i$ paarweise koprim sind.
\end{prop}

\begin{bem}
  %Es gilt $\ker \phi = \bigcap_{i=1}^n \aaa_i$.
  Der Ringhomomor. $\phi$ ist genau dann injektiv, wenn $\bigcap_{i=1}^n \aaa_i = 0$.
\end{bem}

% 6.3. Ideale in Primidealen

\begin{prop}
  Seien $\ppp_1, \ldots, \ppp_n \subseteq A$ Primideale und $\aaa \subseteq A$ ein Ideal. \\
  Gilt $\aaa \subseteq \bigcup_{i=1}^n \ppp_i$, so gibt es ein $j \in \{ 1, \ldots, n \}$ mit $\aaa \subseteq \ppp_j$.
\end{prop}

\begin{prop}
  Seien $\aaa_1, \ldots, \aaa_n \subseteq A$ Ideale und $\ppp \subseteq A$ ein Primideal. \\
  Gilt $\ppp \supseteq \bigcap_{i=1}^n \aaa_i$, so gibt es ein $j \in \{ 1, \ldots, n \}$ mit $\ppp \supseteq \aaa_j$.
\end{prop}

% 6.4. Der Idealquotient

\begin{defn}
  Seien $\aaa, \bbb \subseteq A$ zwei Ideale.
  Der \emph{Idealquotient} von $\aaa$ nach $\bbb$ ist das Ideal $(\aaa : \bbb) \coloneqq \Set{x \in A}{x \bbb \subseteq \aaa}$.
\end{defn}

\begin{nota}
  \inlineitem{$(x : \bbb) \coloneqq ((x) : \bbb)$,} \quad
  \inlineitem{$(\aaa : y) \coloneqq (\aaa : (y))$}
\end{nota}

\begin{defn}
  Der \emph{Annulator} eines Ideals $\bbb \subseteq A$ ist
  $\ann(\bbb) \coloneqq (0 : \bbb)$.
\end{defn}

\begin{lem}
  \inlineitem{$\aaa \subseteq (\aaa : \bbb)$} \quad
  \inlineitem{$(\aaa : \bbb) \bbb \subseteq \aaa$} \quad
  \inlineitem{$((\aaa : \bbb) : \ccc) = (\aaa : \bbb \ccc)$} \\[2pt]
  \inlineitem{$(\bigcap_{i \in I} \aaa_i : \bbb) = \bigcap_{i \in I} (\aaa_i : \bbb)$} \quad
  \inlineitem{$(\aaa : \sum_{i \in I} \bbb) = \bigcap_{i \in I} (\aaa : \bbb_i)$}
\end{lem}

% 6.5. Das Wurzelideal

\begin{defn}
  Das \emph{Wurzelideal} eines Ideals $\aaa \subseteq A$ ist das Ideal
  \[
    \sqrt{\aaa} \coloneqq \Set{x \in A}{\ex{n \in \N} x^n \in \aaa}.
  \]
\end{defn}

\begin{bem}
  Das Nilradikal ist $\sqrt{(0)}$, das Wurzelideal des Nullideals. \\
  Es gilt $\sqrt{\aaa} = \pi^{-1}(\sqrt{(0)})$ mit $\pi : A \to A/\aaa, \enspace x \mapsto [x]$.
\end{bem}

\begin{lem}
  \inlineitem{$\sqrt{\aaa} \supseteq \aaa$} \enspace
  \inlineitem{$\sqrt{\aaa^n} = \sqrt{\aaa}$ für $n \!\geq\! 1$} \enspace
  \inlineitem{$\sqrt{\aaa} = (1) \iff \aaa = (1)$} \\[2pt]
  \inlineitem{$\sqrt{\sqrt{\aaa}} = \sqrt{\aaa}$} \enspace
  \inlineitem{$\sqrt{\aaa \bbb} = \sqrt{\aaa \cap \bbb} = \sqrt{\aaa} \cap \sqrt{\bbb}$} \enspace
  \inlineitem{$\sqrt{\aaa + \bbb} = \sqrt{\sqrt{\aaa} + \sqrt{\bbb}}$}
\end{lem}

\begin{defn}
  Ein Ideal $\aaa \subseteq A$ heißt \emph{Wurzelideal}, falls $\aaa = \sqrt{\aaa}$.
\end{defn}

\begin{prop}
  Das Wurzelideal von $\sqrt{\aaa}$ ist der Schnitt aller Primideale von~$A$, die $\aaa$ enthalten.
\end{prop}

\begin{prop}
  $\{ \text{ Nullteiler von $A$ } \} = \bigcup_{\mathclap{x \in A \setminus \{ 0 \}}} \sqrt{\ann(x)}$
\end{prop}

\begin{lem}
  $\sqrt{\aaa}$ und $\sqrt{\bbb}$ koprim $\implies$ $\aaa$ und $\bbb$ koprim
\end{lem}

% 7. Erweiterungen und Kontraktionen von Idealen

% 7.1. Kontraktionen

\begin{defn}
  Sei $\phi : A \to B$ ein Homomorphismus komm. Ringe. \\
  Die \emph{Kontraktion} von $\bbb \subseteq B$ (bzgl. $\phi$) ist das Ideal $A \cap \bbb \coloneqq \phi^{-1}(\bbb)$.
\end{defn}

\begin{bem}
  Es wird also $\phi$ in der Notation unterdrückt.
  Falls $\phi$ die Inklusion eines Unterrings ist, so ist $A \cap \bbb$ wörtlich zu verstehen.
\end{bem}

\begin{beob}
  $A \cap \bbb = \ker(A \to B \to B/\bbb)$
\end{beob}

\begin{lem}
  Ist $\qqq \subseteq B$ ein Primideal, so auch $A \cap \qqq \subseteq A$.
\end{lem}

\begin{acht}
  %Für $\phi : \Z \hookrightarrow \Q$ und $\mmm = (0) \subset \Q$ gilt: $\mmm$ ist maximal, $\Z \cap \mmm$ aber nicht.
  Die Kontraktion max. Ideale ist i.\,A. nicht maximal!
\end{acht}

% TODO: Distributiv- und Modularitätsgesetz für unendliche Summen?
% TODO: was ist ein faktorieller Ring?
% TODO: Umkehrung von Proposition 4.19 im Skript

% 7.2. Erweiterungen

\begin{defn}
  Sei $\phi : A \to B$ ein Homomorphismus komm. Ringe. \\
  Die \emph{Erweiterung} von $\aaa \subseteq A$ (bzgl. $\phi$) ist das Ideal $B \aaa \coloneqq (\phi(\aaa))$, das von $\phi(\aaa)$ erzeugte Ideal.
\end{defn}

\begin{bem}
  Ist $\phi$ die Inklusion eines Unterrings, so ist $B \aaa$ tatsächlich die Menge der $B$-Linearkombinationen von Elementen in~$\aaa$.
\end{bem}

\begin{bem}
  Die Erweiterung eines Primideals ist i.\,A. nicht mehr prim.
\end{bem}

% 7.3. Operationen mit Erweiterungen und Kontraktionen

\begin{prop}
  Sei $\phi : A \to B$ ein Homomorphismus komm. Ringe. \\
  Die Erweiterung und Kontraktion von Idealen (bzgl. $\phi$) bilden eine Galois-Verbindung, \dh{} für Ideale $\aaa \subseteq A$ und $\bbb \subseteq B$ gilt
  \[
    B \aaa \subseteq \bbb \iff \aaa \subseteq A \cap \bbb.
  \]
  Äquivalent dazu sind Erw. und Kontraktion monoton und es gelten
  \[
    \aaa \subseteq A \cap (B \aaa)
    \quad \text{und} \quad
    \bbb \supseteq B (A \cap \bbb).
  \]
  Außerdem folgt aus den Eigenschaften einer Galois-Verbindung, dass
  \[
    B \aaa = B (A \cap (B \aaa))
    \quad \text{und} \quad
    A \cap \bbb = A \cap (B (A \cap \bbb)).
  \]
  Damit induzieren Erweiterung und Kontraktion eine bijektive ordnungserhaltende Korrespondenz zwischen den kontrahierten Idealen von~$A$ und den erweiterten Idealen von~$B$.
\end{prop}

\begin{lem}
  Für Ideale $\aaa, \aaa_1, \aaa_2 \subseteq A$ und $\bbb, \bbb_1, \bbb_2 \subseteq B$ gilt
  \begin{itemize}
    \miniitem{0.45 \linewidth}{$B \sqrt{\aaa} \subseteq \sqrt{B \aaa}$}
    \miniitem{0.51 \linewidth}{$A \cap \sqrt{\bbb} = \sqrt{A \cap \bbb}$} \\
    \miniitem{0.45 \linewidth}{$B (\aaa_1 + \aaa_2) = B \aaa_1 + B \aaa_2$}
    \miniitem{0.51 \linewidth}{$A \cap (\bbb_1 + \bbb_2) \supseteq A \cap \bbb_1 + A \cap \bbb_2$} \\
    \miniitem{0.45 \linewidth}{$B (\aaa_1 \cap \aaa_2) \subseteq B \aaa_1 \cap B \aaa_2$}
    \miniitem{0.51 \linewidth}{$A \cap (\bbb_1 \cap \bbb_2) = (A \cap \bbb_1) \cap (A \cap \bbb_2)$} \\
    \miniitem{0.45 \linewidth}{$B (\aaa_1 \aaa_2) = (B \aaa_1) (B \aaa_2)$}
    \miniitem{0.51 \linewidth}{$A \cap (\bbb_1 \bbb_2) \supseteq (A \cap \bbb_1) (A \cap \bbb_2)$} \\
    \miniitem{0.45 \linewidth}{$B (\aaa_1 : \aaa_2) \subseteq (B \aaa_1 : B \aaa_2)$}
    \miniitem{0.51 \linewidth}{$A \cap (\bbb_1 : \bbb_2) \subseteq (A \cap \bbb_1 : A \cap \bbb_2)$}
  \end{itemize}
\end{lem}

\end{document}
