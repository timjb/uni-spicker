\documentclass{cheat-sheet}

\pdfinfo{
  /Title (Zusammenfassung Numerik von partiellen Differentialgleichungen)
  /Author (Tim Baumann)
}

% Kleinere Klammern
\delimiterfactor=701

\newcommand{\Cont}{\mathcal{C}} % Menge der stetigen/diff'baren Funktionen
\newcommand{\clos}[1]{\overline{#1}} % topologischer Abschluss

\begin{document}

\maketitle{Zusammenfassung Numerik von PDEs}

% Vorlesung vom 13. Oktober 2015

% 1. Einführung

% Ausgelassen: Notationen

% 1.1. Klassifikation von partiellen DGLn

\begin{defn}
  Sei $\Omega \subseteq \R^n$ offen.
  Eine DGL der Form
  \[ F(x, u, Du, \ldots, D^k u) = 0 \]
  heißt \emph{partielle DGL/PDE} der Ordnung $k \geq 1$, wobei
  \[ F : \Omega \times \R \times \R^n \times \ldots \times \R^{n^k} \to \R \]
  eine gegebene Funktion und $u : \Omega \to \R$ gesucht ist.
\end{defn}

\begin{defn}[\emph{Klassifikation von PDEs}]\mbox{}\\
  \begin{itemize}
    \item Die PDE heißt \emph{linear}, wenn sie die Form
    \[ \sum_{\abs{\alpha} \leq k} a_\alpha(x) D^\alpha u = f(x) \]
    mit Funktionen $a_\alpha, f : \Omega \to \R$ besitzt.
    \item Die PDE heißt \emph{semilinear}, wenn sie die Form
    \[ \sum_{\abs{\alpha} = k} a_\alpha(x) D^\alpha u + a_0(x, u, D_u, \ldots, D^{k-1} u) = 0 \]
    besitzt, wobei $a_\alpha : \Omega \to \R$ und $a_0 : \Omega \times \R \times \R^n \times \ldots \times \R^{n^k} \to \R$ gegeben sind.
    \item Die PDE heißt \emph{quasilinear}, wenn sie die Form
    \[ \sum_{\abs{\alpha} = k} a_\alpha(x, u, Du, \ldots, D^{k-1} u) D^\alpha u + a_0(x, u, D_u, \ldots, D^{k-1} u) = 0 \]
    hat, wobei $a_\alpha, a_0 : \Omega \times \R \times \R^n \times \ldots \times \R^{n^k}$ gegeben sind.
    \item Die PDE heißt \emph{nichtlinear}, falls die Ableitungen der höchsten Ordnung nicht linear vorkommen.
  \end{itemize}
\end{defn}

% Ausgelassen: Beispiele: Poisson-, Laplace-, Wärmeleitungs-, Wellengleichung sowie Navier-Stokes-Gleichung

\begin{defn}
  Sei $\Omega \subseteq \R^n$ offen und $F : \Omega \times \R \times \R^n \times \R^{n \times n} \to \R$ eine gegebene Funktion.
  Eine PDE der Form
  \[ F(x, u, \partial_{x_1} u, \ldots, \partial_{x_n} u, \partial_{x_1} \partial_{x_1} u, \ldots, \partial{x_1} \partial_{x_n} u, \ldots, \partial_{x_n} \partial_{x_n} u) = 0 \]
  heißt \emph{PDE zweiter Ordnung}.
\end{defn}

\begin{nota}
  $p_i \coloneqq \partial_{x_i} u$, $p_{ij} \coloneqq \partial^2_{x_i x_j} u$
  \[
    M(x) \coloneqq \begin{pmatrix}
      \tfrac{\partial F}{\partial p_{11}} & \hdots & \tfrac{\partial F}{\partial p_{1n}} \\
      \vdots && \vdots \\
      \tfrac{\partial F}{\partial p_{n1}} & \hdots & \tfrac{\partial F}{\partial p_{nn}}
    \end{pmatrix} = M(x)^{T}.
  \]
\end{nota}

\begin{defn}[\emph{Typeneinteilung für PDEs der 2. Ordnung}]\mbox{}\\
  Obige PDE zweiter Ordnung heißt
  \begin{itemize}
    \item \emph{elliptisch} in $x$, falls die Matrix $M(x)$ positiv o. definit ist.
    \item \emph{parabolisch} in $x$, falls genau ein EW von $M(x)$ gleich null ist und alle anderen dasselbe Vorzeichen haben.
    \item \emph{hyperbolisch} in $x$, falls genau ein EW ein anderes Vorzeichen als die anderen EWe hat.
  \end{itemize}
\end{defn}

% Ausgelassen: Beispiele

% Vorlesung vom 19. Oktober 2015

% 2. Klassische Lösungstheorie für elliptische PDEs

\begin{defn}
  Sei $\Omega \subset \R^n$ offen, zusammenhängend und beschränkt.
  \begin{itemize}
    \item $\Cont(\clos{\Omega}, \R^m)$ ist der Raum aller auf $\clos{\Omega}$ stetigen Funktionen $u : \Omega \to \R^m$, $\Cont(\clos{\Omega}) \coloneqq \Cont(\clos{\Omega}, \R)$, $\norm{u}_{\Cont(\clos{\Omega}, \R^m)} = \sup_{x \in \clos{\Omega}} \norm{u(x)}$.
    (Damit wird $(\Cont(\clos{\Omega}, \R^m), \norm{\blank}_{\Cont(\clos{\Omega}, \R^m)})$ zu einem Banach-Raum.)
    \item $\Cont^k(\clos{\Omega}, \R^m)$, $k \in \N$ ist der Raum aller auf $\Omega$ $k$-mal stetig diff'baren Funktionen $u : \Omega \to \R^m$, die zusammen mit ihren Ableitungen bis zur Ordnung $k$ stetig auf $\clos{\Omega}$ fortgesetzt werden können.
    \[ \norm{u}_{\Cont^k(\overline{\Omega}, \R^m)} = \sum_{\abs{\alpha} \leq k} \norm{D^\alpha u}_{\Cont(\overline{\Omega}, \R^m)} \]
    % ist wieder ein Banachraum
    \item $\Cont^{0,\alpha}(\clos{\Omega}, \R^m) = \Set{u \in \Cont(\clos{\Omega}, \R^m)}{H_\alpha(u, \clos{\Omega}) \coloneqq \sup_{x, y \in \clos{\Omega}, x \neq y} \frac{\norm{u(x)-u(y)}}{\norm{x-y}^\alpha} < \infty}$ heißt \emph{Raum aller gleichmäßig Hölder-stetigen Funktionen zum Exponent $\alpha \in \cointerval{0}{1}$}.
  \end{itemize}
\end{defn}

\begin{bem}
  \begin{itemize}
    \item $\Cont^{0,1}(\clos{\Omega}, \R^m)$ heißt \emph{Raum der Lipschitz-stetigen Funktionen},
    \item Jede Hölder-stetige Funktion ist gleichmäßig stetig
    \item $\Cont^{k,\alpha}(\clos{\Omega}, \R^m) \coloneqq \Set{u \in \Cont^k(\clos{\Omega}, \R^m)}{\fa{\abs{\gamma} = k} D^\gamma u \in \Cont^{0, \alpha}(\clos{\Omega}, \R^m)}$
    \[ \norm{u}_{\Cont^{k,\alpha}(\clos{\Omega}, \R^m)} \coloneqq \norm{u}_{\Cont^k(\clos{\Omega}, \R^m)} + \sum_{\abs{\gamma} = k} H_\alpha(D^\gamma u, \clos{\Omega}) \]
    Damit wird $(\Cont^{k,\alpha}(\clos{\Omega}, \R^m), \norm{\blank}_{\Cont^{k,\alpha}(\clos{\Omega}, \R^m)})$ zu einem Banach-Raum.
  \end{itemize}
\end{bem}

\begin{bsp}
  Betrachte $u(x) = \abs{x}^\beta$ auf $\ointerval{-1}{1} = \Omega$. Dann ist
  \[ \frac{\abs{u(x) - u(0)}}{\abs{x - 0}^\alpha} = \abs{X}^{\beta - \alpha} \]
  genau dann beschränkt, falls $\beta \geq \alpha$. In dem Fall ist $u$ Hölder-stetig zum Exponent $\alpha$.
\end{bsp}

\begin{defn}
  Sei $\Omega \subset \R^n$ offen, zusammenhängend und beschränkt.
  Das Gebiet $\Omega$ gehört zur Klasse $\Cont^{k,\alpha}$, wenn in jedem Punkt $x \in \partial \Omega$ eine Umgebung in $\partial \Omega$ existiert, die sich in einem geeigneten Koordinatensystem als ein Graph einer Funktion aus $\Cont^{k,\alpha}$ darstellen lässt und $\Omega$ liegt lokal immer auf einer Seite von $\partial \Omega$.
\end{defn}

% Ausgelassen: Beispiele und Gegenbeispiele von Gebieten aus $\Cont^{k,\alpha}$

\begin{satz}[Gauß'scher Integralsatz]
  Sei $\Omega \subset \R^n$ ein Lipschitz-Gebiet und $u \in \Cont(\clos{\Omega}, \R^n) \cap \Cont^1(\Omega, \R^n)$.
  Dann gilt
  \[ \Int{\Omega}{}{\div u}{x} \Int{\Omega}{}{\sum_{i=1}^n \tfrac{\partial u_i}{\partial x_i}}{x} = \Int{\partial \Omega}{}{\sum_{i=1}^n u_i \nu_i}{\rho(x)} = \Int{\partial \Omega}{}{u \cdot \nu}{\rho(x)}, \]
  wobei $\nu = (\nu_1, \ldots, \nu_n)$ der äußere Normalenvektoran an den RAnd von $\Omega$ ist.
\end{satz}

% TODO: Klammer um die Gleichungen, Beschriftung: "(RWP)"
\begin{align*}
  \mathcal{L} u &= f \text{in $\Omega$ (PDE)} \\
  \mathcal{R} u &= g \text{auf $\partial \Omega$ (Randbedingung)}
\end{align*}
wobei $\mathcal{L} u = - \sum_{i,j=1}^n a_{ij}(x) \frac{\partial^2 u}{\partial x_i \partial x_j} + \sum_{i=1}^n b_i(x) \frac{\partial u}{\partial x_i} + c(x) u$ (linearer Differentialoperator)
mit $a_{ij}, b_i, c, f : \Omega \to \R$, $g : \partial \Omega \to \R$, $A(x) \coloneqq (a_{ij}(x))_{i,j}$ symmetrisch. \\

Randbedingungen:
\begin{itemize}
  \item \emph{Dirichlet-Randbedingung}: $u = g$ auf $\partial \Omega$
  \item \emph{Neumann-Randbedingung}: $(A(x) \nabla u) \cdot \nu = g$ auf $\partial \Omega$
  \item \emph{Robin-Randbedingung}: $A(x) \nabla u \cdot \nu + \delta u = g$ auf $\partial \Omega$
\end{itemize}

\begin{bem}
  Man kann auch auf verschiedenen Teilstücken des Randes verschiedene Bedingungen stellen.
\end{bem}

\begin{bem}
  Falls $a_{ij}$ differenzierbar sind, kann $\mathcal{L}$ in Divergenzform geschrieben werden:
  \[
    \mathcal{L} u
    = - \sum_{i,j=1}^n a_{ij}(x) \frac{\partial^2 u}{\partial x_i \partial x_j} + \sum_{i=1}^n b_i(x) \frac{\partial u}{\partial x_i} + c(x) u 
    = - \sum_{i,j=1}^n \left( \frac{\partial}{\partial x_j} \left( a_{ij}(x) \frac{\partial u}{\partial x_i} \right) - \left( \tfrac{\partial}{\partial x_j} a_{ij}(x) \right) \frac{\partial u}{\partial x_i} \right) + \sum_{i=1}^n b_i(x) \frac{\partial u}{\partial x_i} + c(x) u
    = - \sum_{i,j=1}^n \frac{\partial}{\partial x_j} \left( a_{ij}(x) \frac{\partial u}{\partial x_i} \right) + \sum_{i=1}^n \left( \left( \sum_{j=1}^n \frac{\partial}{\partial x_j} a_{ij}(x) \right) + b_i(x) \right) \frac{\partial u}{\partial x_i} + c(x) u = - \div(A(x) \nabla u) + \tilde{b}(x) \cdot \nabla u + c(x) u
  \]
\end{bem}
% \tilde{b}_i(x) ist der Term in Klammern im mittleren Term

% TODO: \nabla ~> \grad

% TODO: Divergenz-makro

Voraussetzungen:

\begin{itemize}
  \item $\mathcal{L}$ ist \emph{gleichmäßig elliptisch}, \dh{}
  \[ \exists{\lambda_0 > 0} \fa{\xi \in \R^n} \fa{x \in \Omega} \xi^T A(x) \xi \geq \lambda_0 \norm{\xi}^2 \]
  Dabei heißt $\lambda_0$ \emph{Elliptizitätskonstante}
  Bemerkung: $\mathcal{L} = f$ ist elliptisch auf $\Omega$ $\iff$ $A(x) > 0$ (spd) für alle $x \in \Omega$
  \item $a_{ij}, b_i, c, f \in \Cont(\clos{\Omega})$, $g \in \Cont(\partial \Omega)$
\end{itemize}

\begin{defn}
  Eine Funktion $u \in \Cont^2(\Omega) \cap \Cont(\clos{\Omega})$ heißt \emph{klassische Lsg} vom (RWP) mit $Ru \coloneqq u$, wenn die beiden Gleichungen in (RWP) in jedem Punkt von $\Omega$ bzw. des Randes $\partial \Omega$ erfüllt sind.
\end{defn}

% 2.1
\begin{satz}[Maximumsprinzip]
  Sei $\Omega \subset \R^n$ offen, zusammenhängend und beschränkt.
  Sei $u \in \Cont^2(\omega) \cap \Cont(\clos{\Omega})$ eine Lösung vom (RWP) und $f \leq 0$ in $\Omega$ und $c \equiv 0$. Dann nimmt $u$ sein Maximum auf dem Rand $\partial \Omega$ an, \dh{}
  \[ \sup_{x \in \clos{\Omega}} u(x) = \sup_{x \in \partial \Omega} u(x) = \sup_{x \in \partial \Omega} g(x) \]
\end{satz}

\end{document}