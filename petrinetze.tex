\documentclass{cheat-sheet}

\pdfinfo{
  /Title (Zusammenfassung Petrinetze)
  /Author (Tim Baumann)
}


\newcommand{\transition}{\square} % Transition
\newcommand{\place}{\bigcirc} % Stelle, Platz
\newcommand{\preset}[1]{{}^\bullet{#1}} % Vorbereich
\newcommand{\postset}[1]{{#1}^\bullet} % Nachbereich
\newcommand{\activeTransition}[1]{[{#1}\rangle} % aktive Transition
\DeclareMathOperator{\FS}{FS} % Menge der Schaltfolgen (firing sequences)
\newcommand{\ReachabilityGraph}{\mathfrak{R}} % Erreichbarkeitsgraph
\newcommand{\StepReachabilityGraph}{\mathfrak{SR}} % Schritt-Erreichbarkeitsgraph
\DeclareMathOperator{\Parikh}{Parikh} % Parikh-Bild
\newcommand{\inferrule}[2]{\frac{{#1}}{{#2}}} % logische Inferenzregel
\newcommand{\Markings}{\mathfrak{M}} % Menge von Markierungen
\DeclareMathOperator{\Fin}{Fin} % Menge von Endzuständen
\newcommand{\Lang}{\mathfrak{L}} % Sprachen von beschrifteten Netzen mit Endmarkierung
\newcommand{\fin}{\mathrm{fin}} % endlich
\newcommand{\Multisets}{\mathfrak{M}} % Multimengen
\DeclareMathOperator{\StepSequences}{SS} % Schrittfolgen

% Kleinere Klammern
\delimiterfactor=701

\setlength{\tabcolsep}{2pt}

\begin{document}

\raggedcolumns % stretche Inhalt nicht über die gesamte Spaltenhöhe

\maketitle{Zusammenfassung Petrinetze}

% §2. Grundbegriffe

% 2.1
\begin{defn}
  Ein \emph{Netzgraph} ist ein Tripel $(S, T, W)$, wobei~$S$ und~$T$ disjunkte, endliche Mengen sind und $W : S \times T \cup T \times S \to \N$.
  Dadurch ist ein gerichteter, gewichteter, bipartiter Graph mit Kantenmenge $F = \Set{(x, y)}{W(x, y) \neq 0}$ gegeben.
\end{defn}

\begin{center}
  \begin{tabular}{r l l}
    Notation & Bezeichnung & Symbol \\ \hline
    $t \in T$ & Transition & $\transition$ \\
    $s \in S$ & Stelle, Platz & $\place$ \\
    $(x, y) \in F$ & Kante & $\xrightarrow{\enspace}$ falls $W(x, y) = 1$ \\
    && $\xrightarrow{w}$ falls $w \coloneqq W(x, y) > 1$
  \end{tabular}
\end{center}

\begin{defn}
  Sei $x \in S \cup T$.
  \begin{itemize}
    \item $\preset{x} \coloneqq \Set{y}{(y, x) \in F}$ heißt \emph{Vorbereich} von~$x$ und
    \item $\postset{x} \coloneqq \Set{y}{(x, y) \in F}$ heißt \emph{Nachbereich} von~$x$.
    \item $x$ heißt \emph{isoliert}, falls $\preset{x} \cup \postset{x} = \emptyset$.
    \item $x$ heißt \emph{vorwärts-verzweigt}, falls $\card{\postset{x}} \geq 2$ 
    \item $x$ heißt \emph{rückwärts-verzweigt}, falls $\card{\preset{x}} \geq 2$ 
  \end{itemize}
\end{defn}

\begin{defn}
  $(x, y) \in S \times T \cup T \times S$ bilden eine \emph{Schlinge} falls $(x, y) \in F$ und $(y, x) \in F$.
\end{defn}

\begin{defn}
  Eine \emph{Markierung} ist eine Abbildung $M : S \to \N$. \\
  Eine Teilmenge $S' \subseteq S$ heißt \emph{markiert} unter~$M$, falls $\ex{s \in S'} M(s') > 0$, andernfalls \textit{unmarkiert}. \\
  Ein Element $s \in S$ heißt \textit{(un-)markiert}, falls $\{ s \} \subseteq S$ es ist.
\end{defn}

\begin{nota}
  $\Markings(S) \coloneqq \{ M : S \to \N \}$
\end{nota}

\begin{defn}
  Ein \emph{Petrinetz} $N = (S, T, W, M_N)$ besteht aus
  \begin{itemize}
    \item einem Netzgraphen $(S, T, W)$ und
    \item einer \textit{Anfangsmarkierung}~$M_N : S \to \N$.
  \end{itemize}
\end{defn}

\begin{nota}
  Für eine feste Transition $t \in T$ ist
  \[
    t^{-} : S \to \N, \enspace s \mapsto W(s, t), \qquad
    t^{+} : S \to \N, \enspace s \mapsto W(t, s)
  \]
\end{nota}

% 2.2
\begin{defn}
  Eine Transition $t \in T$ heißt \emph{aktiviert} unter einer Markierung~$M$, notiert $M \activeTransition{t}$, falls
  \[
    \fa{s \in S} W(s, t) \leq M(s) \iff
    t^{-} \leq M.
  \]
  Ist~$t$ aktiviert, so kann~$t$ \textit{schalten} und es entsteht die \textit{Folgemarkierung} $M' \coloneqq M + \Delta t$, wobei
  \[
    \Delta t : S \to \Z, \enspace s \mapsto W(t, s) - W(s, t).
  \]
\end{defn}

\begin{nota}
  $M \activeTransition{t} M'$
\end{nota}

\begin{defn}
  Für $w = t_1 \cdots t_n \in T^{*}$ und Markierungen~$M$ und~$M'$ gilt
  \[ M \activeTransition{w} M' \coloniff M \activeTransition{t_1} M_1 \activeTransition{t_2} \cdots \activeTransition{t_{n-1}} M_{n-1} \activeTransition{t_n} M' \]
  für (eindeutig bestimmte) Markierungen $M_1, \ldots, M_{n-1}$. \\
  Ein Wort $w \in T^{*}$ heißt \emph{Schaltfolge} (\textit{firing sequence}) von~$N$, notiert $M_N \activeTransition{w}$, falls $\ex{M'} M_N \activeTransition{w} M'$.
\end{defn}

\begin{nota}
  \begin{minipage}[t]{0.8 \linewidth}
    $\activeTransition{M} \coloneqq \Set{M'}{\ex{w \in T^{*}} M \activeTransition{w} M'}$ \\
    $\FS(N) \coloneqq \Set{w \in T^{*}}{M_N \activeTransition{w}}$ \enspace
    für ein Petrinetz~$N$
  \end{minipage}
\end{nota}

\begin{defn}
  $M'$ heißt \emph{erreichbar} von~$M$, falls $M' \in \activeTransition{M}$.
\end{defn}

\begin{defn}
  $w \in T^\omega$ heißt \emph{unendliche Schaltfolge} von~$N$, falls alle endlichen Präfixe von~$w$ Schaltfolgen von~$N$ sind.
\end{defn}

\begin{defn}
  Der \emph{Erreichbarkeitsgraph} $\ReachabilityGraph(N)$ zu~$N$ besitzt die Knoten~$\activeTransition{M_N}$ und die Kanten $\Set{(M, M')}{\ex{t} M \activeTransition{t} M'}$.
\end{defn}

% 2.5
\begin{defn}
  Für $w = a_1 \cdots a_n \in A^{*}$ ist $\Parikh(w) : A \to \N, \enspace a \mapsto \card{i}{a_i = a}$.
\end{defn}

% 2.6
\begin{lem}
  In $M \activeTransition{w} M'$ hängt~$M'$ nur von~$M$ und $\Parikh(w)$ ab, genauer
  \[
    M' = M + {\sum}_{t \in T} \Parikh(w)(t) \cdot \Delta t.
  \]
\end{lem}

% 2.7
\begin{lem}
  $M_1 \activeTransition{w} M_2 \implies M + M_1 \activeTransition{w} M + M_2$
\end{lem}

\TODO{Satz 2.8}

% 2.10
\begin{lem}
  Sei $N$ ein Petri-Netz.
  Dann gilt:
  \begin{itemize}
    \item $\FS(N)$ ist \textit{präfix-abg.}, \dh{} $w = v u \in \FS(N) \implies v \in \FS(N)$.
    \item Ist $\activeTransition{M_N}$ endlich, so ist $\FS(N)$ regulär.
  \end{itemize}
\end{lem}

% 2.11
\begin{defn}
  Ein \emph{beschriftetes Petrinetz} $N = (S, T, W, M_N, \ell)$ best. aus
  \begin{itemize}
    \item einem Petrinetz $(S, T, W, M_N)$ und
    \item einer Transitionsbeschriftung (\textit{labelling}) $\ell : T \to \Sigma \cup \{ \lambda \}$, wobei~$\Sigma$ eine Menge von \textit{Aktionen} ist.
  \end{itemize}
\end{defn}

\begin{sprechweise}
  $t \in T$ mit $\ell(t) = \lambda$ heißt \textit{intern} oder \textit{unsichtbar}.
\end{sprechweise}

\begin{nota}
  \begin{minipage}[t]{0.8 \linewidth}
    Für $t \in T^{*}$ ist $\ell(w) \coloneqq \ell(t_1) \cdots \ell(t_n) \in \Sigma^{*}$. \\
    Dabei wird $\lambda$ als das leere Wort in $\Sigma^{*}$ aufgefasst.
  \end{minipage}
\end{nota}

\begin{defn}
  Mit $t \in T$, $w \in T^{*}$ und Markierungen $M$, $M'$ ist definiert:
  \[
    \inferrule
      {M \activeTransition{t} M'}
      {M \activeTransition{\ell(t)} M'} \quad
    \inferrule
      {M \activeTransition{t}}
      {M \activeTransition{\ell(t)}} \quad
    \inferrule
      {M \activeTransition{w} M'}
      {M \activeTransition{\ell(w)} M'} \quad
    \inferrule
      {M \activeTransition{w}}
      {M \activeTransition{\ell(w)}}
  \]
\end{defn}

\begin{defn}
  Die \emph{Sprache} eines beschrifteten Netzes~$N$ ist
  \[
    L(N) \coloneqq \Set{v \in \Sigma^{*}}{M_n \activeTransition{v}}.
  \]
\end{defn}

% 2.12
\begin{defn}
  Ein \emph{beschriftetes Netz mit Endmarkierung} ist ein Tupel $N = (S, T, W, M_N, \ell, \Fin)$ wobei
  \begin{itemize}
    \item $(S, T, W, M_N, \ell)$ ein beschriftetes Netz und
    \item $\Fin \subseteq \Markings(S)$ eine endliche Menge ist.
  \end{itemize}
  Die entspr. Sprache ist $L_\fin(N) \coloneqq \Set{v \in \Sigma^{*}}{\ex{M \in \Fin} M_N \activeTransition{v} M}$.
\end{defn}

\begin{nota}
  $\Lang^\lambda \coloneqq \Set{L_\fin(N)}{N \text{ beschr. Netz mit Endmarkierung}}$ \\
  $\Lang \coloneqq \Set{L_\fin(N)}{N \text{ beschr. Netz mit Endmark. ohne interne Trans.}}$
\end{nota}

% 2.13
\begin{satz}
  $\{ \text{ reguläre Sprachen } \} \subseteq \Lang$
\end{satz}

\subsection{Nebenläufigkeit I}

% 2.14
\begin{defn}
  Eine Multimenge über~$X$ ist eine Funktion $\mu : X \to \N$.
\end{defn}

\begin{nota}
  \begin{minipage}[t]{0.8 \linewidth}
    $\Multisets(X) \coloneqq \{ \mu : X \to \N \}$ \\
    $\mu_Y \in \Multisets(X), x \mapsto \card{\Set{\star}{x \in Y}}$ für $Y \subset X$, \\
    $\emptyset \coloneqq \mu_\emptyset \in \Multisets(X)$, \enspace
    $\mu_x \coloneqq \mu_{\{ x \}} \in \Multisets(X)$ für $x \in X$
  \end{minipage}
\end{nota}

% 2.15
\begin{defn}
  Ein \emph{Schritt}~$\mu$ ist eine Multimenge $\mu \neq \emptyset \in \Multisets(T)$. \\
  Der Schritt~$\mu$ ist \emph{aktiviert} unter~$M$, notiert $M \activeTransition{\mu}$, falls
  \[
    \fa{s \in S} \mu^{-}(s) \coloneqq {\sum}_{t \in T} \mu(t) W(s, t) \leq M(s).
  \]
  Durch \textit{Schalten} von~$\mu$ entsteht die Folgemarkierung $M' \in \Markings(S)$ mit
  \[
    M'(s) = M(s) + {\sum}_{t \in T} \mu(t) \cdot (W(t, s) - W(s, t)).
  \]
\end{defn}

\begin{bem}
  Analog wird verallgemeinert:
  $M \activeTransition{\mu} M'$, $M \activeTransition{w}$, $M \activeTransition{w} M'$
  für $\mu \in \Markings(T) \setminus \{ \emptyset \}$ bzw. $w \in (\Markings(T) \setminus \{ \emptyset \})^{*}$.
\end{bem}

\begin{defn}
  \begin{minipage}[t]{0.8 \linewidth}
    $\StepSequences(N) \coloneqq \Set{w \in (\Markings(T) \setminus \{ \emptyset \})^{*}}{M_N \activeTransition{w}}$ \\
    heißen \emph{Schrittfolgen} (\textit{step sequences}).
  \end{minipage}
\end{defn}

\begin{defn}
  Zwei Transitionen $t, t' \in T$ sind
  \begin{itemize}
    \item \emph{nebenläufig} \textit{unter~$M$}, falls $M \activeTransition{t + t'}$,
    \item \emph{in Konflikt} \textit{unter~$M$}, falls $\neg M \activeTransition{t + t'}$.
  \end{itemize}
\end{defn}

\begin{nota}
  Für $\mu \in \Multisets(T)$ ist $\ell(\mu)$ die Multimenge mit
  \[
    \ell(\mu) : \Sigma \to \N, x \mapsto {\sum}_{t \in T, \ell(t) = x} \mu(t)
  \]
  (falls die rechte Zahl endlich ist für alle $x \in \Sigma$). \\
  Für $w = \mu_1 \cdots \mu_n \in \Multisets(T)^{*}$ ist 
  $\ell(w) \coloneqq \ell(\mu_1) \cdots \ell(\mu_n)$.
\end{nota}

\begin{defn}
  Mit $\mu \in \Markings(T) \setminus \{ 0 \}$, $w \in (\Markings(T) \setminus \{ 0 \})^{*}$ und $M$, $M'$ ist defin.:
  \[
    \inferrule
      {M \activeTransition{\mu} M'}
      {M \activeTransition{\ell(\mu)} M'} \quad
    \inferrule
      {M \activeTransition{\mu}}
      {M \activeTransition{\ell(\mu)}} \quad
    \inferrule
      {M \activeTransition{w} M'}
      {M \activeTransition{\ell(w)} M'} \quad
    \inferrule
      {M \activeTransition{w}}
      {M \activeTransition{\ell(w)}}
  \]
\end{defn}

% 2.16
\begin{lem}
  $M \activeTransition{t_1}, \ldots, M \activeTransition{t_n} \wedge \fa{i \neq j} \preset{t_i} \cap \preset{t_j} = \emptyset \implies M \activeTransition{t_1 + \ldots t_n}$
\end{lem}

% 2.17
\begin{lem}
  $M \activeTransition{\mu} M' \wedge \Parikh(w) = \mu \implies M \activeTransition{w} M'$
\end{lem}

\begin{bem}
  Über Schrittfolgen werden somit dieselben Markierungen erreicht wie über Schaltfolgen.
\end{bem}

\begin{defn}
  Der \emph{schrittweise Erreichbarkeitsgraph}~$\StepReachabilityGraph(N)$ besitzt die Knoten $\activeTransition{M}$ und die Kanten $\Set{(M, M')}{\ex{\mu \in \Markings(T) \!\setminus\! \{ \emptyset \}} M \activeTransition{\mu} M'}$.
\end{defn}

% 2.18
\begin{lem}
  Sei~$N$ schlingenfrei. Dann gilt:
  \[
    (\fa{w \in T^{*}, \Parikh(w) = \mu} M \activeTransition{w}) \iff M \activeTransition{\mu}
  \]
\end{lem}

% Erreichbarkeit

\begin{problem}[\emph{Erreichbarkeit}]
  Gegeben seien ein Netz~$N$ und eine Markierung~$M$.
  Frage: Ist~$M$ erreichbar in~$N$?
\end{problem}

\begin{problem}[\emph{$0$-Erreichbarkeit}]
  Gegeben seien ein Netz~$N$.
  Frage: Ist die Nullmarkierung erreichbar?
\end{problem}

\begin{bem}
  Diese Probleme sind lösbar, falls der Erreichbarkeitsgraph endlich ist.
\end{bem}

% 2.19
\begin{defn}
  Eine Stelle $s \in S$ heißt \emph{$n$-beschränkt} / \emph{beschränkt}, falls
  \[
    \sup \Set{M(s)}{M \in \activeTransition{M_N}} \leq n
    \quad / \quad
    \sup \Set{M(s)}{M \in \activeTransition{M_N}} < \infty.
  \]
  Ein Netz heißt ($n$-) \textit{beschränkt}, wenn alle Stellen $s \in S$ ($n$-) beschränkt sind.
  Ein Netz heißt \emph{sicher}, wenn es 1-beschränkt ist.
\end{defn}

% 2.20
\begin{prop}
  $\activeTransition{M_N}$ endlich $\iff$ $N$ beschränkt
\end{prop}

% Verklemmung

% 2.21
\begin{defn}
  Eine Transition $t \in T$ heißt \emph{tot} \textit{unter~$M$}, falls $\fa{M' \in \activeTransition{M}} \neg M' \activeTransition{t}$.
  \begin{itemize}
    \item $M$ heißt \textit{tot}, falls alle Transitionen unter~$M$ tot sind.
    \item $N$ heißt \textit{tot}, falls $M_N$ tot ist.
    \item $N$ heißt \emph{verklemmungsfrei}, falls $\fa{M \in \activeTransition{M_N}} \neg (M \text{ tot})$
    \item $t$ heißt \emph{lebendig} \textit{unter $M$}, falls $\fa{M' \in \activeTransition{M}} \neg (t \text{ ist tot unter~$M$})$
    \item $t$ heißt \textit{lebendig}, falls $t$ lebendig unter~$M_N$ ist.
    \item $M$ heißt \textit{lebendig}, wenn alle $t \in T$ unter~$M$ lebendig sind.
    \item $N$ heißt \textit{lebendig}, wenn $M_N$ lebendig ist.
  \end{itemize}
\end{defn}

\begin{problem}[\emph{Lebendigkeit}]
  Gegeben $N$. Frage: Ist $N$ lebendig?
\end{problem}

\begin{problem}[\emph{Einzellebendigkeit}]
  Gegeben seien $N$ und $t \in T$.
  Frage: Ist $t$ lebendig?
\end{problem}

\end{document}
