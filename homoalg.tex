\documentclass{cheat-sheet}

\pdfinfo{
  /Title (Zusammenfassung Homologische Algebra)
  /Author (Tim Baumann)
}

\usepackage{pgffor} % \foreach-Schleifen

\newcommand{\nspace}[1]{\foreach \i in {1,...,#1}{ \! }} % Negativer Abstand
\newcommand{\SetC}{\mathbf{Set}} % Kategorie der Mengen
\newcommand{\Top}{\mathbf{Top}} % Kategorie der topologischen Räume
\newcommand{\AbGrp}{\mathbf{AbGrp}} % Kategorie der abelschen Gruppen
\newcommand{\op}{\mathrm{op}} % opposite category
\DeclareMathOperator{\sk}{sk} % Skelett
\newcommand{\CC}[1]{{#1}_{\bullet}} % Kettenkomplex (chain complex)
\newcommand{\CCC}[1]{{#1}^{\bullet}} % Ko-Kettenkomplex (cochain complex)
\DeclareMathOperator{\Hom}{Hom} % Homomorphisms


% Kleinere Klammern
\delimiterfactor=701


\begin{document}

\maketitle{Zusammenfassung Homologische Algebra}

% Basierend auf dem Buch "Methods of Homological Algebra" von Gelfand und Manin

% §I.2 Triangulierte Räume

\begin{defn}
  \emph{Verklebedaten} sind gegeben durch einen Funktor
  \[ X : \Delta_{\text{strikt}}^\op \to \SetC. \]
  Dabei ist $\Delta_{\text{strikt}}$ die Kategorie mit den Mengen
  $[n] \coloneqq \{ 0, 1, ..., n \}$ für $n \in \N$ als Objekten und streng monotonen Abbildungen.
\end{defn}

\begin{nota}
  $X_{(n)} \coloneqq X([n])$ heißt Menge der $n$-Simplizes.
\end{nota}

\begin{defn}
  Das \emph{Standard-$n$-Simplex} $\Delta_n \subset \R^{n+1}$ ist die von den $(n{+}1)$ Standardbasisvektoren aufgespannte lineare Hülle. Eine streng monotone Abb $f : [n] \to [m]$ induziert durch Abbilden des $i$-ten Basisvektors auf den $f(i)$-ten eine Inklusion $\Delta_f : \Delta_n \to \Delta_m$, 
\end{defn}

\begin{defn}
  Die \emph{geometrische Realisierung} von Verklebedaten $X$ ist der topologische Raum
  \[ \abs{X} \coloneqq \left( \coprod_{n \in \N} (\Delta_n \times X_{(n)}) \right) / R \]
  Dabei ist $X_{(n)}$ diskret. Die Äquivalenzrelation $R$ wird erzeugt von
  \[
    (\Delta_f(t), x) \sim (t, X(f)(x)) \enspace
    \text{mit $t \in \Delta_m$, $x \in X_{(n)}$, $f : [m] {\to} [n]$ s.m.s.}
  \]
\end{defn}

% Ausgelassen: Proposition I.1.3

\begin{defn}
  Das \emph{$k$-Skelett} $\sk_k X$ von Verklebedaten $X$ ist definiert durch
  \[
    (\sk_k X)_{(n)} \coloneqq
    \Set{ x \in X_{(n)} }{ n \leq k }, \enspace
    %\begin{cases}
    %  X_{(n)}, & \text{falls $n \leq k$,} \\
    %  \emptyset, & \text{sonst.}
    %\end{cases}, \quad
    (\sk_k X)(f) \coloneqq X(f) \enspace \text{sofern möglich}
  \]
\end{defn}

% Ausgelassen: I.1.4 (Triangulierung des Produktes zweier Simplizes)

% §I.2 Simpliziale Mengen

\begin{defn}
  Eine \emph{simpliziale Menge} ist ein Funktor
  \[ X : \Delta^\op \to \SetC. \]
  Dabei ist $\Delta$ die Kategorie mit den Mengen
  $[n] \coloneqq \{ 0, 1, ..., n \}$ für $n \in \N$ als Objekten und monotonen Abbildungen.
\end{defn}

\begin{nota}
  $X_n \coloneqq X([n])$ heißt Menge der $n$-Simplizes.
\end{nota}

\begin{defn}
  Die \emph{geometrische Realisierung} einer simplizialen Menge $X$ ist der topologische Raum
  \[ \abs{X} \coloneqq \left( \coprod_{n \in \N} (\Delta_n \times X_{n}) \right) / R \]
  %Dabei ist $\Delta_n \subset \R^{n+1}$ das Standard-$n$-Simplex und $X_{n}$ trägt die diskrete Topologie.
  Die Äquivalenzrelation $R$ wird dabei erzeugt von
  \[
    (\Delta_f(t), x) \sim (t, X(f)(x)) \enspace
    \text{mit $t \in \Delta_m$, $x \in X_n$ und $f : [m] {\to} [n]$ monot.}
  \]
\end{defn}

\begin{defn}
  Ein topologischer Raum heißt \emph{trianguliert}, wenn er die Realisierung von Verklebedaten ist.
\end{defn}

\begin{defn}
  Der \emph{Nerv} einer Überdeckung $X = \cup_{\alpha \in A} U_\alpha$ eines topologischen Raumes ist die simpliziale Menge
  \begin{align*}
    X_n & \coloneqq \Set{(\alpha_0, ..., \alpha_n) \in A^{n+1}}{ U_{\alpha_0} \cap ... \cap U_{\alpha_n} \not= \emptyset } \\
    X(f)(\alpha_0, ..., \alpha_n) & \coloneqq (\alpha_{f(0)}, ..., \alpha_{f(m)}) \quad \text{für } f : [m] \to [n].
  \end{align*}
\end{defn}

\begin{bem}
  Falls die Überdeckung lokal endlich ist und alle nichtleeren, endlichen Schnitte $U_{\alpha_1} \cap ... \cap U_{\alpha_n}$ zusammenziehbar sind, so ist die geom. Realisierung des Nerves der Überdeckung homotopieäq. zu $X$.
\end{bem}

% Ausgelassen: I.2.4 (Singuläre Simplizes)

% I.2.5
\begin{defn}
  $\Delta[p]_n \coloneqq \{ \, g : [n] \to [p] \text{ monoton steigend} \, \}$, $\Delta[p](f)(g) \coloneqq g \circ f$
\end{defn}

% I.2.8
\begin{defn}
  Der \emph{klassifizierende Raum} einer Gruppe $G$ ist gegeben durch die Realisierung der simpl. Menge $BG$ mit $(BG)_n \coloneqq G^n$ und
  \begin{align*}
    BG(f : [m] \to [n])(g_1, ..., g_n) \coloneqq
    (h_1, ..., h_m), \quad h_i = \nspace{3} \prod_{j=f(i-1)+1}^{f(i)} \nspace{3} g_j
    %(g_{f(0)+1} \cdot ... \cdot g_{f(1)}, g_{f(1)+1} \cdot ... \cdot g_{f(2)}, ..., g_{f(m-1)+1} \cdot ... \cdot g_{f(m)})
  \end{align*}
\end{defn}

% I.2.9
\begin{defn}
  Ein $n$-Simplex $x \in X_n$ heißt \emph{degeneriert}, falls eine monotone surjektive Abbildung $f : [n] \to [m]$, $n > m$ und ein Element $y \in X_m$ existiert mit $x = X(f)(y)$.
\end{defn}

% I.2.6
\begin{defn}
  Seien $X$ Verklebedaten. Wir konstruieren eine dazugehörende simpliziale Menge $\tilde{X}$ wie folgt:
  \[ \tilde{X}_n \coloneqq \Set{ (x, g) }{ x \in X_{(k)}, g : [n] \to [k] \text{ monoton und surjektiv} }, \]
  Für eine monotone Abbildung $f : [m] \to [n]$ und $(x, g) \in \tilde{X}_n$ schreiben wir zunächst $g \circ f = f_1 \circ f_2$ mit einer Injektion $f_1$ und einer Surjektion $f_2$ und setzen
  $\tilde{X}(f)(x, g) \coloneqq (X(f)(x), f_2)$.
\end{defn}

% I.2.7
\begin{prop}
  Eine simpliziale Menge $\tilde{X}$ kann genau dann aus (eindeutigen) Verklebedaten gewonnen werden, falls für alle nicht-degenerierten Simplizes $x \in \tilde{X}_n$ und streng monotonen Abbildungen $f : [m] \to [n]$ auch $\tilde{X}(f)(x) \in \tilde{X}_m$ nicht degeneriert ist.
\end{prop}

% Ausgelassen: I.2.10-12

% I.2.13
\begin{prop}
  Seien $X$ Verklebedaten, $\tilde{X}$ die entsprechende simpliziale Menge. Dann gilt $\abs{X} \approx \abs{\tilde{X}}$.
\end{prop}

% I.2.14
\begin{defn}
  Das \emph{$k$-Skelett} $\sk_k X$ einer simplizialen Menge $X$ ist geg. durch
  \[ (\sk_k X)_n \coloneqq \Set{ X(f)(x) }{ p \leq k, f : [n] \to [p] \text{ monoton}, x \in X_p }. \]
\end{defn}

\begin{defn}
  Eine simpliziale Menge $X$ hat \emph{Dimension} $n$, falls $X = \sk_n X$.
\end{defn}

% I.2.15
\begin{defn}
  Eine \emph{simpliziale Abbildung} zwischen simplizialen Mengen $X$ und $Y$ ist eine natürliche Transformation zwischen den beiden Funktoren $\Delta^\op \to \SetC$.
\end{defn}

\begin{defn}
  Die Kategorie der simplizialen Mengen ist die Funktorkategorie $[\Delta^\op, \SetC]$.
\end{defn}

\begin{prop}
  Geom. Realisierung ist ein Funktor $\abs{\blank} : [\Delta^\op, \SetC] \to \Top$.
\end{prop}

\begin{bspe}
  \begin{itemize}
    \item Eine Überdeckung $(U_\alpha)_{\alpha \in A}$ eines topologischen Raumes ist Verfeinerung von $(V_\beta)_{\beta \in B}$, wenn es eine Abbildung $\psi : A \to B$ gibt, sodass $U_\alpha \subset V_{\psi(\alpha)}$ für alle $\alpha \in A$. Dies induziert eine simpliziale Abb. zwischen den Nerven der Überdeckungen durch
    \[ F_n(\alpha_0, ..., \alpha_n) \coloneqq (\psi(\alpha_0), ..., \psi(\alpha_n)). \]
    % Ausgelassen: I.2.17
    \item Ein Gruppenhomomorphismus $\phi : G \to H$ stiftet eine Abbildung $BG \to BH$ zwischen den klassifizierenden Räumen durch
    \[ F(g_1, ..., g_n) \coloneqq (\phi(g_1), ..., \phi(g_n)). \]
  \end{itemize}
\end{bspe}

% §1.3 Simpliziale topologische Räume und das Eilenberg-Zilber-Theorem

% I.3.2
\begin{defn}
  Ein \emph{simplizialer topologischer Raum} ist ein Funktor
  \[ X : \Delta^\op \to \Top. \]
  Die geometrische Realisierung eines simplizialen topologischen Raumes definiert wie bei simplizialen Mengen mit dem Unterschied, dass $X_n$ im Allgemeinen nicht die diskrete Topologie trägt.
\end{defn}

% I.3.3
\begin{defn}
  Eine \emph{bisimpliziale Menge} ist ein Funktor
  \[ X : \Delta^\op \times \Delta^\op \to \SetC. \]
\end{defn}

\begin{nota}
  $X_{nm} \coloneqq X([n],[m])$
\end{nota}

% I.3.4
\begin{bsp}
  Das \emph{direkte Produkt} von simplizialen Mengen $X$ und $Y$ ist die bisimpliziale Menge
  \[
    (X \times Y)_n \coloneqq X_n \times Y_n, \quad
    (X \times Y)(f, g)(x, y) \coloneqq (f(x), g(y)).
  \]
\end{bsp}

% I.3.5
\begin{defn}
  Die \emph{Diagonale} $DX$ einer bisimplizialen Menge $X$ ist die simpliziale Menge mit
  $(DX)_n \coloneqq X_{nn}$ und $DX(f) \coloneqq X(f, f)$.
\end{defn}

% I.3.6
\begin{defn}
  Sei $X$ eine bisimpliziale Menge.
  \begin{itemize}
    \item Setze $\abs{X}^D \coloneqq \abs{DX}$.
    \item Definiere einen simplizialen topologischen Raum $X^I$ durch
    \[ X^I_n \coloneqq \abs{X_{\bullet n}}, \quad X^I(g) \coloneqq \abs{X(\id, g)}. \]
    Setze $\abs{X}^{I,II} \coloneqq \abs{II,I}$.
    \item Definiere analog $\abs{X}^{II,I}$.
  \end{itemize}
\end{defn}

\begin{satz}[\emph{Eilenberg-Zilber}]
  $\abs{X}^D \cong \abs{X}^{I,II} \cong \abs{X}^{II,I}$ kanonisch.
\end{satz}

% I.4 Homologie und Kohomologie

% I.4.3
\begin{defn}
  \begin{itemize}
    \item Ein \emph{Kettenkomplex} $\CC{C}$ ist eine Folge $(C_n)_{n \in \N}$ von abelschen Gruppen und Gruppenhomomorphismen $\partial_n : C_n \to C_{n-1}$ mit der Eigenschaft $\partial_{n-1} \circ \partial_n = 0$.
    \item Ein \emph{Kokettenkomplex} $\CCC{C}$ ist eine Folge $(C^n)_{n \in \N}$ von abelschen Gruppen und Gruppenhomomorphismen $\delta^n : C^n \to C^{n+1}$ mit der Eigenschaft $\delta^{n+1} \circ \delta^n = 0$.
  \end{itemize}
\end{defn}

% I.4.4
\begin{defn}
  Sei $\CC{C}$ ein Kettenkomplex.
  \begin{itemize}
    \item $C_n$ heißt Gruppe der \emph{$n$-Ketten},
    \item $\partial : C_n \to C_{n-1}$ heißt \emph{Randabbildung},
    \item $Z_n(\CC{C}) \coloneqq \ker \partial_n \subset C_n(\CC{C})$ heißt Gruppe der \emph{$n$-Zykel},
    \item $B_n(\CC{C}) \coloneqq \im \partial_{n+1} \subset Z_n(\CC{C})$ heißt Gruppe der \emph{$n$-Ränder},
    \item $H_n(\CC{C}) \coloneqq Z_n(\CC{C}) / B_n(\CC{C})$ heißt \emph{$n$-te Homologiegruppe}.
  \end{itemize}
  Analog nennt man für einen Kokettenkomplex $\CCC{C}$
  \begin{itemize}
    \miniitem{0.46 \linewidth}{$\delta^n$ \emph{Korandabbildung},}
    \miniitem{0.41 \linewidth}{$C^n$ \emph{$n$-Koketten},}
    \miniitem{0.46 \linewidth}{$Z^n \coloneqq \ker \delta^n$ \emph{$n$-Kozykel},}
    \miniitem{0.45 \linewidth}{$B^n \coloneqq \im \delta^{n-1}$ \emph{$n$-Koränder},}
    \item $H^n(\CCC{C}) \coloneqq Z^n(\CCC{C}) / B^n(\CCC{C})$ $n$-te \emph{Kohomologiegruppe}.
  \end{itemize}
\end{defn}

% I.4.1
\begin{defn}
  Sei $X$ eine simpl. Menge. Sei $C_n(X)$ die von den $n$-Simplizes $X_n$ erzeugte abelsche Gruppe (\dh{} die Gruppe der endl. formalen Linearkombinationen mit Koeffizienten in $\Z$). Sei $\delta_n^i : [n-1] \to [n]$ diejenige streng monotone Abb. mit $i \not\in \im \delta_n^i$. Definiere
  \[
    \partial_n : C_n(X) \to C_{n-1}(X), \quad
    \sum_{\sigma \in X_n} \lambda_\sigma \cdot \sigma \enspace \mapsto \sum_{\sigma \in X_n} \lambda_\sigma \sum_{i=0}^n (-1)^i X(\partial_n^i)(\sigma).
  \]
\end{defn}

% I.4.2
\begin{prop}
  $(C_\bullet(X),\partial_\bullet)$ ist ein Kettenkomplex (\dh{} $\partial_{n-1} \circ \partial_n = 0$)
\end{prop}

\begin{defn}
  Sei $X$ eine simpl. Menge und $A$ eine ab. Gruppe. Dann ist ...
  \begin{itemize}
    \item ... der \emph{Kettenkomplex} $(C_\bullet(X; A), \partial_\bullet)$ von $X$ \emph{mit Koeffizienten} in $A$ definiert durch
    \[
      C_n(X; A) \coloneqq C_n(X) \otimes_{\Z} A, \quad
      \partial_n \coloneqq \partial_n \otimes \id : C_n(X; A) \to C_{n-1}(X; A).
    \]
    \item ... der \emph{Kokettenkomplex} $(C^\bullet(X; A), \delta^\bullet)$ von $X$ mit \emph{Koeffizienten} in $A$ definiert durch
    \begin{align*}
      & C^n(X; A) \coloneqq \Hom(C^n(X), A), \\
      \delta^n : \,\, & C^n(X; A) \to C^{n+1}(X; A), \enspace f \mapsto f \circ \delta_{n+1}.
    \end{align*}
  \end{itemize}
\end{defn}

\begin{beob}
  $C_n(X; \Z) = C_n(X)$
\end{beob}

\begin{nota}
  Sei $X$ eine simpliziale Menge. Setze
  \begin{itemize}
    \miniitem{0.48 \linewidth}{$H_n(X) \coloneqq H_n(C_\bullet(X))$,}
    \miniitem{0.48 \linewidth}{$H^n(X) \coloneqq H^n(C^\bullet(X; \Z))$,}
    \miniitem{0.48 \linewidth}{$H_n(X; A) \coloneqq H_n(C_\bullet(X; A))$,}
    \miniitem{0.48 \linewidth}{$H^n(X; A) \coloneqq H^n(C^\bullet(X; A))$.}
  \end{itemize}
\end{nota}

% I.4.5 (Geometrie von Ketten)
\begin{prop}
  Für jede simpl. Menge $X$ ex. ein kanonischer Isomorphismus
  \[ H_0(X, \Z) \cong \text{freie ab. Gr. erzeugt von Zshgskomponenten von $\abs{X}$}. \]
\end{prop}

\begin{defn}
  Der \emph{Kegel} $CX$ über Verklebedaten $X$ ist definiert durch
  \begin{align*}
    (CX)_{(0)} & \coloneqq X_{(0)} \amalg \{ \star \}, \quad (CX)_{(n)} \coloneqq X_{(n)} \amalg (X_{(n-1)} \times \{ \star \}) \\
    (CX)(f)(x) & \coloneqq X(f)(x) \\
    (CX)(f)(x,*) &  \coloneqq \begin{cases}
      X(i \mapsto f(i) - 1)(x), & \text{wenn $f(0) > 0$,} \\
      (X(i \mapsto f(i{+}1) - 1)(x), *), & \text{wenn $f(0) = 0$.}
    \end{cases}
  \end{align*}
\end{defn}

\begin{defn}
  Für Verklebedaten sind die (Ko-)Kettenkomplex (mit Koeffizienten) genauso definiert wie für simpliziale Mengen.
\end{defn}

\begin{prop}
  $H_0(CX) = \Z$, $H_{>0}(CX) = 0$
\end{prop}

% Ausgelassen: I.4.6 Geometrie von Koketten :-)

% I.4.7 Koeffizientensysteme

% I.4.8
\begin{defn}
  Sei $X$ eine simpliziale Menge.
  \begin{itemize}
    \item Ein \emph{homol. Koeffizientensystem} $\mathcal{A}$ auf $X$ ist ein Funktor
    \[ \mathcal{A} : (1 \downarrow X) \to \AbGrp. \]
    Dabei ist $1 : \Delta \to \SetC$ der Funktor, der konstant $\{ \star \}$ ist.\\
    Expliziter besteht ein Koeffizientensystem aus einer abelschen Gruppe $\mathcal{A}_\sigma$ für jedes $n$-Simplex $\sigma \in X_n$ und Abbildungen $\mathcal{A}(f, \sigma) : \mathcal{A}_\sigma \to \mathcal{A}_{X(f)(\sigma)}$ für alle $\sigma \in X_n$, $f \in \Hom_{\Delta}([m], [n])$ mit
    \[
      \mathcal{A}(\id, \sigma) = \id, \quad
      \mathcal{A}(f \circ g, \sigma) = \mathcal{A}(g, X(f)(\sigma)) \circ \mathcal{A}(f, \sigma).
    \]
    \item Ein \emph{kohomol. Koeffizientensystem} $\mathcal{B}$ auf $X$ ist ein Funktor
    \[ \mathcal{B} : (1 \downarrow X)^\op \to \AbGrp. \]
  \end{itemize}
\end{defn}

% I.4.9 (Bemerkungen und Beispiele)

% Ausgelassen: a) Konstante Koeffizientensysteme, c) Koeffizientensystem auf BG
\begin{bsp}
  Sei $Y$ ein topol. Raum, $(U_\alpha)_{\alpha \in A}$ eine offene Überdeckung und $X$ deren Nerv. Dann definiert
  \begin{align*}
    \mathcal{F}_{\alpha_0, ..., \alpha_n} & \coloneqq \{ U_{\alpha_0} \cap ... \cap U_{\alpha_n} \to \R \text{ stetig} \}, \\
    \mathcal{F}(f, (\alpha_0, ..., \alpha_n))(\phi) & \coloneqq \text{passende Einschränkung von $\phi$}.
  \end{align*}
  ein kohomologisches Koeffizientensystem auf $X$.
\end{bsp}

% I.4.10 (Homologie und Kohomologie mit einem Koeffizientensystem)

\begin{defn}
  Sei $\mathcal{A}$ ein homologisches Koeffizientensystem auf einer simplizialen Menge $X$. Wir setzen
  \[ C_n(X; \mathcal{A}) \coloneqq \{ \text{ formale endl. Linearkomb. } \sum_{\sigma \in X_n} \lambda_\sigma \cdot \sigma \text{ mit } \lambda_\sigma \in \mathcal{A}_\sigma \, \} \]
  und definieren $\partial_n : C_n(X; \mathcal{A}) \to C_{n-1}(X; \mathcal{A})$ durch
  \[ \sum_{\sigma \in X_n} \lambda_\sigma \cdot \sigma \enspace \mapsto \sum_{\sigma \in X_n} \sum_{i=0}^n \enspace (-1)^i \mathcal{A}(\partial_n^i, \sigma)(\lambda_\sigma) \cdot X(\partial_n^i)(\sigma). \]
  Die Homologiegruppen des so def. Kettenkomplexes $C_\bullet(X; \mathcal{A})$ heißen \emph{Homologiegruppen} von $X$ \emph{mit Koeffizienten in $\mathcal{A}$}.
\end{defn}

\begin{defn}
  Sei $\mathcal{B}$ ein kohomologisches Koeffizientensystem auf einer simplizialen Menge $X$. Wir setzen
  \[ C^n(X; \mathcal{B}) \coloneqq \{ \text{ Funktionen } f : (\sigma \in X_n) \to \mathcal{B}_\sigma \, \} \]
  und definieren $\delta_n : C^n(X; \mathcal{B}) \to C_{n+1}(X; \mathcal{B})$ durch
  \[ \delta^n(f)(\sigma) \coloneqq \sum_{i=0}^{n+1} (-1)^i \mathcal{B}(\partial_{n+1}^i, \sigma)(f(X(\partial_{n+1}^i)(\sigma))). \]
  Die Kohomologiegruppen des so def. Kokettenkomplexes $C^\bullet(X; \mathcal{B})$ heißen \emph{Kohomologiegruppen} von $X$ \emph{mit Koeffizienten in $\mathcal{B}$}.
\end{defn}

% I.4.11

\begin{bsp}
  Sei $Y$ ein topol. Raum, $U = (U_\alpha)_{\alpha \in A}$, $X$ und $\mathcal{F}$ wie im letzten Beispiel. Die Homologiegruppen $H^n(X, \mathcal{F})$ werden Čech-Kohomologiegruppen der Garbe der stetigen Funktionen auf $Y$ bzgl. der Überdeckung $U$ genannt.
\end{bsp}
% Ausgelassen: Beispiel c)

% I.5 (Garben)

\end{document}