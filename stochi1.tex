\documentclass{cheat-sheet}

\pdfinfo{
  /Title (Zusammenfassung Stochastik 1)
  /Author (Tim Baumann)
}

\newcommand{\Alg}{\mathfrak{A}}
\newcommand{\Ring}{\mathfrak{R}}
\newcommand{\LebAlg}{\mathfrak{L}} % Lebesgue-Borel-Mengen
\renewcommand{\P}{\mathbb{P}}
\newcommand{\E}{\mathbb{E}} % elementary functions
\newcommand{\ER}{\overline\R} % extended reals
\newcommand{\Bor}{\mathfrak{B}} % Borel
\newcommand{\Leb}{\mathcal{L}} % Lebesgue

\newcommand{\IntO}[2]{\Int{\Omega}{}{#1}{#2}} % Integral über \Omega
\newcommand{\IntOmu}[1]{\Int{\Omega}{}{#1}{\mu}} % Integral über \Omega bzgl. \mu

\usepackage{relsize}
\let\myBinom\binom
\renewcommand{\binom}[2]{\mathsmaller{\myBinom{#1}{#2}}}

\begin{document}

\maketitle{Zusammenfassung Stochastik \rom{1}}

\section{Der abstrakte Maßbegriff}

% TODO: Ereignisalgebra?

\begin{defn}
  Eine \emph{Ereignisalgebra} oder \emph{Boolesche Algebra} ist eine Menge $\mathfrak{A}$ mit zweistelligen Verknüpfungen $\wedge$ (\glqq und\grqq) und $\vee$ (\glqq oder\grqq), einer einstelligen Verknüpfung $\overline{\,\cdot\,}$ (Komplement) und ausgezeichneten Elementen $U \in \mathfrak{A}$ (unmögliches Ereignis) und $S \in \mathfrak{A}$ (sicheres Ereignis), sodass für $A, B, C \in \mathfrak{A}$ gilt:

  \begin{multicols}{2}
    \scriptsize
    \begin{enumerate}[label=\roman*.,leftmargin=2em]
      \item $A \wedge A = A$
      \item $A \wedge B = B \wedge A$
      \item $A \wedge S = A$
      \item $A \wedge U = U$
      \item $A \wedge \overline{A} = U$
      \item $A \wedge (B \wedge C) = (A \wedge B) \wedge C$
      \item $A \vee A = A$
      \item $A \vee S = S$
      \item $A \vee U = A$
      \item $A \vee \overline{A} = S$
      \item $A \vee (B \vee C) = (A \vee B) \vee C$
      \item $A \wedge (B \vee C) = (A \wedge B) \vee (A \wedge C)$
    \end{enumerate}
  \end{multicols}
\end{defn}

\begin{defn}
  Sei $\mathfrak{A}$ eine Boolesche Algebra. Dann definiert
  \[ A \leq B \colon\iff A \wedge B = B \]
  eine Partialordnung auf $\mathfrak{A}$, gesprochen $A$ impliziert $B$.
\end{defn}

\begin{defn}
  Eine \emph{Algebra} (auch Mengenalgebra) $\mathfrak{A} \subset \mathcal{P}(\Omega)$ ist ein System von Teilmengen einer Menge $\Omega$ mit $\emptyset \in \mathfrak{A}$, das unter folgenden Operationen stabil ist:
  \begin{itemize}
    \item Vereinigung: $A, B \in \mathfrak{A} \implies A \cup B \in \mathfrak{A}$
    \item Durchschnitt: $A, B \in \mathfrak{A} \implies A \cap B \in \mathfrak{A}$
    \item Komplementbildung: $A \in \mathfrak{A} \implies A^c \coloneqq \Omega \backslash A \in \mathfrak{A}$
  \end{itemize}
\end{defn}

\begin{satz}[Isomorphiesatz von Stone]
Zu jeder Booleschen Algebra $\mathfrak{A}$ gibt es eine Menge $\Omega$ derart, dass $\mathfrak{A}$ isomorph zu einer Mengenalgebra $\mathfrak{A}$ in $\mathcal{P}(\Omega)$ ist.
\end{satz}

\begin{defn}
  Eine \emph{$\sigma$-Algebra} ist eine Algebra $\mathfrak{A} \subset \mathcal{P}(\Omega)$, die nicht nur unter endlichen, sondern sogar unter abzählbaren Vereinigungen stabil ist, d.\,h.

  \[ (A_n)_{n \in \N} \text{ Folge in } \mathfrak{A} \implies \bigcup_{n = 0}^{\infty} A_n \in \mathfrak{A}. \]
\end{defn}

\begin{bem}
  Es gilt damit:

  \begin{itemize}
    \item $\Omega = \emptyset^c \in \mathfrak{A}$
    \item Abgeschlossenheit unter abzählbaren Schnitten:
  \[ (A_n)_{n \in \N} \text{ Folge in } \mathfrak{A} \implies \bigcap_{n = 0}^{\infty} A_n = \left( \bigcup_{n = 0}^{\infty} (A_n)^c \right)^c \in \mathfrak{A}. \]
  \end{itemize}
\end{bem}

\begin{defn}
  Sei $(A_n)_{n \in \N}$ eine Folge in einer $\sigma$-Algebra $\mathfrak{A}$. Setze
  \[
    \limsup_{n \to \infty} A_n \coloneqq \bigcap_{n = 1}^{\infty} \bigcup_{m = n}^{\infty} A_n \in \mathfrak{A}, \quad
    \liminf_{n \to \infty} A_n \coloneqq \bigcup_{n = 1}^{\infty} \bigcap_{m = n}^{\infty} A_n \in \mathfrak{A}.
  \]
\end{defn}

\begin{bem}
  In einer $\sigma$-Algebra, in der die Mengen mögliche Ereignisse beschreiben, ist der Limes Superior das Ereignis, das eintritt, wenn unendlich viele Ereignisse der Folge $A_n$ eintreten. Der Limes Infinum tritt genau dann ein, wenn alle bis auf endlich viele Ereignisse der Folge $A_n$ eintreten.
\end{bem}

\begin{defn}
  Ein \emph{Ring} $\mathfrak{A} \subset \mathcal{P}(\Omega)$ ist ein System von Teilmengen einer Menge $\Omega$ mit $\emptyset \in \mathfrak{A}$, das unter folgenden Operation stabil ist:

  \begin{itemize}
    \item Vereinigung: $A, B \in \mathfrak{A} \implies A \cup B \in \mathfrak{A}$
    \item Differenz: $A, B \in \mathfrak{A} \implies B \backslash A = B \cap A^c \in \mathfrak{A}$
  \end{itemize}

  Ein Ring, der nicht nur unter endlicher, sondern sogar unter abzählbarer Vereinigung stabil ist, heißt \emph{$\sigma$-Ring}.
\end{defn}

\begin{bem}
  $\mathfrak{A}$ ($\sigma$-)\,Algebra $\iff$ $\mathfrak{A}$ ($\sigma$-)\,Ring und $\Omega \in \mathfrak{A}$.
\end{bem}

\begin{satz}
  Sei $(\mathfrak{A}_i)_{(i \in I)}$ eine Familie von ($\sigma$-)\,Ringen / ($\sigma$-)\,Algebren über einer Menge $\Omega$. Dann ist auch $\cup_{i \in I} \mathfrak{A}_i$ ein ($\sigma$-)\,Ring / eine ($\sigma$-)\,Algebra über $\Omega$.
\end{satz}

% TODO: Definition Inhalt

\begin{satz}
  Sei $\Ring$ ein Ring und $\mu$ ein Inhalt. Es gelten für $n \in \N$ und $A_1, ..., A_n \in \Ring$ die Ein- und Ausschlussformeln
  \begin{align*}
    \mu(A_1 \cup ... \cup A_n) &= \sum_{k=1}^n (-1)^{k-1} \quad \sum_{\mathclap{1 \leq i_1 < ... < i_k \leq n}} \quad \mu(A_{i_1} \cap ... \cap A_{i_k}), \\
    \mu(A_1 \cap ... \cap A_n) &= \sum_{k=1}^n (-1)^{k-1} \quad \sum_{\mathclap{1 \leq i_1 < ... < i_k \leq n}} \quad \mu(A_{i_1} \cup ... \cup A_{i_k}).
  \end{align*}
\end{satz}





% Anfang strukturierte sortierte Einträge



% Vorlesung 9

\begin{bem}
  Sei $\mu$ ein W-Maß auf $\Leb(\R^1)$. Dann definiert $x \mapsto F_{\mu}(x) \coloneqq \mu(\left] -\infty, x \right])$ eine VF. Für eine VF $F : \R \to [0,1]$ definiert umgekehrt $\mu_F(\left] a, b \right]) \coloneqq F(b) - F(a)$ ein W-Maß auf $\Leb(\R^1)$. Analog funktioniert dies auf dem $\R^d$.
\end{bem}

% Wichtige Verteilungsfunktionen:

\begin{defn}[Wichtige Verteilungsfunktionen]\mbox{}\\
  \begin{itemize}
    \item \emph{Normalverteilung} (Gaußverteilung) mit EW $\mu$ und Varianz $\sigma^2$:
    \[ F_{\mu \sigma^2}(x) = \tfrac{1}{\sqrt{2 \pi \sigma^2}} \Int{-\infty}{x}{\exp\left(\tfrac{-(t-\mu)^2}{2 \sigma^2}\right)}{t} \]
    erfüllt $F_{\mu \sigma^2}'(x) = \exp\left(\tfrac{-(x-\mu)^2}{2 \sigma^2}\right)$, $F_{\mu \sigma^2}(\mu-x) = 1 - F_{\mu \sigma^2}(\mu+x)$
    \item \emph{Exponentialverteilung} mit Parameter $\lambda > 0$:
    \[ F_{\lambda}(x) = \begin{cases} 0, & \text{ für } x \leq 0 \\
                                      1 - \exp(- \lambda x), & \text{ für } x > 0
    \end{cases} \]
    \item \emph{Poisson-Verteilung} mit Parameter $\lambda > 0$:
    \[ F_{\lambda}(x) = \sum_{0 \leq n \leq x} \tfrac{\lambda^n}{n!} \exp(-\lambda) \]
  \end{itemize}
\end{defn}

% Ausgelassen: Mehrdimensionale Normalverteilung, zweidimensionale Exponentialverteilung
% Ausgelassen: Gauß-Levy-Kuzmin-Theorem, und zugehörige Verteilung
% Ausgelassen: Wiener Maß für Modellierung von Brown'scher Bewegung

% Kapitel 2. Elementare Wahrscheinlichkeitsrechnung

% Kapitel 2.1. Definition der relativen und bedingten relativen Häufigkeit

\begin{defn}
  Ein Ereignis $A \in \Alg$ trete bei $n$ Versuchen genau $h_n(A) \in \N$ mal auf. Dann heißt
  \begin{itemize}
    \item $h_n(A)$ \emph{absolute Häufigkeit} von $A$,
    \item $H_n(A) \coloneqq \tfrac{h_n(A)}{n}$ \emph{relative Häufigkeit} von $A$.
  \end{itemize}
\end{defn}

\begin{bem}
  Unmittelbar klar:
  \begin{itemize}
    \begin{minipage}{0.35\linewidth}
      \item $H_n(A) \in [0,1]$
    \end{minipage}
    \begin{minipage}{0.60\linewidth}
      \item $H_n(A) \leq H_n(B)$ für $A \subset B$
    \end{minipage}
    \item $H_n(A \sqcup B) = H_n(A) + H_n(B)$ für $A \cap B = \emptyset$
  \end{itemize}
\end{bem}

\begin{bem}
  Bei wachsendem $n$ stabilisiert sich normalerweise der Wert $H_n(A)$. Dieser Grenzwert ist die Wahrscheinlichkeit von $A$.
\end{bem}


% Vorlesung 10

\begin{defn}
  Seien $A, B \in \Alg$ Ereignisse, $n \in \N$ die Anzahl der Versuche. Dann heißt
  \[ H_n(A \mid B) \coloneqq \tfrac{H_n(A \cap B)}{H_n(B)} = \tfrac{h_n(A \cap B)}{h_n(B)} \]
  die \emph{relative Wahrscheinlichkeit} von $A$ unter der Bedingung $B$.
\end{defn}

\begin{bem}
  Offenbar gilt:
  \begin{itemize}
    \begin{minipage}{0.35\linewidth}
      \item $H_n(A \mid B) \in [0,1]$
    \end{minipage}
    \begin{minipage}{0.60\linewidth}
      \item $H_n(A_1 \mid B) \leq H_n(A_2 \mid B)$ für $A_1 \subset A_2$
    \end{minipage}
    \item $H_n(A_1 \sqcup A_2 \mid B) = H_n(A_1 \mid B) + H_n(A_2 \mid B)$ für $A_1 \cap A_2 = \emptyset$
  \end{itemize}
\end{bem}

% Kapitel 2.2. Geometrische Wahrscheinlichkeiten

\begin{defn}
  Sei $\Omega \in \Leb(\R^d)$ mit $\lambda_d(\Omega) > 0$. Dann heißt das W-Maß
  \[ \P : \Leb(\Omega) \to [0, 1], \quad A \mapsto \tfrac{\lambda_d(A)}{\lambda_d(\Omega)} \]
  auf $(\Omega, \Leb(\Omega))$ \emph{Gleichverteilung}.
\end{defn}

\begin{defn}
  Sei $\Omega$ eine endliche Menge. Dann definiert
  \[ \P : \mathcal{P} \to [0, 1], \quad A \mapsto \tfrac{|A|}{|\Omega|} = \tfrac{\text{\# günstige Fälle}}{\text{\# mögliche Fälle}} \]
  ein W-Maß auf $(\Omega, \mathcal{P}(\Omega))$, genannt \emph{Laplace'sche Wkt}.
\end{defn}

\begin{bem}
  Damit sind Berechnungen von Wkten mit kombinatorischen Überlegungen möglich.
\end{bem}

% Kapitel 2.4. Berechnung von Wahrscheinlichkeiten durch kombinatorische Überlegungen

\begin{lem}[Fundamentalprinzip des Zählens]
  Seien $A_1, ..., A_n$ endliche Mengen. Dann gilt $| A_1 \times ... \times A_n | = |A_1| \cdots |A_n|$.
\end{lem}

\begin{lem}
  Sei $A$ eine endliche Menge, $r \leq n \coloneqq |A| < \infty$. Dann ist die Anzahl der $r$-Tupel mit Elementen aus $A$ gleich

  \begin{center}
    \begin{tabular}{ r | l l }
      & Mit Wdh. & Ohne Wdh. \\ \hline
      Mit Ordnung & $n^r$ & $\tfrac{n!}{(n-r)!}$ \\
      Ohne Ordnung & $\tfrac{(n+r-1)!}{r!}$ & $\binom{n}{r} \coloneqq \tfrac{n!}{r!(n-r)!}$
    \end{tabular}
  \end{center}
\end{lem}


% Vorlesung 11

\begin{lem}
  Sei $A$ eine endliche Menge, $n \coloneqq |A| < \infty$. Dann ist die Anzahl der möglichen Zerlegungen von $A$ in disjunkte Mengen $B_1, ..., B_k$ mit $|B_i| = n_i$ und $n_1 + ... + n_k = n$ gleich
  \[
    \binom{n}{n_1, ..., n_k} \coloneqq \tfrac{n!}{n_1! \cdots n_k!}. \quad \text{\emph{(Multinomialkoeffizient)}}
  \]
\end{lem}

% Ausgelassen: Beispiel Kartenverteilungen beim Skatspiel

% Kapitel 2.5. Berechnung von geometrischen Wahrscheinlichkeiten

% Ausgelassen: Beispiel Bertrand'sches Paradoxon (http://en.wikipedia.org/wiki/Bertrand_paradox_(probability))

% Ausgelassen: Beispiel Buffonsches Nadelproblem (http://de.wikipedia.org/wiki/Buffonsches_Nadelproblem)
% und dessen Verallgemeinerung auf Polygone


% Vorlesung 12

% Kapitel 2.6. Hypergeometrische Verteilung

\begin{modell}
  Eine Urne enthalte $N$ Kugeln, darunter $M \leq N$ schwarze. Dann ist ist die Wkt für das Ereignis $A^n_m$, dass sich unter $n$ gezogenen Kugeln genau $m \leq \min(n, M)$ schwarze Kugeln befinden,
  \[ \P(A^n_m) = \frac{\binom{M}{m} \binom{N - M}{n - m}}{\binom{N}{n}}. \quad \text{\emph{(hypergeometrische Verteilung)}} \]
\end{modell}

% Ausgelassen: Beispiel: $m$-er beim Lotto
% Ausgelassen: Beispiel: Capture-Recapture-Problem, Maximum-Likelihood-Schätzung

\begin{bem}
  Für Maximum-Likelihood-Schätzungen:
  \begin{itemize}
    \item Der Ausdruck $\binom{N-M}{n-m} / \binom{N}{n}$ wird maximal bei $N \coloneqq \lfloor \tfrac{n-M}{m} \rfloor$.
    \item Der Ausdruck $\binom{M}{m} \cdot \binom{N-M}{n-m}$ wird maximal bei $M \coloneqq \lfloor \tfrac{m (N-1)}{n} \rfloor$.
  \end{itemize}
\end{bem}

% Ausgelassen: Beispiel statistische Qualitätskontrolle

\begin{modell}
  Eine Urne enthalte $N$ Kugeln in $k \leq N$ verschiedenen Farben, darunter $N_1$ in der ersten Farbe, ..., $N_k$ in der $k$-ten Farbe, $N_1 + ... + N_k = N$. Dann ist ist die Wkt für das Ereignis $A^n_{n_1,...,n_k}$, dass sich unter $n$ gezogenen Kugeln genau $n_1 \leq N_1$ Kugeln der ersten Farbe, ..., und $n_k \leq N_k$ Kugeln der $k$-ten Farbe befinden, $n_1 + ... + n_k = n$, gleich
  \[ \P(A^n_{n_1, ..., n_k}) = \frac{\binom{N_1}{n_1} \cdots \binom{N_k}{n_k}}{\binom{N}{n}}. \]
  Diese W-Verteilung heißt \emph{polyhypergeometrische Verteilung}.
\end{modell}

% Ausgelassen: Zusatz-5er beim Lotto

% Kapitel 2.7. Bedingte Wahrscheinlichkeiten

\begin{defn}
  Sei $(\Omega, \Alg, \P)$ ein W-Raum und $A, B \in \Alg$. Dann heißt
  \[ \P(A \mid B) \coloneqq \begin{cases} \tfrac{\P(A \cap B)}{\P(B)}, & \text{ falls } \P(B) > 0 \\
  0, & \text{ falls } \P(B) = 0 \end{cases} \]
  Wahrscheinlichkeit von $A$ unter der Bedingung $B$.
\end{defn}

\begin{bem}
  Falls $\P(B) > 0$ gilt, so ist $\P(- \mid B)$ ein W-Maß über $B$ auf der Spur-$\sigma$-Algebra $\Alg|_B$.
\end{bem}

\begin{lem}
  Seien $A_1, ..., A_k \in \Alg$, dann gilt die Pfadregel:
  \[ \P(A_1 \cap ... \cap A_k) = \P(A_1) \cdot \prod_{i=2}^k \P(A_i \mid A_1 \cap ... \cap A_{i-1}). \]
\end{lem}

% Ausgelassen: Beispiel: Lose ziehen

% Kapitel 2.8. Formel der totalen Wahrscheinlichkeit und Bayes'sche Formel


% Vorlesung 13

\begin{satz}
  Sei $(\Omega, \Alg, \P)$ ein W-Raum und $A_1, ... \in \Alg$ ein vollständiges Ereignissystem, d.\,h. paarweise disjunkt mit
  \[ \Omega = \bigsqcup_{i = 1}^\infty A_i. \]
  Dann gilt für jedes $B \in \Alg$ mit $\P(B) > 0$
  \begin{align*}
    \P(B) &= \sum_{i=1}^{\infty} \P(B \mid A_i) \cdot \P(A_i) & \text{(Formel der totalen Wkt)} \\
    \P(A_n \mid B) &= \frac{\P(B \mid A_n) \cdot \P(A_n)}{\sum_{i=1}^\infty \P(B \mid A_i) \cdot \P(A_i)} & \text{(\emph{Bayessche Formel})}
  \end{align*}
\end{satz}

% Ausgelassen: Interpretation als "Mischformel"

\begin{sprech}
  In der Bayesischen Statistik heißt
  \begin{itemize}
    \item $\P(A_i)$ \quad \,\, \emph{A-priori-Wahrscheinlichkeit},
    \item $\P(A_i \mid B)$ \emph{A-posteriori-Wahrscheinlichkeit}.
  \end{itemize}
\end{sprech}

% Ausgelassen: Beispiel der Produktion auf zwei Maschinen mit untersch. Ausschussraten

% Kapitel 2.9. Unabhängigkeit von Ereignissen

\begin{defn}
  Zwei Ereignisse $A, B \in \Alg$ heißen \emph{($\P$-)unabhängig}, falls
  \[ \P(A \cap B) = \P(A) \cdot \P(B). \]
\end{defn}

\begin{bem}
  \begin{itemize}
    \item $A \in \Alg$ mit $\P(A) = 0$ ist unabhängig zu jedem $B \in \Alg$.
    \item Wenn $A, B \in \Alg$ unabhängig, dann sind auch unabhängig:
    \[ (A^c, B), \quad (A, B^c), \quad (A^c, B^c) \]
  \end{itemize}
\end{bem}

\begin{satz}
  $A, B \in \Alg$ unabhängig $\iff$ $\P{B \mid A} = \P(B)$.
\end{satz}

% Ausgelassen: Beispiel: Ziehen von Karo-, Ass- und Karo-Ass-Karten

\begin{defn}
  Sei $(A_i)_{i \in I}$ ($I$ bel.) eine Familie von Ereignissen in $\Alg$.

  \begin{itemize}
    \item \emph{vollständig unabhhängig}, falls
    \[ \P(A_{i_1} \cap A_{i_2} \cap ... \cap A_{i_m}) = \P(A_{i_1}) \cdot \P(A_{i_2}) \cdots \P(A_{i_n}) \]
    für alle $i_1, ..., i_n \in I$ mit $2 \leq n < \infty$ und
    \item \emph{paarweise unabhängig}, falls
    \[ \P(A_i \cap A_j) = \P(A_i) \cdot \P(A_j) \quad \text{ für alle } i, j \in I, i \not= j. \]
  \end{itemize}
\end{defn}

\begin{acht}
  Aus paarweiser Unabhängigkeit folgt nicht vollständige Unabhängigkeit (Gegenbeispiel: Bernsteins Tetraeder).
\end{acht}

\begin{defn}
  Sei $(\Omega, \Alg, \P)$ ein W-Raum und $\Alg_1, \Alg_2 \subset \Alg$ Ereignissysteme. Dann heißen $\Alg_1$ und $\Alg_2$ \emph{unabhängig}, falls
  \[ \P(A_1 \cap A_2) = \P(A_1) \cdot \P(A_2) \quad \text{für alle } A_1 \in \Alg_1, A_2 \in \Alg_2. \]
\end{defn}

\begin{satz}
  Seien $\Alg_1, \Alg_2 \subset \Alg$ unabhängige Ereignissysteme, die Algebren sind. Dann sind auch die $\sigma$-Algebren $\sigma(\Alg_1)$ und $\sigma(\Alg_2)$ unabhängig.
\end{satz}

% Ausgelassen: Anwendung in der Zuverlässigkeitsanalyse (Reihenschaltung von Parallelschaltungen)


% Vorlesung 14

% Kapitel 2.10. Bernoulli-Schema

\begin{satz}
  Sei $(\Omega, \Alg, \P)$ ein W-Raum, $(A_i)_{i \in \N}$ Folge von unabhängigen Ereignissen mit gleicher Erfolgswkt $\P(A_i) = p$ für alle $i \in \N$. Für $k \leq n$, $k, n \in \N$ ist dann die Wahrscheinlichkeit, dass genau $k$ Stück der Ereignisse $A_1, ..., A_n$ eintreten, genau
  \[ B(k, n, p) \coloneqq \binom{n}{k} \, p^k \, (1-p)^{n-k} \]
  Die zugehörige VF $x \mapsto \sum_{\mathclap{0 \leq k \leq x}} B(k,n,p)$ heißt \emph{Binomialverteilung}.
\end{satz}

% Ausgelassen: Beispiel: Rosinen werden in einen Teig gemengt

\begin{lem}
  Voraussetzung wir im vorherigen Satz. Sei $r, k \in \N$, $1 \leq r$, dann ist die Wkt für das Ereignis $A_k^{(r)}$, dass beim Versuch $A_{k+r}$ der $r$-te Erfolg eintritt, gleich
  \[ \P(A_k^{(r)}) = \binom{k+r-1}{r-1} \, p^r \, (1-p)^k. \]
  Im Spezialfall $r = 1$ ist $\P(A_k^{(1)}) = p \, (1-p)^k$.
\end{lem}

\begin{satz}
  Sei $(\Omega, \Alg, \P)$ ein W-Raum, $A_1, ..., A_r \in \Alg$ mit $p_i \coloneqq \P(A_i)$ für $i = 1, ..., k$ und $p_1 + ... + p_r = 1$. Dann ist die Wahrscheinlichkeit, dass bei $n \in \N$ Versuchen $A_1$ genau $n_1$-mal, $A_2$ genau $n_2$-mal, ..., $A_r$ genau $n_r$-mal auftritt ($n_1 + ... + n_r = n$), genau
  \[ B(n_1, ..., n_r, n, p_1, ..., p_r) \coloneqq \binom{n}{n_1, ..., n_r} p_1^{n_1} \cdots p_r^{n_r}. \]
  Diese W-Verteilung heißt \emph{Multinomialverteilung}.
\end{satz}

% Kapitel 2.11. Grenzwertsatz von Poisson -- Gesetz der kleinen Zahlen

\begin{satz}
  Für $0 \leq m \leq n$, $p \in [0, 1]$ gilt
  \[ \frac{\binom{M}{m} \binom{N-M}{n-m}}{\binom{N}{n}} \enspace \xrightarrow[M/N \to p]{M, N \to \infty} \enspace \binom{n}{m} \, p^m \, (1-p)^{n-m}. \]
\end{satz}

% Intuitive Interpretation: Ein Urnenmodell mit vielen Kugeln können wir auch als Bernoulli-Experiment auffassen, ohne uns einen zu großen Fehler einzuhandeln

\begin{satz}[GWS von Poisson]
  Für $k \in \N$, $\lambda \in \R_{>0}$ gilt
  \[ \binom{n}{m} \, p_n^m \, (1-p_n)^{n-m} \enspace \xrightarrow[n p_n \to \lambda]{n \to \infty} \enspace \frac{\lambda^k}{k!} \exp(-\lambda). \]
\end{satz}

% Vorlesung 15

% Vorlesung 16

% Vorlesung 17

% Vorlesung 18

% Vorlesung 19

% Vorlesung 20


































% End sortierte Einträge







\begin{satz}[von Lusin]
  $f : ([a, b], \LebAlg([a, b])) \to (\R^1, \LebAlg(\R^1))$ ist Borel-messbar $\iff$ $\forall \epsilon > 0 : \exists K\epsilon \subset [a, b]$ abgeschlossen mit $\lambda_1(\R^1 \setminus K_\epsilon)$ und $f|_{K_\epsilon}$ stetig.
\end{satz}

\begin{satz}
  Folgerung: Es sind messbar
  \begin{itemize}
    \item monotone Funktionen
    \item Funktionen mit endlicher Variation
    \item Càdlàg-Funktionen, das sind Funktionen $f : [a, b] \to \R$ mit $\lim_{\epsilon \downarrow 0} f(x+\epsilon) = f(x)$ für alle $x \in \left[a, b\right[$.
  \end{itemize}
\end{satz}

\begin{lem}[Borel-Cantelli]
  Sei $(A_n)_{n \in \N}$ eine Folge von Ereignissen über $(\Omega, \Alg, \P)$. Dann gilt für $A = \limsup_{n \to \infty}$
  \[ \sum_{n=1}^\infty \P(A_n) < \infty \implies \P(A) = 0. \]
  Falls die Ereignisse $(A_n)_{n \in \N}$ unabhängig sind, so gilt
  \[ \sum_{n=1}^\infty \P(A_n) = \infty \implies \P(A) = 1. \]
\end{lem}

\begin{defn}
  Sei $(\Alg_n)_{n \in \N}$ Folge von $\sigma$-Algebren über $\Omega$. Dann ist
  \[ \mathcal{T}_\infty = \cap_{n=1}^\infty \mathcal{T}_n \quad \text{mit} \quad \mathcal{T}_n \coloneqq \sigma \left( \cup_{k=n}^\infty \Alg_k \right) \]
  die \emph{terminale $\sigma$-Algebra} von $(\Alg_n)_{n \in \N}$.
\end{defn}

% Was bedeutet Unabhängigkeit von $\sigma$-Algebren?

\begin{satz}[Null-Eins-Gesetz von Kolmogorow]
  Sei $(\Alg_n)_{n \in \N}$ eine Folge von unabhängigen Unter-$\sigma$-Algebren in einem W-Raum $(\Omega, \Alg, \P)$. Dann gilt $\P(A) \in \{ 0, 1 \}$ für alle Ereignisse $A \in \mathcal{T}_\infty$ der terminalen $\sigma$-Algebra.
\end{satz}

\begin{defn}
  Eine $\Alg$-messbare numerische Funktion $X$ über einem Wahrscheinlichkeitsraum $(\Omega, \Alg, \P)$ heißt \emph{Zufallsgröße} (ZG) oder \emph{Zufallsvariable}.
\end{defn}

\begin{defn}
  Die durch die ZG $X$ auf $(\R^1, \LebAlg(\R^1))$ induzierte Bildmaß $P_X$
  \[ P_X(B) = \P(X^{-1}(B)) = \P(\Set{ \omega \in \Omega  }{ X(\omega) \in B }) \]
  heißt \emph{Verteilung} der ZG $X$.
  \[ F_X(x) = P_X(\left]-\infty, x\right]) = \P(\Set{ \omega \in \Omega }{ X(\omega) \leq x }) \]
  heißt die \emph{Verteilungsfunktion} (VF) der ZG $X$.
\end{defn}

\begin{satz}
  $F$ sei eine VF auf $\R^1$. Dann existiert ein Wahrscheinlichkeits-Raum $(\Omega, \Alg, \P$ und eine ZG $X$ derart, dass
  \[ F_X(x) = F(x) \text{ für } x \in \R^1 \]
\end{satz}

\begin{nota}
  Sei $X$ eine Zufallsgröße und $B \in \LebAlg(\overline{\R}^1)$. Dann schreibe
  \[ \{ X \in B \} = X^{-1}(B). \]
\end{nota}

\begin{defn}
  Eine endliche Familie von Zufallsgrößen $X_1, ..., X_n$ heißt \emph{stochastisch unabhängig}, falls
  \[ \P(\bigcap_{i=1}^n \{ X_i \in B_i \}) = \prod_{i=1}^n \P(\{ X_i \in B_i \}) \text{ für alle } B_i \in \mathcal{L}(\overline{R}^1), i = 1, ..., n. \]
\end{defn}

\begin{satz}
  Seien $X_1, ..., X_n$ unabhängige Zufallsgrößen über $(\Omega, \Alg, \P)$ von $g_1, ..., g_n$ Borel-messbare Funktionen von $\R^1$ nach $\R^1$. Dann sind auch die Zufallsgrößen $Y_i \coloneqq g_i \circ X_i$ unabhängig über $(\Omega, \Alg, \P)$.
\end{satz}

% Definition: einfache Funktionen
% Definition: $\mu$-Integral für einfache Funktionen

% Eigenschaften: Linearität, Monotonie, \IntOmu{\ind_A} = \mu(A)

\begin{satz}
  Sei $0 \leq f_1 \leq f_2 \leq ...$ eine isotone Folge elementarer Funktionen über $(\Omega, \Alg)$. Dann gilt für jede elementare Funktion $f$ mit $f \leq \sup_{n \in N} f_n$ die Ungleichung $\IntOmu{f} \leq \sup_{n \in \N} \IntOmu{f_n}$.
\end{satz}

% Folgerung
\begin{satz}
  Seien $(f_n)_{n \in \N}$ und $(g_n)_{n \in \N}$ isotone Folgen elementarer Funktionen mit $\sup_{n \in \N} f_n = \sup_{n \in \N} g_n$. Dann ist $\sup_{n \in \N} \IntOmu{f_n} = \sup_{n \in \N} \IntOmu{g_n}$.
\end{satz}

% Satz: Jede nichtnegative, messbare, numerische Funktion ist Grenzwert einer Folge einfacher Funktionen
% Def: $\mu$-Integral von solchen Funktionen

% Integral von nicht unbedingt nicht-negativen messbaren Funktionen

\begin{satz}
  Sei $f : (\Omega, \Alg, \mu) \to (\overline{R}^1, \LebAlg(\R^1))$ sein $\Alg$-messbar, numerisch. Dann sind äquivalent:
  \begin{itemize}
    \item $f$ ist $\mu$-integrierbar
    \item $f^+$ und $f^-$ sind $\mu$-integrierbar mit $\IntOmu{f^{\pm}} < \infty$
    \item $\IntOmu{|f|} < \infty$
    \item $\IntOmu{g} < \infty$ für eine $\Alg$-messbare, numerische Funktion mit $|f| \leq g$
  \end{itemize}
\end{satz}

\begin{satz}
  Seien $f, g : (\Omega, \Alg, \mu) \to (\R^1, \LebAlg(\R^1))$ $\mu$-integrierbar. Dann sind $f \pm g$, $f \vee g$, $f \wedge g$ und $\alpha \cdot f$ für $\alpha \in \R^1$ $\mu$-integrierbar und es gilt
  \begin{align*}
    \IntOmu{\alpha \cdot f + \beta \cdot g} = \alpha \IntOmu{f} + \beta \IntOmu{g}, & \quad
    \left| \IntOmu{f} \right| \leq \IntOmu{|f|}, \\
    f \leq g \implies \IntOmu{f} &\leq \IntOmu{g}
  \end{align*}
\end{satz}

\begin{defn}
  Mit $L^p(\mu) = L^p(\Omega, \Alg, \mu)$ bezeichnen wir den normierten Vektorraum der aus den Funktionen $f : (\Omega, \Alg, \mu) \to (\R^1, \LebAlg(\R^1))$ mit $\IntOmu{|f|^p} < \infty$ für $1 \leq p \leq \infty$ besteht. Die Norm in diesem Raum wird durch
  \[ \|f\|_p \coloneqq \left( \IntOmu{|f|^p} \right)^{1/p} \]
  definiert. Es kann gezeigt werden, dass die Normeigenschaften erfüllt sind.
\end{defn}

% Definition: Konvergenz im $L^p(\mu)$

\begin{bem}
  Der $L^p(\mu)$ ist ein vollständiger normierter Raum, d.\,h. jede Cauchy-Folge bzgl. der Norm $\| \cdot \|_p$ ist auch konvergent. Im Spezialfall $p = 2$ heißt $L^p(\mu)$ Hilbertraum der quadratisch integrierbaren Funktionen mit Skalarprodukt $\langle f , g \rangle = \IntOmu{f \cdot g}$. Es gilt in diesem Fall außerdem die Cauchy-Schwarz-Bunjakowski-Ungleichung:
  \[ \| f \cdot g \|_{1} \leq \|f\|_2 \cdot \|g\|_2 \]
  Höldersche Ungleichung:
  \[ \| f \cdot g \|_{1} \leq \|f\|_p \cdot \|g\|_q \]
  wobei $p, q \in [1, \infty]$ mit $\tfrac{1}{p} + \tfrac{1}{q} = 1$.
\end{bem}

% Konvergenzsätze für Integrale messbarer Funktionen

% 3.5.1 Satz von der monotonen Konvergenz

\begin{satz}
  Sei $f_n : (\Omega, \Alg, \mu) \to (\R^1, \LebAlg(\R^1))$ $\Alg$-messbar und $0 \leq f_1 \leq f_2 \leq ...$. Dann gilt
  \[ \IntOmu{\sup_{n \in \N} f_n} = \sup_{n \in \N} \IntOmu{f_n} \]
\end{satz}

% Folgerung
\begin{satz}[von Beppo Levi]
  Sei $(f_n)_{n \in \N}$ eine Folge monotoner nichtnegativer, $\Alg$-messbarer, numerischer Funktionen auf $(\Omega, \Alg, \mu)$. Dann gilt:
  \[ \IntOmu{\sum_{n=1}^{\infty} f_n} = \sum_{n=1}^\infty \IntOmu{f_n} \]
\end{satz}

% Folgerung
\begin{satz}
  $f$ sei $\Alg$-messbar, nichtnegativ und $\mu$-integrierbar. Dann ist
  \[ \nu(A) \coloneqq \Int{A}{}{f}{\mu} = \IntOmu{f \cdot \chi_A} \]
  ein endliches Maß auf $(\Omega, \Alg)$.
\end{satz}

\begin{satz}[Lemma von Fatou]
  Sei $f_n : (\Omega, \Alg, \mu) \to (\R^1, \LebAlg(\R^1))$ eine Folge $\Alg$-messbarer, nichtnegativer Funktionen. Dann gilt:
  \[ \IntOmu{\liminf_{n \to \infty} f_n} \leq \liminf_{n \to \infty} \IntOmu{f_n} \]
\end{satz}

% Approximationssatz für Maße?

% TODO: Verteilungen:
% * Bernoulli
% * Binomial
% * geometrische Verteilung
% * Pascal-Verteilung
% * (verallgemeinerte) hypergeometrische Verteilung

% Unabhängigkeit von Mengensystemen

\begin{satz}
  Seien $(\Omega, \Alg, \mu)$ ein Maßraum, $(\Omega', \Alg')$ ein messbarer Raum und $f : \Omega' \to \Omega$ messbar. Bezeichne mit $\mu' \coloneqq \mu \circ f^{-1}$ das Bildmaß von $\mu$ unter $f$. Dann gilt für alle $\mu'$-integrierbaren Funktionen $g : \Omega' \to \R$:
  \[ \Int{\Omega'}{}{g}{\mu'} = \IntOmu{g \circ f} \]
\end{satz}

\begin{satz}[Transformationssatz]
  Sei $U, \widetilde{U} \opn \R^d$ und sei $\phi : U \to \widetilde{U}$ ein $\mathcal{C}^1$-Diffeomorphismus. Dann ist eine Funktion $f : \widetilde{U} \to \ER$ genau dann auf $\widetilde{U}$ Lebesgue-Borel-integrierbar, wenn $(f \circ \phi) \cdot \left|\det(D\phi)\right| : U \to \ER$ auf $U$ Lebesgue-Borel-interierbar ist. In diesem Fall gilt
  \[ \Int{U}{}{(f \circ \phi) \cdot \left|\det(D\phi)\right|}{\mu_{LB}} = \Int{\phi(U)}{}{f}{\mu_{LB}} = \Int{\widetilde{U}}{}{f}{\mu_{LB}}. \]
  Obige Gleichung ist auch erfüllt, wenn lediglich $f \geq 0$ gilt (also $f \in \overline{\E}(\widetilde{U}, \Bor(\widetilde{U}))$; dann kann das Integral auch den Wert $\infty$ annehmen).
\end{satz}

\begin{defn}
  Für eine ZG $X : (\Omega, \Alg, \P) \to (\ER^1, \Leb(\ER^1))$ heißt die Zahl
  \[ \E X := \Int{\Omega}{}{X}{\P} = \Int{\R^1}{}{\id}{P_X} \]
  der \emph{Erwartungswert} der ZG X, wobei $P_X = \P \circ X^{-1}$.
\end{defn}

\begin{kor}
  Sei $g : \R^1 \to \R^1$ Borel-messbar und $P_X$-integrierbar. Dann gilt
  \[ \E g(X) = \Int{\R^1}{}{g}{P_X} = \Int{- \infty}{\infty}{g(x)}{F_X(x)}, \]
  wobei das rechte Integral das uneigentliche Riemann-Stieltjes-Integral bzgl. $F_X$ ist.
\end{kor}

\begin{defn}
  Für Zufallsvektoren $X = (X_1, ..., X_k)$ mit Werten in $\R^k$ ist
  \[ \E X = (\E X_1, ..., \E X_k) \]
\end{defn}

Sei $g : \R^k \to \R$ Borel-messbar und $P_{(X_1, ..., X_k)}$-integrierbar. Dann ist
\[ \E g(X_1, ..., X_k) = \Int{\R^k}{}{g(x_1, ...)}{P_{(X_1, ..., X_k)}} = \Int{-\infty}{\infty}{g(x_1, ..., x_k)}{F_X(x_1, ..., x_k)} \]

% 3.7 Berechnung von Erwartungswerten

% 3.7.1 Grundtypen & deren Mischung

$F = F_X$ sei die VF einer ZG $X : (\Omega, \Alg, \P) \to (\R^1, \Leb(\R^1), P_X)$

\begin{defn}
  \begin{itemize}
    \item $F_X$ heißt diskret, falls $F_X$ höchstens abzählbar viele Sprünge $x_1, x_2, ... \in \R$ mit $p_k \coloneqq F(x_k) - \lim_{x \uparrow x_k} F(x) > 0$ mit $\sum_{k=1}^\infty p_k = 1$ besitzt (dann ist $F_X$) zwischen den Sprüngen konstant)
    item 
    \item $F_X$ heißt \emph{stetig} (diffus, atomlos), wenn $F_X$ in jedem Punkt stetig ist. Dann gilt $P_X(\{ X = x \}) = 0$.
    \item $F_X$ heißt \emph{absolut stetig} (totalstetig), wenn es für alle $\epsilon > 0$ ein $\delta > 0$ gibt, sodass für abzählbare viele, disjunkte Intervalle $I_k = \left]a_k, b_k\right]$ mit $\sum_{k} (b_k - a_k) < \delta$ sich $\sum_{k} (F_X(b_k) - F_X(a_k)) \leq \epsilon$ ergibt.
    % Bsp: Lipschitz-stetige Funktionen
    \item \emph{singulärstetig} (stetig, aber nicht absolutstetig), wenn die Wachstumspunkte VF $F_X$ eine Lebesgue-Nullmenge bilden, also
    \[ \lambda_1(\Set{ x \in \R^2 }{ \forall \epsilon > 0 : F(x+\epsilon)- F(x-\epsilon) > 0 }) = 0 \]
    % Bsp: Cantor-Treppe (Teufelstreppe)
  \end{itemize}
\end{defn}

% TODO: was ist die Annahme bei dem Satz.
\begin{satz}
  $F_X'(x)$ existiert für Lebesgue-fast-alle $x \in \R^1$.
\end{satz}

\begin{satz}
  Jede VF $F$ auf $\R^1$ besitzt eine eindeutige Zerlegung (Lebesgue-Zerlegung) als konvexe Linearkombination einer diskreten, singulär-stetigen und absolut-stetigen VF:
  \[ F = \alpha_d F_d + \alpha_s F_s + \alpha_a F_a \]
  mit $\alpha_d, \alpha_s, \alpha_a \geq 0$ und $\alpha_d + \alpha_s + \alpha_a = 1$.
\end{satz}

\begin{defn}
  Falls $F_X$ absolut-stetig, dann heißt die nicht-negative, Lebesgue-messbare Funktion $ff_X(x) \coloneqq \begin{cases} F'_X(x) & \text{falls Ableitung ex.} \\ 0 & \text{sonst,} \end{cases}$
  welche $\Int{\R^1}{}{f_X}{\lambda_1} = 1$ erfüllt, die \emph{W-Dichte} von $F_X$.
\end{defn}

% 10.1.2014

\[ \E X = \Int{\R^1}{}{x}{F_X} = \Int{\R^1}{}{\id \cdot f_X}{\lambda} = \Int{- \infty}{\infty}{x \cdot f_X(x)}{x} \]

\begin{bem}
  Für diskrete $F_X$, also
  \[ F(x) = c_i \quad \text{für alle} x \in \left] x_i, x_{i+1} \right[ \]
  für $x_1, ... \in \R$ und $c_1, ... \in \R$ gilt
  \[ \E X = \sum_{i=1} (x_{i}) \]
\end{bem}

% \[ \E X = \Int \]

Deutung der $\E X$ als Massenschwerpunkte

...

% Analoge Betrachtungen für $k$-dimensionale Z-Verteilungen $(X_1, ..., X_k)$

$F_(X_1, ..., X_k)$ heißt \emph{absolut stetig}, falls für alle $\epsilon > 0$ ein $\delta > 0$ existiert, sodass für $I_{\alpha} = \left] a_j, b_j \right]$, $j = 1, 2, ...$ mit $\sum_{j \geq 1} \lambda_k(I_j) \leq \delta$ gilt:
\[ \sum_{j \geq 1} \P_{(X_1, ..., X_k)}(I) = \sum_{j \geq 1} (triangle F_{(X_1, ..., X_k)}) I_j \leq \epsilon \]

Genau dann existiert eine (Lebesgue-) Borel-messbare Funktion $f_{(X_1, ..., Xk)}(x_1, ..., x_k) \geq 0$ mit $\Int{\R^1}{}{f_{(X_1, ..., X_k)}}{\lambda_k} = 1$

Sei $g : \R^k \to \R^1$ Borel-messbar

$\E g(X_1, ..., X_k) = \Int{\R^1}{}{g \cdot f_{(X_1, ..., X_k)}}{\lambda_k}$
 
Falls $F_{(X_1, ..., X_k)}$ "hinreichend glatt", so ergibt sich
$f_{X_1, ..., X_k}(x_1, ..., x_k) = \frac{\partial^k}{\partial_{x_1} \cdot \partial_{x_k}} F_{(X_1, ..., X_k)}(x_1, ..., x_k)$

$F_{X_1, ..., X_k}$ heißt singulär-stetig, falls $P_{(X_1, ..., X_k)}(\{ x \}) = 0 \forall x \in \R^k$
und es existiert eine Lebesgue-messbare Menge $S$ mit $\lambda_k(S) = 0$ und $P((X_1, ..., X_k))(S) = 1$.

$F_{(X_1, ..., X_k)}$ heißt diskret, falls eine höchstens abzählbare Punktmenge $S = \{ x_1, ... \} \subset \R^k$ und $p_i = P_{(X_1, ..., X_k)}(\{ x_i \}) > 0$ mit $\sum_{i \geq 1} p_i = 1$

Sei $x_i = (x_i^{(1)}, ..., x_i^{(k)}) \in \R^k$

$\E g(X_1, ..., X_k) = \sum_{i \geq 1} g(x_i^{(1)}, ..., x_i^{(k)}) p_i$

% Riemann-Stieltjes-Integral

$g : \R^1 \to \R^1$ sei zunächst beliebig (stetig oder hinreichend glatt)

Sei $a = \xi_0^{(n)} < \xi_1^{(n)} < ... < x_{k_n}^{(n)}$ und $x_k^{(n)} \in \left] \xi_{k-1}^{(n)}, \xi_{k}^{(n)} \right[$.

\begin{defn}
  $(\xi_n)$ sei eine Zerlegungsfolge mit $\max_{1 \leq k \leq k_n} (x_k^{(n)} - x_{k-1}^{(n)}) \xrightarrow{n \to \infty} 0$

  $(x_k^{(n)})$ sei eine Zwischenwertfolge

  \[ \lim_{n \to \infty} \sum_{k=1}^{k_n} g(x_k^{(n)}) (F(x_k^{(n)}) - F(x_{k-1}^{(n)})) = \Int{a}{b}{g(x)}{F(x)} = \Int{[a, b]}{}{g}{F\lambda_1} \]

  wobei $F : \R^1 \to \R^1$ zunächst beliebig (monoton oder von beschränkter Variation)
\end{defn}

% partielle Integration für R-S-Integrale

Sei $g$ bzgl. $F$ R-S-integrierbar, d.\,h. der Grenzwert oben existiert
Dann ist auch $F$ bzgl. $g$ R-S-integrierbar und es gilt

\[ \Int{a}{b}{g(x)}{F(x)} = [g(x) \cdot F(x)]_a^b - \Int{a}{b}{F(x)}{g(x)} \]

Ausnutzen der partiellen Integration zur Berechnung von Erwartungswerten

\[ \E X = \Int{- \infty}{\infty}{x}{F(x)} \]
\[ \Int{a}{b}{x}{F_X(x)} = \lim_{a \to -\infty, b \to \infty} [x \cdot F_X(-x)]_0^{-a} - \Int{0}{-a}{F_X(-x)}{x} + [x (F_X(x) - 1)]_0^b - \Int{0}{b}{(F_X(x) - 1)}{x} \]

Falls $\lim{x \to \infty} x F_X(-x) = \lim{x \to \infty} x (1 - F_X(x)) = 0$, so gilt

$\E X = \Int{0}{\infty}{1 - F_X(x) - F_X(-x)}{x}$, falls $\E |X| < \infty$

$\E |X| = \Int{0}{\infty}{1 - F_X(x) - F_X(-x)}{x}$

Genauso werden Erwartungswerte von Funktionen von $X$ berechnet, z.B. mit $x^2 F_X(-x) \xrightarrow{x \to \infty} 0$ und $x^2 (1 - F_X(x)) \xrightarrow{x \to \infty} 0$

$\E X^2 = 2 \Int{0}{\infty}{x (1 - F_X(x) + F_X(-x))}{x} = \P(|X| > x)$ (falls $F_X$ stetig)

$\E |X|^k = k \Int{0}{\infty}{x^{k-1} (1 - F_X(x) + F_X(-x))}{x}$

\begin{defn}
  $\E X^k$ ($\E |X|^k$) heißt $k$-tes (absolutes) Moment der ZG $X$. $\E (X - \E X)^k$ heißt $k$-tes zentriertes Moment der ZG X. $\mathrm{Var}(X) \coloneqq D^2 X = \E (X - \E X)^2 = \E X^2$ heißt \emph{Streuung} (Dispersion, Varianz) der ZG $X$.
\end{defn}

\end{document}
