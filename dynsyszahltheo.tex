\documentclass{cheat-sheet}

\pdfinfo{
  /Title (Dynamische Systeme und Zahlentheorie)
  /Author (Tim Baumann)
}

\DeclareMathOperator{\Aut}{Aut} % Automorphismengruppe
\DeclareMathOperator{\End}{End} % Endomorphismenmonoid
\newcommand{\clos}[1]{\overline{#1}} % topologischer Abschluss

\begin{document}

\maketitle{Dyn. Systeme in der Zahlentheorie}

% TODO:
% Van der Waerdens Theorem
% Szemerédis Theorem
% Poincarés Wiederkehrsatz
% Rados Theorem
% "Fast-Periodizität"
% gleichförmig wiederkehrend
% "proximality"
% Schur und Brauers Resultat (SB)
% Hindmans Resultat (NH)
% Grünwalds Resultat (G)
% Hilberts Resultat (DH)
% Theorem (MBR = "multiple Birkhoff Recurrence")
% Stabilität:
%   * Im Sinne von Poisson
%   * Im Sinne von Lagrange
%   * Im Sinne von Lyapunov

% Kapitel 1. Wiederkehr und gleichmäßige Wiederkehr in kompakten Räumen

% 1.1. Dynamische Systeme und wiederkehrende Punkte

\begin{defn}
  Ein \emph{dynamisches System} ist ein kompakter metrischer Raum $X$ mit einer Gruppen-Wirkung $\varphi : G \to \Aut(X), \, g \mapsto T_g$ oder einer Monoid-Wirkung $\rho : M \to \End(X), \, m \mapsto T_m$.
\end{defn}

\begin{bem}
  Falls $G \!=\! \Z$ oder $M \!=\! \N$, dann bezeichnen wir mit $T \coloneqq T_1$ den Erzeuger der Aktion und nennen $(X, T)$ ein \emph{zykl. System}.
\end{bem}

% Def 1.1
\begin{defn}
  Sei $X$ ein topol. Raum, $T : X \to X$ stetig.
  Ein Punkt $x \in X$ heißt \emph{wiederkehrend}, falls für für alle Umgebungen $V \subset X$ von $x$ ein $n \geq 1$ existiert mit $T^n(x) \in V$.
\end{defn}

\begin{bem}
  Sei $X$ sogar ein metrischer Raum, $x \in X$ wiederkehrend. \\
  Dann gibt es eine Folge $(n_k)$ mit $d(T^{n_k}(x), x) \to 0$ für $k \to \infty$.
\end{bem}

% nicht dort im Buch
\begin{defn}
  Sei $X$ ein topol. Raum, $T : X \to X$ stetig. Dann heißt
  \[ Q(x) \coloneqq \clos{\Set{T^n x}{n \geq 1}} \subseteq X \]
  \emph{abgeschlossener Vorwärtsorbit} von $x \in X$.
\end{defn}

% nicht explizit im Buch
\begin{lem}
  \begin{itemize}
    \item $x \in X$ ist wiederkehrend $\iff$ $x \in Q(x)$
    \item $x \in Q(y) \implies T(x) \in Q(y) \iff Q(x) \subseteq Q(y)$
    \item Die Relation $xRy \!\!\coloniff\!\! x \in Q(y)$ ist transitiv.
  \end{itemize}
\end{lem}

\begin{thm}
  Sei $X$ ein kompakter topol. Raum, $T : X \to X$ stetig. \\
  Dann gibt es einen wiederkehrenden Punkt $x \in X$.
\end{thm}
% XXX: Bemerkung: Beweis benutzt Auswahlaxiom?

% Defn 1.2
\begin{defn}
  Sei $K$ eine kompakte Gruppe, $a \in K$ und $T(x) \coloneqq ax$. Dann heißt $(K, T)$ ein \emph{Kronecker-System}.
\end{defn}

% Thm 1.2
\begin{thm}
  In einem Kronecker-System sind alle $x \in K$ wiederkehrend.
\end{thm}

% 1.2 Automorphismen und Homomorphismen von Dynamischen Systemen, Faktoren und Erweiterungen

\begin{defn}
  Ein Homomorphismus zwischen zwei dyn. Systemen $(X, G)$ und $(X', G)$ (zweimal die gleiche Gruppe oder Monoid $G$) ist eine $G$-äquivariante stetige Abbildung $\phi : X \to X'$.
\end{defn}

% Defn 1.3
\begin{defn}
  Ein dyn. System $(Y, G)$ ist \emph{Faktor} eines dyn. System $(X, G)$, wenn es einen surjektiven Homomorphismus $(X, G) \to (Y, G)$ gibt. \\
  Man nennt $(X, G)$ dann eine \emph{Erweiterung} von $(Y, G)$.
\end{defn}

% XXX: Auslassen?
\begin{bem}
  Sei $\phi : X \to Y$ surjektiv. Dann kann man $Y$ mit der Menge der Fasern von $\phi$ identifizieren.
\end{bem}

% Thm 1.3
\begin{thm}
  Sei $\phi : (X, T) \to (Y, T)$ ein Morphismus von zyklischen Systemen.
  Wenn $x \in X$ wiederkehrend ist, dann auch $\phi(x)$. \\
  Allgemeiner: $x \in Q(y) \implies \phi(x) \in Q(\phi(y))$
\end{thm}

% Defn 1.4
\begin{defn}
  Sei $(Y, T : Y \to Y)$ ein zyklisches System, $K$ eine kompakte Gruppe und $\psi : Y \to K$ stetig. Setze
  \[
    X \coloneqq Y \times K, \quad
    T : X \to X, \enspace (y, k) \mapsto (Ty, \psi(y)k).
  \]
  Das System $(X, T)$ wird \emph{Gruppenerweiterung} von $(Y, T)$ mit $K$ oder \emph{Schiefprodukt} von $(Y, T)$ mit $K$ genannt.
\end{defn}

\begin{bem}
  Die Gr. $K$ wirkt auf $(X, T) = (Y \!\times\! K, T)$ durch Rechtstransl.:
  \[
    R : K \to \Aut(X), \enspace k \mapsto R_k, \quad
    R_k(y,k') \coloneqq (y,k'k).
  \]
  Die Homöomorphismen $R_k$ kommutieren mit $T$, sind also Automorphismen des dyn. Systems $(X, T)$.
\end{bem}

% Thm 1.4
\begin{thm}
  Sei $(X \!=\! Y \!\times\! K, T)$ eine Gruppenerw. von $(Y, T)$ und $y_0 \in Y$ wiederkehrend. Dann sind die Pkte $\Set{(y_0, k)}{k \in K}$ wiederkehrend.
\end{thm}

% XXX: "forward orbit (closure)"

\end{document}