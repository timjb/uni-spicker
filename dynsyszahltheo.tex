\documentclass{cheat-sheet}

\pdfinfo{
  /Title (Dynamische Systeme und Zahlentheorie)
  /Author (Tim Baumann)
}

\usepackage{mathabx} % \divides

\DeclareMathOperator{\Aut}{Aut} % Automorphismengruppe
\DeclareMathOperator{\End}{End} % Endomorphismenmonoid
\newcommand{\AutEnd}{\Aut\!/\!\End} % Automorphismengruppe bzw. Endomorphismenmonoid
\DeclareMathOperator{\Iso}{Iso} % Isometriegruppe
\newcommand{\clos}[1]{\overline{#1}} % topologischer Abschluss
\DeclareMathOperator{\inte}{int} % Inneres (interior)

\begin{document}

\maketitle{Dyn. Systeme in der Zahlentheorie}

% TODO:
% Van der Waerdens Theorem
% Szemerédis Theorem
% Poincarés Wiederkehrsatz
% Rados Theorem
% "Fast-Periodizität"
% gleichförmig wiederkehrend
% "proximality"
% Schur und Brauers Resultat (SB)
% Hindmans Resultat (NH)
% Grünwalds Resultat (G)
% Hilberts Resultat (DH)
% Theorem (MBR = "multiple Birkhoff Recurrence")
% Stabilität:
%   * Im Sinne von Poisson
%   * Im Sinne von Lagrange
%   * Im Sinne von Lyapunov

Dies ist eine übersetzte Zusammenfassung der ersten Kapitel des Buches "`Recurrence in Ergodic Theory and Combinatorial Number Theory"' von Harry Furstenberg.

% Kapitel 1. Wiederkehr und gleichmäßige Wiederkehr in kompakten Räumen

% 1.1. Dynamische Systeme und wiederkehrende Punkte

\begin{defn}
  Ein \emph{dynamisches System} ist ein Paar $(X, G)$ bestehend aus einem kompakten metrischen Raum $X$ und einer Gruppe oder einem Monoid $G$ mit Wirkung
  $\varphi : G \to \AutEnd(X), \, g \mapsto T_g, \, T_g(x) \coloneqq g.x$.
\end{defn}

\begin{defn}
  Ein \emph{Untersystem} eines dynamischen Systems $(X, G)$ ist eine Teilmenge $Z \subseteq X$ mit $T_g(Z) \subseteq Z$ für alle $g \in G$.
\end{defn}

\begin{bem}
  Falls $G \!=\! \Z$ oder $M \!=\! \N$, dann bezeichnen wir mit $T \coloneqq T_1$ den Erzeuger der Aktion und nennen $(X, T)$ ein \emph{zykl. System}.
\end{bem}

% Def 1.1
\begin{defn}
  Sei $X$ ein topol. Raum, $T : X \to X$ stetig.
  Ein Punkt $x \in X$ heißt \emph{wiederkehrend}, falls für für alle Umgebungen $V \subset X$ von $x$ ein $n \geq 1$ existiert mit $T^n(x) \in V$.
\end{defn}

\begin{bem}
  Sei $X$ sogar ein metrischer Raum, $x \in X$ wiederkehrend. \\
  Dann gibt es eine Folge $(n_k)$ mit $d(T^{n_k}(x), x) \to 0$ für $k \to \infty$.
\end{bem}

% nicht dort im Buch
\begin{defn}
  Sei $X$ ein topol. Raum, $T : X \to X$ stetig. Dann heißt
  \[ Q(x) \coloneqq \clos{\Set{T^n x}{n \geq 1}} \subseteq X \]
  \emph{abgeschlossener Vorwärtsorbit} von $x \in X$.
\end{defn}

% nicht explizit im Buch
\begin{lem}
  \begin{itemize}
    \item $x \in X$ ist wiederkehrend $\iff$ $x \in Q(x)$
    \item $x \in Q(y) \implies T(x) \in Q(y) \iff Q(x) \subseteq Q(y)$
    \item Die Relation $xRy \!\!\coloniff\!\! x \in Q(y)$ ist transitiv.
  \end{itemize}
\end{lem}

\begin{thm}
  Sei $X$ ein kompakter topol. Raum, $T : X \to X$ stetig. \\
  Dann gibt es einen wiederkehrenden Punkt $x \in X$.
\end{thm}
% XXX: Bemerkung: Beweis benutzt Auswahlaxiom?

% Defn 1.2
\begin{defn}
  Sei $K$ eine kompakte Gruppe, $a \in K$ und $T(x) \coloneqq ax$. Dann heißt $(K, T)$ ein \emph{Kronecker-System}.
\end{defn}

% Thm 1.2
\begin{thm}
  In einem Kronecker-System sind alle $x \in K$ wiederkehrend.
\end{thm}

% 1.2 Automorphismen und Homomorphismen von Dynamischen Systemen, Faktoren und Erweiterungen

\begin{defn}
  Ein Homomorphismus zwischen zwei dyn. Systemen $(X, G)$ und $(X', G)$ (zweimal die gleiche Gruppe oder Monoid $G$) ist eine $G$-äquivariante stetige Abbildung $\phi : X \to X'$.
\end{defn}

% Defn 1.3
\begin{defn}
  Ein dyn. System $(Y, G)$ ist \emph{Faktor} eines dyn. System $(X, G)$, wenn es einen surjektiven Homomorphismus $(X, G) \to (Y, G)$ gibt. \\
  Man nennt $(X, G)$ dann eine \emph{Erweiterung} von $(Y, G)$.
\end{defn}

% XXX: Auslassen?
\begin{bem}
  Sei $\phi : X \to Y$ surjektiv. Dann kann man $Y$ mit der Menge der Fasern von $\phi$ identifizieren.
\end{bem}

% Thm 1.3
\begin{thm}
  Sei $\phi : (X, T) \to (Y, T)$ ein Morphismus von zyklischen Systemen.
  Wenn $x \in X$ wiederkehrend ist, dann auch $\phi(x)$. \\
  Allgemeiner: $x \in Q(y) \implies \phi(x) \in Q(\phi(y))$
\end{thm}

% Defn 1.4
\begin{defn}
  Sei $(Y, T : Y \to Y)$ ein zyklisches System, $K$ eine kompakte Gruppe und $\psi : Y \to K$ stetig. Setze
  \[
    X \coloneqq Y \times K, \quad
    T : X \to X, \enspace (y, k) \mapsto (Ty, \psi(y)k).
  \]
  Das System $(X, T)$ wird \emph{Gruppenerweiterung} von $(Y, T)$ mit $K$ oder \emph{Schiefprodukt} von $(Y, T)$ mit $K$ genannt.
\end{defn}

\begin{bem}
  Die Gr. $K$ wirkt auf $(X, T) = (Y \!\times\! K, T)$ durch Rechtstransl.:
  \[
    R : K \to \Aut(X), \enspace k \mapsto R_k, \quad
    R_k(y,k') \coloneqq (y,k'k).
  \]
  Die Homöomorphismen $R_k$ kommutieren mit $T$, sind also Automorphismen des dyn. Systems $(X, T)$.
\end{bem}

% Thm 1.4
\begin{thm}
  Sei $(X \!=\! Y \!\times\! K, T)$ eine Gruppenerw. von $(Y, T)$ und $y_0 \in Y$ wiederkehrend.
  Dann sind die Pkte $\Set{(y_0, k)}{k \in K}$ wiederkehrend.
\end{thm}

% Fußnote 1 Seite 22
\begin{bem}
  Durch Erweiterung mit der zykl. Gr. $\Z_m$ kann man zeigen:
\end{bem}

% Fußnote 1 Seite 22
\begin{prop}
  Ist $x \in X$ in $(X, T)$ wiederkehrend, dann auch in $(X, T^m)$.
\end{prop}

\begin{bsp}
  Sei $T \coloneqq \R / \Z$ und $\alpha \in \R$. Dann ist das System
  \[ (T^2, (\theta, \phi) \mapsto (\theta + \alpha, \phi + 2 \theta + \alpha)) \]
  eine Gruppenerweiterung des Kronecker-Systems $(T, \theta \mapsto \theta + \alpha)$. \\
  Somit sind alle Punkte des Torus $T^2$ wiederkehrend. \\
  Aus der Wiederkehr des Punktes $(0, 0)$ erhält man:
\end{bsp}

% Thm 1.5
\begin{prop}
  Für jedes $\alpha \in \R$ und $\epsilon > 0$ gibt es eine ganzzahlige Lsg der diophantinischen Ungleichung
  $\abs{\alpha n^2 - m} < \epsilon$.
\end{prop}

\begin{bem}
  Durch Verallgemeinerung auf den $d$-dim Torus zeigt man:
\end{bem}

% Thm 1.6
\begin{prop}
  Sei $p(X) \in \R[X]$ mit $p(0) = 0$. Dann gibt es für alle $\epsilon > 0$ eine Lsg der diophantinischen Ungleichung
  $\abs{p(n) - m} < \epsilon$, $n > 0$.
\end{prop}

% Defn
\begin{defn}
  Sei $M$ ein topol. Raum und $K \subseteq \Iso(M)$ kompakt. \\
  Sei $(Y, T)$ ein zykl. System und $\psi : Y \to K$ stetig. Setze
  \[
    X \coloneqq Y \!\times\! M, \quad
    T : X \to X, \enspace (y, u) \mapsto (Ty, \psi(y)u).
  \]
  Das System $(X, T)$ heißt \emph{isometrische Erweiterung} von $(Y, T)$.
\end{defn}

% Prop 1.7
\begin{prop}
  Sei $(X, T)$ eine isom. Erweiterung von $(Y, T)$. \\
  Dann ist $X = \cup X_\alpha$, wobei $X_\alpha$ abgeschlossene $T$-invariante Teilmengen von $X$ sind, sodass das System $(X_\alpha, T|_{X_\alpha})$ Faktor einer Gruppenerweiterung von $(Y, T)$ ist.
\end{prop}

% Thm 1.8
\begin{prop}
  Sei $(X, T)$ eine isom. Erweiterung von $(Y, T)$ und $y_0 \in Y$ wiederkehrend.
  Dann sind die Pkte $\Set{(y, m)}{m \in M}$ wiederkehrend.
\end{prop}

% 1.3. Wiederkehrende Punkte von Bebutov-Systemen

% Defn 1.6
\begin{defn}
  Sei $G$ eine abz. Gruppe/Monoid und $\Lambda$ ein kompakter metr. Raum.
  Sei $\Omega \coloneqq \Lambda^G \cong \prod \Lambda$ der kompakte metrisierbare Raum der Funktionen von $G$ nach $\Lambda$. Die \emph{reguläre Wirkung} von $G$ auf $\Omega$ ist
  \[
    G \mapsto \AutEnd(\Omega), \enspace g \mapsto T_g, \quad
    T_g(\omega)(g') \coloneqq \omega(g'g).
  \]
  Ein \emph{Bebutov-System} ist ein Untersystem von $(\Omega, G)$.
\end{defn}

\begin{bem}
  Sei $\{ g_1, g_2, \ldots \} = G$ eine Abzählung von $G$. \\
  Dann ist eine Metrik auf $\Omega$ definiert durch
  \[ d(\omega, \omega') \coloneqq \sum 2^{-n} d(\omega(g_n), \omega'(g_n)). \]
\end{bem}

\begin{defn}
  Für $\omega_0 \in \Omega$ ist der Abschluss des Orbits von $\omega_0$,
  \[ X_{\omega_0} \coloneqq \clos{\Set{T_g(\omega_0)}{g \in G}}, \]
  $G$-invariant. Das dynamische System $(X_{\omega_0}, G)$ wird das von $\omega_0$ \emph{erzeugte Bebutov-System} genannt.
\end{defn}

\begin{defn}
  Ein \emph{symbolischer Fluss} ist ein Bebutov-System mit endlichem $\Lambda$ und $G \in \{ \N, \Z \}$.
  Die Elemente von $\Omega$ sind dann unendliche/doppelt-unendliche Folgen von Elementen von $\Lambda$. \\
  Man bezeichnet $\Lambda$ dann als \emph{Alphabet}.
\end{defn}

% Ausgelassen: Theorem 1.9 über die Beziehung von fast-periodischen Funktionen (im Sinne von Bohr) mit der Theorie der Kronecker-Systeme

\begin{defn}
  Ein \emph{Wort} über $\Lambda$ ist eine endl. Sequenz von Elementen aus $\Lambda$. \\
  Die Länge $\abs{w}$ eines Wortes ist die Länge der Sequenz.
\end{defn}

% Prop 1.10
\begin{prop}
  Für eine Sequenz $\omega \in \Lambda^\N$ sind äquivalent:
  \begin{itemize}
    \item $\omega$ ist wiederkehrend.
    \item Jedes Wort in $\omega$ kommt ein 2.\,Mal an einer anderen Pos. in $\omega$ vor.
    \item Jedes Wort aus $\omega$ kommt an unendlich oft in $\omega$ vor.
  \end{itemize}
\end{prop}

\begin{bem}
  Ein wiederkehrendes Wort $\omega \in \Lambda^\N$ hat die allgemeine Form
  \[ \omega = [(aw^{(1)}a)w^{(2)}(aw^{(1)}a)]w^{(3)}[(aw^{(1)}a)w^{(2)}(aw^{(1)}a)]\ldots \]
  mit $a \in \Lambda$ und Wörtern $w^{(1)}, w^{(2)}, \ldots$.
  Damit kann man zeigen:
\end{bem}

\begin{lem}[Hilbert]
  Sei $\N = B_1 \cup B_2 \cup \ldots \cup B_q$ eine Partition von $\N$ und $l \in \N_{>0}$ beliebig.
  Schreibe
  \[ P(x_1, \ldots, x_l) \coloneqq \Set{ x_{i_1} + \ldots + x_{i_k} }{0 \leq k \leq l, \, 1 \leq i_1 < \ldots i_k \leq l}. \]
  Dann gibt es $m_1 \leq m_2 \leq \ldots \leq m_l$, sodass unendlich viele Translationen von $P(m_1, \ldots, m_l)$ in demselben $B_j$ enthalten sind.
\end{lem}

\begin{bem}
  Sei $(X, T)$ ein zykl. System und $f : X \to \Lambda$ stetig. Dann ist
  \[
    (X, T) \to (\Lambda^\N, T), \quad
    x \mapsto (f(x), f(Tx), f(T^2x), \ldots)
  \]
  ein Homomorphismus zyklischer Systeme.

\end{bem}

% Thm 1.11
\begin{thm}
  Seien $\Lambda_1$, $\Lambda_2$ komp. Räume und $\phi : \Lambda_1 \to \Lambda_2$ eine Abbildung. \\
  Für $\omega \in \Lambda_1^\N$ definiere $\omega' \in \Lambda_2^\N$ durch $\omega'(n) \coloneqq f(\omega(n))$. \\
  Falls $\omega$ wiederkehrend ist und zusätzlich $f$ in allen Punkten $\omega(n)$ stetig ist, dann ist auch $\omega'$ wiederkehrend.
\end{thm}

% Ausgelassen: Beispiel

% Prop 1.12
\begin{prop}
  Sei $K$ eine komp. Gruppe und $\xi \in K^\N$ wiederkehrend. Dann ist $\eta \!\in\! K^\N$ definiert durch $\eta(n) \coloneqq \xi(n) \xi(n-1) \cdots \xi(1)$ wiederkehrend.
\end{prop}

% Ausgelassen: Beispiel

% 1.4. Gleichmäßige Wiederkehr und minimale Systeme

% Defn 1.7
\begin{defn}
  Eine Teilmenge $S$ einer abelschen topologischen Gruppe / eines Monoids heißt $G$ \emph{syndetisch}, wenn eine kompakte Menge $K \subset G$ existiert, sodass $\fa{g \in G} \ex{k \in K} gk \in S$.
\end{defn}

% TODO: gute deutsche Übersetzung
\begin{bem}
  Eine Teilmenge $\{ s_1 < s_2 < \ldots \} = S \subset \N$ ist genau dann syndetisch, wenn die Größe $s_i - s_{i-1}$ der "`Lücken"' zw. Elementen aus $S$ beschränkt ist.
  Solche Mengen heißen auch \emph{relativ dicht}.
\end{bem}

% Defn 1.8
\begin{defn}
  Sei $(X, G)$ ein dyn. System. Ein Punkt $x \in X$ heißt \emph{gleichmäßig wiederkehrend}, falls für alle Umgebungen $V \subset X$ von $x$ die Menge $\Set{g \in G}{g.x \in V}$ syndetisch ist.
\end{defn}

% Defn 1.9
\begin{defn}
  Ein dyn. System $(X, G)$ heißt \emph{minimal}, wenn es keine echte abgeschl. Teilmenge von $X$ gibt, die inv. unter der $G$-Wirkung ist.
\end{defn}

% Lem 1.13 und 1.14
\begin{lem}
  Sei $(X, G)$ ein dyn System. Es sind äquivalent: \\
  \inlineitem{$(X, G)$ ist minimal} \quad
  \inlineitem{$\fa{x \in X}$ der Orbit $Gx$ ist dicht in $X$} \\
  \inlineitem{$\fa{\emptyset \not= V \subset X \text{ offen}} \ex{\text{endlich viele Elemente } g_1, \ldots, g_n \in G}$}
  \[ g_1^{-1} V \cup \ldots \cup g_n^{-1} V = X. \]
\end{lem}

% Thm 1.15
\begin{thm}
  Sei $(X, G)$ ein minimales dynamisches System. \\
  Dann sind alle $x \in X$ gleichmäßig wiederkehrend.
\end{thm}

\begin{bem}
  Aus Zorns Lemma folgt: Jedes dyn. System besitzt ein minimales Untersystem. Es folgt:
  % XXX: Warum wird im Buch extra erwähnt, dass $X$ kompakt ist? Das ist doch in der Definition eines dyn. Systems enthalten?
\end{bem}

% Thm 1.16
\begin{thm}
  Jedes dyn. System hat einen gleichm. wiederkehrenden Pkt.
\end{thm}

% Thm 1.17, etwas erweitert
\begin{thm}
  Sei $(X, G)$ ein dyn. System, $x \in X$. Dann sind äquivalent: \\
  \inlineitem{$x$ ist glm. wiederkehrend.} \enspace
  \inlineitem{Das Untersystem $\clos{Gx}$ ist minimal.}
\end{thm}

% Thm 1.18
\begin{thm}
  In einem Kronecker-System ist jeder Pkt glm. wiederkehrend.
\end{thm}

% Thm 1.19 und 1.20
\begin{thm}
  Sei $(X, T)$ eine Gruppenerw. oder isometrische Erweiterung von $(Y, T)$ mit Projektion $\pi : (X, T) \to (Y, T)$ und $y_0 \in Y$ glm. wiederkehrend.
  Dann sind die Pkte $\pi^{-1}(y_0)$ glm. wiederkehrend.
\end{thm}

\begin{bem}
  Es folgt durch Betr. eines dyn. Systems auf dem $k$-dim Torus:
\end{bem}

% Thm 1.21
\begin{thm}
  Seien $p_1(X), \ldots, p_k(X) \in \R[X]$ Polynome. \\
  Für alle $\epsilon > 0$ ist die Teilmenge der nat. Zahlen, die
  \[
    \abs{e^{2 \pi i p_1(n) - e^{2 \pi i p_1(0)}}} < \epsilon, \ldots,
    \abs{e^{2 \pi i p_k(n) - e^{2 \pi i p_k(0)}}} < \epsilon
  \]
  gleichzeitig erfüllen, syndetisch.
\end{thm}

% 1.5 "Substitution Minimal Sets and Uniform Recurrence in Bebutov Systems"

% Prop 1.22
\begin{prop}
  Sei $\Lambda$ ein endl. Alphabet. Ein Punkt $\omega \in \Lambda^\N$ oder $\omega \in \Lambda^\Z$ ist genau dann glm. wiederkehrend, wenn für jedes Wort in $\omega$ die Menge der Positionen, an denen dieses Wort auftaucht, syndetisch ist.
\end{prop}

\begin{bem}
  Eine wiederkehrende Sequenz $\omega$ in der allgemeinen Form
  \[ \omega = [(aw^{(1)}a)w^{(2)}(aw^{(1)}a)]w^{(3)}[(aw^{(1)}a)w^{(2)}(aw^{(1)}a)]\ldots \]
  ist glm. wiederkehrend, wenn die Länge der $w^{(n)}$ beschränkt ist. \\
  Es existieren also nichtperiodische, glm. wiederkehrende Sequenzen.
\end{bem}

\begin{defn}
  Ein \emph{Vokabular} ist eine Teilmenge $V$ aller Wörter über einem Alphabet $\Lambda$, für die gilt:
  \begin{itemize}
    \item Jedes Teilwort eines Wortes aus $V$ ist ebenfalls in $V$.
    \item Jedes Wort in $V$ ist Teilwort eines längeren Wortes aus $V$.
  \end{itemize}
\end{defn}

\begin{defn}
  Sei $V$ ein Vokabular. Sei dann $S(V) \subset \Lambda^\N$ die abgeschl., trans- lations-inv. Menge aller Sequenzen, die nur Wörter aus $V$ enthalten.
\end{defn}

\begin{lem}
  Sei $V$ ein Vokabular, das folgende Bedingung erfüllt: Für alle $l \in \N$ gibt es ein $L \in \N$, sodass für alle Wörter $w \in V$ der Länge $\abs{w} = l$ gilt: $w$ ist in jedem Wort $v \in V$ der Länge $\abs{v} = L$ enthalten. \\
  Dann sind alle Sequenzen in $S(V)$ glm. wiederkehrend.
\end{lem}

\begin{bem}
  Sei $\Lambda = \{ a_1, \ldots, a_r \}$ und $w_1, \ldots, w_r$ Wörter über $\Lambda$, die jeweils alle Buchstaben aus $\Lambda$ enthalten. Sei $V_1 \coloneqq \Lambda$ die Menge der Wörter der Länge 1 über $\Lambda$ und induktiv $V_n$ die Menge der Wörter, die aus einem Wort $w \in V_{n-1}$ durch simultane Substitution
  \[ a_1 \to w_1, \ldots, a_r \to w_r \]
  hervorgehen und deren Teilwörter. Das Vokabular $V \coloneqq \cup_{n \in \N} V_n$ heißt dann \emph{substitution minimal set}. Alle Sequenzen in $S(V)$ sind gleichmäßig wiederkehrend.
\end{bem}

\begin{bem}
  Seien $d_1, d_2, \ldots \in \N$ mit $d_n \divides d_{n+1}$ für alle $n$.
  Schreibe nun $\Z$ als disjunkte Vereinigung
  \[
    \Z = \bigsqcup_{k=1}^\infty (d_k \Z + a_k) \quad
    \text{mit $a_1, a_2, \ldots \in \Z$.}
  \]
  Sei $\Lambda$ kompakt und $(\lambda_i)_{i \in \N}$ eine Folge in $\Lambda$. Setze
  $\omega(n) \coloneqq \lambda_k$, falls $n \in d_k \Z + a_k$.
  Wenn nun $[-N, N] \subset \sqcup_{k=1}^l (d_k \Z + a_k)$, dann gilt für alle $n \in [-N, N], q \in \Z$: $\omega(n) = \omega(n + q \cdot d_l)$. Somit tritt jedes Wort in $\omega$ periodisch auf und $\omega$ ist glm. wiederkehrend.
\end{bem}

% 1.6. Kombinatorische Anwendungen

% Defn 1.10
\begin{defn}
  Eine Teilmenge $R \subset \N$ oder $R \subset \Z$ heißt \emph{dick}, wenn sie Intervalle $\cinterval{a_n}{a_n + n}$ beliebiger Länge enthält.
\end{defn}

\begin{bem}
  Eine Menge ist genau dann syndetisch, wenn ihr Schnitt mit jeder dicken Menge nichtleer ist.
  Eine Menge ist genau dann dick, wenn ihr Schnitt mit jeder syndetischen Menge nichtleer ist.
\end{bem}

% Defn 1.11
\begin{defn}
  Eine Teilmenge $A \subset \N$ oder $A \subset \Z$ heißt \emph{stückw. syndetisch}, wenn sie Schnitt einer dicken und einer syndetischen Menge ist.
\end{defn}

% Ausgelassen: Thm 1.23, da nur Spezialfall von 1.24

% Thm 1.24
\begin{thm}
  Sei $B \subseteq \N$ oder $B \subseteq \Z$ stückw. syndetisch, $B = B_1 \cup \nldots \cup B_q$ eine Partition.
  Dann ist auch eine Menge $B_i$ stückweise syndetisch.
\end{thm}

% 1.7. Mehr diophantische Approximation

\begin{bem}
  Seien $\tau_1, \ldots, \tau_n$ die kanonischen Erzeuger von $H_1(T^n) \cong \Z^n$.
\end{bem}

% Lem 1.25
\begin{lem}
  Habe $T : T^d \to T^d$ die Form
  \[
    T(\theta_1, \ldots, \theta_d) \coloneqq
    (\theta_1 + \alpha, \theta_2 + f_1(\theta_1), \ldots, \theta_d + f_{d-1}(\theta_1, \ldots, \theta_{d-1}))
  \]
  mit $\alpha$ irrational, $f_i : T^i \!\to\! T$ stetig mit $(f_i)_*(\tau_i) \!\not=\! 0$ für $i \!=\! 1, \nldots, d{-}1$. \\
  Dann ist $(T^d, T)$ ein minimales dynamisches System.
\end{lem}

% Thm 1.26
\begin{thm}
  Sei $p(X) \in \R[X]$ mit mind. einem irrationalen Koeffizienten. \\
  Dann gibt es $\forall \, \epsilon > 0$ eine Lsg der Ungleichung $\abs{p(n) - m} < \epsilon$.
\end{thm}

% 1.8 Nicht wandernde Transformationen und Wiederkehr

\begin{defn}
  Sei $X$ ein kompakter metrischer Raum und $T : X \to X$ stetig. \\
  Eine Teilmenge $A \subset X$ heißt \emph{wandernd}, wenn die Urbilder $T^{-1}(A)$, \nldots{}, $T^{-n}(A)$, \nldots{} disjunkt von $A$ (und damit auch voneinander) sind.
\end{defn}

\begin{defn}
  Ein dyn. System $(X, T)$ heißt \emph{nicht wandernd}, wenn keine offene, nichtleere Menge $A \subset X$ wandernd ist.
\end{defn}

% Ausgelassen: Hinreichende Bedingung: $X$ ist Support eines inv. endl. Maßes

\begin{defn}
  Eine Teilmenge $A \!\subset\! X$  heißt \emph{nirgends dicht}, falls $\inte(\clos{A}) = \emptyset$.
\end{defn}

\begin{defn}
  Eine Teilmenge $A \!\subset\! X$ heißt \emph{mager}, wenn sie Vereinigung abzählbar vieler nirgends dichter Mengen ist.
\end{defn}

% Thm 1.27
\begin{thm}
  Sei $(X, T)$ nicht wandernd. Dann ist die Menge der nicht wiederkehrenden Punkte in $X$ mager.
\end{thm}

% Ausgelassen: Lem 1.28 über halbstetige Funktionen

\end{document}