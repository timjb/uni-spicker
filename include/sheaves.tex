% Garbentheorie-Makros

% Konzepte

% Konkrete Garben
\renewcommand{\O}{\mathcal{O}} % Strukturgarbe der stetigen Funktionen
\newcommand{\constSh}[1]{\underline{#1}} % konstante Garbe

% Kategorien von Garben
% Notation übernommen von http://stacks.math.columbia.edu/download/sheaves.pdf
\newcommand{\PShSet}{\mathbf{PSh}} % Prägarben von Mengen
\newcommand{\ShSet}{\mathbf{Sh}} % Garben von Mengen
\newcommand{\PShAb}{\mathbf{PAb}} % Prägarben von abelschen Gruppen
\newcommand{\ShAb}{\mathbf{Ab}} % Garben von abelschen Gruppen

% Bezeichnungen für Variablen, die für Garben stehen
\newcommand{\Fais}{\mathcal{F}} % Faisceau-F (Garbe auf französisch)
\newcommand{\Garb}{\mathcal{G}} % Garben-G
\newcommand{\Harb}{\mathcal{H}} % Garben-H
\newcommand{\Karb}{\mathcal{H}} % Garben-K (Kern)
\newcommand{\Carb}{\mathcal{H}} % Garben-C (Kokern)
\newcommand{\Iarb}{\mathcal{H}} % Garben-I (Image)
