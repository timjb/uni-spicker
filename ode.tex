\documentclass{cheat-sheet}

\pdfinfo{
  /Title (Zusammenfassung Gewöhnliche Differentialgleichungen)
  /Author (Tim Baumann)
}

% Kleinere Klammern
\delimiterfactor=701

\newcommand{\D}{\mathcal{D}}

\begin{document}

\maketitle{Zusammenfassung Gew. Diff'gleichungen}

% Kapitel 1. Einführung

% Kapitel 1.1. Beispiele

\iffalse
\begin{bsp}
  Gesucht: Funktion $y : \R \to \R$ mit $\fa{t \in \R} \dot{y}(t) = y(t)$\\
\end{bsp}

\begin{lsg}
  $y(t) = c \cdot e^t$ für $c \in \R$ beliebig. Wenn man als Anfangsbe- dingung $y(0) = 1$ fordert, erhält man eine eindeutige Lösung ($c = 1$).
\end{lsg}

\begin{bsp}
  Gesucht: Lösung von $\left(\dot{y}(t)\right)^2 + \left(y(t)\right)^2 = a$ für $a \in \R$\\
\end{bsp}

\begin{lsg}
  Anzahl der Lösungen hängt von $a$ ab:
  \begin{itemize}
    \miniitem{0.47\linewidth}{Falls $a < 0$: keine reelle Lsg}
    \miniitem{0.51\linewidth}{Falls $a = 0$: Einzige Lsg $y(t) = 0$}
    \item Falls $a > 0$: Lsgn: $y(t) = \sqrt{a} \cos(t + \phi)$ für $\phi \in \R$ bel., $y(t) = \pm \sqrt{a}$
  \end{itemize}
\end{lsg}

% Populationsmodell aus der Biologie/Chemie
\begin{bsp}
  Sei $p(t)$ ist Populationsgröße zur Zeit $t$.
  Angenommen, $\tfrac{\dot{p}(t)}{p(t)} = a$ ist konstant, also $\dot{p}(t) = p(t)$. Sei $p(t_0) = p_0$.
\end{bsp}

\begin{lsg}
  $p(t) = p_0 e^{(t-t_0) a}$
\end{lsg}

% 1837
\begin{bsp}[Verhulst-Modell]
  Gesucht: Lösung zu
  \[ \dot{p}(t) = a_0 p(t) - a_1 \left(p(t)\right)^2 \]
\end{bsp}

\begin{lsg}
  $p(t) = \frac{a_0}{a_1 (1 - c e^{-a_0 t})}$
\end{lsg}

% Mechanik: Mathematisches Pendel
% Skizze: Pendel der
% * Masse $m$
% * Länge $l$
% * Auslenkungswinkel $\phi(t)$ zum Zeitpunkt $t$
% * Position $l \phi(t)$ zum Zeitpunkt $t$
% * Geschwindigkeit $v(t) = \dot{p}(t) = l \dot{\phi}(t)$
% * $a(t) = \ddot{p}(t) = \dot{v}(t) = l \ddot{\phi}(t)$ Beschleunigung
% $F_E = - mg$, $F = ma$
% $F_T$ -- tangentiale Komponente der Gewichtskraft
% $F_T = - mg \sin(\phi)$
% => Gleichung $-mg \sin(\phi) = m a(t) = m l \ddot{t}$, also $\ddot{\phi}(t) = -\tfrac{g}{l} \sin(\phi(t))$
\fi

% Vorlesung vom 9.4.2014

% Literatur
% Alle Bücher haben den Titel "`Gewöhnliche Differentialgleichungen"'
% * B. Aulbach, 2004
% * H. Henser, 2009
% * L. Grüne, O. Junge, 2009
% * W. Walter, 2000

% Kapitel 1.2. Klassifikation von Differentialgleichungen (DGLn)

\begin{defn}[Klassifikation von DGLn]\mbox{}\\
  \begin{enumerate}[label=(\Roman*),leftmargin=2em]
    \item \emph{Gewöhnliche} DGL: Gesucht ist Funktion in einer Variable\\
    \emph{Partielle} DGL: Gesucht ist Funktion in mehreren Variablen
    \item \emph{Ordnung} einer DGL: Höchste Ableitung der gesuchten Funktion, die in Gleichung vorkommt
    \item \emph{Explizite DGL}: Gleichung der Form
    $y^{(k)} {=} f(t, y, \dot{y}, ..., y^{(k{-}1)})$
    \emph{Implizite DGL}: Allgemeinere Form $F(t, y, \dot{y}, ..., y^{(k)}) = 0$
    \item \emph{Skalare} DGL: Gesucht ist Funktion mit Wert in $\R$\\
    \emph{$n$-dimensionale} DGL: Gesuchte Funktion hat Wert in $\R^n$
    \item \emph{Lineare} DGL: Gleichung hat die Form
    \[ a_k(t) y^{(k)}(t) + a_{(k-1)}(t) y^{k-1}(t) + ... + a_1(t) \dot{y}(t) + a_0(t)y(t) = 0 \]
    \item \emph{Autonome} DGL: Gleichung der Form $F(y, \dot{y}, ..., y^{(k)}) = 0$\\
    (keine Abhängigkeit von $t$, Zeitinvarianz)
  \end{enumerate}
\end{defn}

\iffalse
% I)

Unterscheidung zwischen gewöhnliche DGL und partielle DGLn

Beispiele für gewöhnliche DGL
$\dot{y}(t) = h y(t)$
$(\dot{y}(t))^2 + (y(t))^2 = a$

Beispiele für partielle DGLn:

$y_t = \alpha y_{xx} + y$, wobei $y_t(t,x) = \tfrac{\partial}{\partial t} y(t, x)$, $y_{xx}(t, x) = \tfrac{\partial^2}{\partial x^2} y(t, x)$

% II)

Unterscheidung zwischen DGLn 1. Ordnung, DGLn 2. Ordnung und DGLn $k$-ter Ordnung

Beispiel für DGL 1. Ordnung:
$\dot{y} = \alpha y(t)$

Beispiel für DGL 2. Ordnung:
$\ddot{\phi}(t) = - \tfrac{\delta}{e} \sin(\phi(t))$

Beispiel für DGL $k$-ter Ordnung:
$F(t, y(t), \dot{y}(t), ..., y^{(k)}(t)) = 0$

% III)

Unterscheidung zischen expliziten und impliziten DGLn

Beispiel für explizite DGLn:
$\dot{y}(t) = \alpha y(t)$
$\ddot{\phi}(t) = - \tfrac{g}{e} \sin(\phi(t))$
$y^{(k)}(t) = f(t, y, \dot{y}, ..., y^{(k-1)})$

Beispiele für implizite DGLn:
$(\dot{y}(t))^2 + (y(t))^2 = a$
$F(t, y, \dot{y}, ..., y^{(k)}(t)) = 0$

Oder (Gleichungen gehören zusammen)
$\dot{y}_2(t) + y_1(t) = f_1(t)$
$y_2(t) = f_2(t)$
(differentiell-algebraische Gleichung)

% IV)
Unterscheidung zwischen Skalaren DGLn und $n$-dimensionalen DGLn (Systeme von DGLn)

Beispiel für Skalare DGL:
$\dot{y}(t) = f(t, y(t))$, wobei $f : \R \times \R \to \R$ gegeben ist.

Beispiel für ein System von DGLn:
$\dot{y}(t) = f(t, y(t))$, wobei $f : \R \times \R^n \to \R^n$ gegeben und $y : \R \to \R^n$ gesucht

% V)
Unterscheidung zwischen linearen und nicht linearen DGLn

Beispiele für lineare DGLn:
$\dot{y}(t) = \alpha y(t)$
$\dot{y}(t) = A y(t) + g(t)$, $A \in \R^{n \times n}$
$a_k(t) y^{(k)}(t) + a_{(k-1)}(t) y^{k-1}(t) + ... + a_1(t) \dot{y}(t) + a_0(t)y(t) = 0$

Beispiele für nicht lineare DGLn:
$\ddot{\phi}(t) = - \tfrac{g}{e} \sin(\phi(t))$
$(\dot{y}(t))^2 + (y(t))^2 = a$

% VI)
Unterscheidung zwischen autonomen und nicht autonomen DGLn

Beispiele für autonome DGLn:
\begin{itemize}
  \item $\dot{y} = \alpha y(t)$
  \item $(\dot{y}(t))^2 + (y(t))^2 = a$
  \item $\dot{y}(t) = f(y(t))$
  \item $F(y(t), \dot{y}(t), ..., y^{(k)}(t)) = 0$
\end{itemize}

Beispiele für nicht autonome DGLn:
\begin{itemize}
  \item $\dot{y} = \alpha y(t) + e^{t}$
  \item $(\dot{y}(t))^2 + (y(t))^2 0= a + t^2$
  \item $\dot{y}(t) = f(t, y(t))$
  \item $F(t, y(t), \dot{y})(t), ..., y^{(k)}(t)) = 0$
\end{itemize}

Unterschied: Autonome DGLn hängen nicht explizit von der Zeit $t$ ab

\begin{defn}
  Es sei $\D \subset \R \times \R^n$ offen, $f : \D \to \R^n$ und $(t_0, y_0) \in \D$. Das System von Gleichungen
  \begin{align*}
    \dot{y}(t) &= f(t, y(t))\\
    y(t_0) &= y_0
  \end{align*}
  heißt \emph{Anfangswertproblem} (AWP).
  % Ausgelassen: Obiges Gleichungssystem hat die Nummer (1.1)
\end{defn}
\fi

\begin{nota}
  Sei im Folgenden $I$ stets ein Intervall in $\R$.
\end{nota}

\begin{defn}
  \begin{itemize}
    \item Es sei $\D \subset \R \times \R^n$, $f : \D \to \R^n$. Eine differenzierbare Funktion $y : I \to \R^n$ heißt \emph{Lösung} von $\dot{y} = f(t, y)$, falls für alle $t \in I$ gilt: $\dot{y}(t) = f(t, y(t))$.
    \item Es sei $\D \subset \R \times \underbrace{\R^n \times ... \R^n}_{k \text{ mal}}$, $f : \D \to \R^n$. Eine $k$-mal differenzierbare Funktion $y : I \to \R^n$ heißt \emph{Lösung} von $y^{(k)} = f(t, y, \dot{y}, ..., y^{(k-1)})$, falls für alle $t \in I$ gilt:
    \[ y^{(k)} = f(t, y(t), \dot{y}(t), ..., y^{(k-1)}(t)) \]
  \end{itemize}
\end{defn}

% 1.1.
\begin{satz}
  \begin{itemize}
    \item Ist $y : I \to \R^n$ eine Lösung von $y^{(k)} = f(t, y, \dot{y}, ..., y^{(k-1)})$ (1.2), dann ist
    \begin{align*}
      (y_1, ..., y_k) : I &\to \R^{kn}\\
      t &\mapsto (y_1(t), ..., y_k(t)) = (y(t), \dot{y}(t), ..., y^{(k-1)}(t))
    \end{align*}
    eine Lösung von System (1.3)
    \begin{align*}
      \dot{y}_1 = y_2\\
      \dot{y}_2 = y_3\\
      \dot{y}_3 = y_4\\
      \vdots\\
      \dot{y}_{k-1} = y_k\\
      \dot{y}_k = f(t, y_1, y_2, ..., y_{k-1}, y_k)
    \end{align*}
    \item Ist $(y_1, ..., y_k) : I \to \R^n$ eine Lösung von (1.3), dann ist $y = y_1 : I \to \R^n$ eine Lösung von (1.2).
  \end{itemize}
\end{satz}

% 1.2.
\begin{satz}
  \begin{itemize}
    \item Ist $y : I \to \R^n$ eine Lösung von AWP (1.1), dann ist
    \begin{align*}
      (y_1, y_2) : I &\to \R^{n+1}\\
      t &\mapsto (y_1(t), y_2(t)) = (t, y(t))
    \end{align*}
    eine Lösung des AWP (1.4)
    \begin{align*}
      \dot{y}_1(t) = 1, & y_1(t_0) = t_0
      \dot{y}_2(t) = f(y_1(t), y_2(t)), & y_2(t_0) = y_0
    \end{align*}
    \item Ist $(y_1, y_2) : I \to \R^{n+1}$ eine Lösung von (1.4), dann ist $y = y_2 : I \to \R^n$ eine Lösung von (1.1).
  \end{itemize}
\end{satz}

\end{document}