\documentclass{cheat-sheet}

\pdfinfo{
  /Title (Zusammenfassung Topologie)
  /Author (Tim Baumann)
}

\newcommand{\Tau}{\mathcal{T}} % Großes Tau
\DeclareMathOperator{\inte}{int} % Inneres (interior)
\DeclareMathOperator{\grad}{grad} % Gradient
\DeclareMathOperator{\dive}{div} % Divergenz
\DeclareMathOperator{\divergence}{div} % Divergenz
\newcommand{\HM}{\mathcal{H}} % Hausdorff-Maß
\usepackage{bbm} % Für 1 mit Doppelstrich (Indikatorfunktion)
\newcommand{\ind}{\mathbbm{1}} % Indikatorfunktion
\DeclareMathOperator{\dist}{dist} % Entfernung (distance)
\newcommand{\scp}[2]{\langle #1 , #2 \rangle} % Skalarprodukt
\DeclareMathOperator*{\esssup}{ess\,sup} % Essentielles Supremum

% Abschnittsnummerierung einschalten (entgegen cheat-sheet.cls)
\makeatletter
  % Abstand von Nummerierung und Titel verringern
  \renewcommand*{\@seccntformat}[1]{\csname the#1\endcsname\hspace{0.2cm}}
\makeatother
\renewcommand{\thesection}{\arabic{section}.} % Punkt nach Nummer
\setcounter{secnumdepth}{1}

% Kleinere Klammern
\delimiterfactor=701

% Integral mit Strich durch, siehe
% http://www.tex.ac.uk/cgi-bin/texfaq2html?label=prinvalint
\def\Xint#1{\mathchoice
   {\XXint\displaystyle\textstyle{#1}}%
   {\XXint\textstyle\scriptstyle{#1}}%
   {\XXint\scriptstyle\scriptscriptstyle{#1}}%
   {\XXint\scriptscriptstyle\scriptscriptstyle{#1}}%
   \!\int}
\def\XXint#1#2#3{{\setbox0=\hbox{$#1{#2#3}{\int}$}
     \vcenter{\hbox{$#2#3$}}\kern-.5\wd0}}
%\def\ddashint{\Xint=}
\def\dashint{\Xint-}

% Mittelwerts-Integrale (mean value integrals)
\newcommand{\mymvint}[2]{{\textstyle \dashint\limits_{#1}^{#2}}}
\newcommand{\MVInt}[4]{\mymvint{#1}{#2} #3 \,\d #4}

\begin{document}

\maketitle{Zusammenfassung Partielle DGLn}

% Vorlesung vom 8.4.2014

% Kapitel 1.
\section{Einleitung}

% Kapitel 1.1. Partielle Differentialgleichungen und klassische Lösungen

% Ausgelassen: Definition ODE

\begin{defn}
  Eine \emph{partielle Differentialgleichung} (PDGL) hat die Form
  \begin{align*}
    E(x, u(x), Du(x), ..., D^k u(x)) = 0 \quad \text{in $\Omega \subset \R^n$ offen}, \tag{$\star$}
  \end{align*}
  wobei $E : \Omega \times \R \times \R^n \times ... \times \R^{n^k} \to \R$ gegeben und $u : \Omega \to \R$ gesucht ist.
  Die höchste Ableitungsordnung von $u$, die in $E$ vorkommt, heißt \emph{Ordnung} der PDGL.
\end{defn}

% Nicht definiert: Multiindizes

\begin{defn}
  Eine PDGL von der Ordnung $k$ heißt
  \begin{itemize}
    \item \emph{linear}, falls sie folgende Form besitzt:
    \[ \sum_{\mathclap{\abs{\alpha} \leq k}} a_\alpha(x) D^{\alpha} u(x) - f(x) = 0 \]
    \item \emph{semilinear}, falls sie linear in der höchsten Ableitungsordnung ist, man sie also schreiben kann als
    \[ \sum_{\mathclap{\abs{\alpha} = k}} a_\alpha(x) D^{\alpha} u(x) + E_{k-1}(x, u(x), Du(x), ..., D^{k-1} u(x)) = 0. \]
    \item \emph{quasilinear}, falls sie sich schreiben lässt als
    \begin{align*}
      & \sum_{\mathclap{\abs{\alpha} = k}} a_\alpha(x, u(x), D u(x), ..., D^{k-1} u(x)) D^{\alpha} u(x)\\[-3pt]
      +\, & E_{k-1}(x, u(x), D u(x), ..., D^{k-1} u(x)) = 0.
    \end{align*}
    \item sonst \emph{voll nichtlinear}.
  \end{itemize}
\end{defn}

\begin{bem}
  $\{\text{ lineare PDGLn }\} \subsetneq \{\text{ semilineare PDGLn }\} \subsetneq \{\text{ quasilineare PDGLn }\} \subsetneq \{\text{ PDGLn }\}$
\end{bem}

% Ausgelassen: Schwierigkeitsfaustregel

% Typeinteilung für lineare PDGLn 2. Ordnung
\begin{defn}[Typeinteilung für lineare PDGLn 2. Ordnung]
  Seien $a_{ij}, b_i$, $c, f : \Omega \to \R$ ($i, j \in \{ 1, ..., n \}$) vorgegebene Fktn. auf $\Omega \subset \R^n$ offen.
  \begin{itemize}
    \item Die lineare PDGL
    \[ \sum_{1 \leq i, j \leq n} a_{ij}(x) D_i D_j u(x) + \sum_{1 \leq j \leq n} b_j(x) D_j u(x) + c(x) u(x) + f(x) = 0 \]
    heißt \emph{elliptisch}, falls die $(n \times n)$-Matrix $(a_{ij})_{1 \leq i,j \leq n}$ für alle $x \in \Omega$ positiv definit ist.
    \item Die lineare PDGL
    \[ D_1 D_1 u(x) - \sum_{\mathclap{2 \leq i, j \leq n}} a_{j}(x) D_i D_j u(x) + \sum_{\mathclap{1 \leq i \leq n}} b_i(x) D_i u(x) + c(x) u(x) + f(x) = 0 \]
    heißt \emph{hyperbolisch}, falls die $(n{-}1) \times (n{-}1)$-Matrix $(a_{ij})_{2 \leq i, j \leq n}$ für alle $x \in \Omega$ positiv definit ist.
    \item Die lineare PDGL
    \[ D_1 u(x) - \sum_{\mathclap{2 \leq i, j \leq n}} a_{ij}(x) D_i D_j u(x) + \sum_{\mathclap{2 \leq i \leq n}} b_i(x) D_i u(x) + c(x) u(x) + f(x) = 0 \]
    heißt \emph{parabolisch}, falls die $(n{-}1) \times (n{-}1)$-Matrix $(a_{ij})_{2 \leq i,j \leq n}$ für alle $x \in \Omega$ positiv definit ist.
  \end{itemize}
\end{defn}

% Modellfälle: Laplace-Gleichung, Wellengleichung, Wärmeleitungsgleichung

% Ausgelassen: Zielsetzung der Theorie der PDGLn

\begin{defn}
  Eine Funktion $u : \Omega \to \R$ heißt \emph{klassische Lösung}, falls $u \in \mathcal{C}^k(\Omega)$ und die Differentialgleichung ($\star$) überall in $\Omega$ erfüllt ist.
\end{defn}

% Kapitel 1.2. Einige Beispiele partieller Differentialgleichungen

% (ausgelassen)

% Vorlesung vom 10.4.2014

\subsection{Grundlagen}

\begin{nota}
  Für $\Omega \subset \R^n$ schreibe
  \[
    V \Subset \Omega
    \quad \text{für} \quad
    \text{$V \subset \R^n$ mit $\overline{V}$ kompakt und $\overline{V} \subset \Omega^{\circ} $.}
  \]
\end{nota}

\begin{nota}
  Seien $f : \R^n \to \R$ und $F = (F_1, ..., F_n)^T : \R^n \to \R^n$ Funktionen. Dann heißt
  \begin{itemize}
    \item $\dive F \coloneqq \sum_{i=1}^n D_i F_i : \R^n \to \R$ \emph{Divergenz} von $F$,
    \item $\grad f \coloneqq \nabla f \coloneqq (\partial_1 f, ..., \partial_n f)^T : \R^n \to \R^n$ \emph{Gradient} von $f$,
    \item $\Delta$ mit $\Delta f = \dive (\grad f) = \sum_{i=1}^n D_i D_i f$ \emph{Laplace-Operator}.
  \end{itemize}
\end{nota}

% Kapitel 2.1. Vorbereitung

\begin{satz}[Transformationssatz]
  Sei $T : \Omega \to T(\Omega)$ für $\Omega \subset \R^n$ ein $\mathcal{C}^1$-Diffeo, dann gilt für $f : T(\Omega) \to \ER$
  \[ f \in L^1(T(\Omega)) \iff (f \circ T) \circ \abs{\det(DT)} \in L^1(\Omega) \quad \text{mit} \]
  \[ \Int{T(\Omega)}{}{f}{x} = \Int{\Omega}{}{(f \circ T) \cdot \abs{\det(DT)}}{x}. \]
\end{satz}

\begin{bsp}[Polarkoordinaten]
  Sei $f \in L^1(B_r(K))$. Dann ist $f$ auf fast jeder Sphäre $\partial B_\rho(K)$ für $\rho \in \cinterval{0}{r}$ integrierbar und es gilt
  \[ \Int{B_r(x)}{}{f(x)}{x} = \Int{0}{r}{\Int{\partial B_\rho(x_0)}{}{\!\!\!f}{S}}{\rho} \] % dH^{n-1}
\end{bsp}

\begin{satz}[Gauß]
  Sei $\Omega \subset \R^n$ beschränkt, offen mit $\mathcal{C}^1$-Rand $\partial \Omega$. Ist $F \in \mathcal{C}^0(\overline{\Omega}, \R^n) \cap \mathcal{C}^1(\Omega, \R^n)$ mit $\dive F \in L^1(\Omega)$, so gilt
  \[ \Int{\Omega}{}{\dive F}{x} = \Int{\partial \Omega}{}{(F \circ \nu)}{S}, \]
  wobei $\nu$ der äußere Einheitsnormalenvektor ist.
\end{satz}

% 1. Übungsblatt, 1. Aufgabe
\begin{kor}
  Sei $\Omega \subset \R^n$ beschränkt, offen mit $\mathcal{C}^1$-Rand $\partial \Omega$. Sind $f, g \in \mathcal{C}^1(\overline{\Omega})$, dann gilt die partielle Integrationsregel
  \[ \Int{\Omega}{}{D_i f g}{x} = - \Int{\Omega}{}{f D_i g}{x} + \Int{\partial \Omega}{}{f g \nu^i}{\HM^{n-1}} \]
  Sind $f, g \in \mathcal{C}^2(\overline{\Omega})$, dann gelten die Greenschen Formeln
  \begin{align*}
    \Int{\Omega}{}{Df \cdot Dg}{x} &= - \Int{\Omega}{}{f \Delta g}{x} + \Int{\partial \Omega}{}{f D_{\nu} g}{\HM^{n-1}}\\
    \Int{\Omega}{}{(f \Delta g - g \Delta f)}{x} &= \Int{\partial \Omega}{}{(f D_{\nu} g - g D_{\nu} f)}{\HM^{n-1}}
  \end{align*}
\end{kor}

% 1. Übungsblatt, 2. Aufgabe: Differentiation von parameterabhängigen Integralen
\begin{prop}[Diff. parameterabh. Integrale]
  Sei $\Omega \subset \R^n$ messbar mit $\abs{\Omega} < \infty$, $I = \ointerval{a}{b} \subset \R$ und $f : \Omega \times I \to \R$. Angenommen,
  \begin{itemize}
    \item $f(x, \blank) \in \mathcal{C}^1(I)$ für fast alle $x \in \Omega$,
    \item $f(\blank, t) \in L^1(\Omega)$, $\tfrac{\partial f}{\partial t}(\blank, t) \in L^1(\Omega)$ für alle $t \in I$ und
    \item für alle $t \in I$ gibt es $\epsilon > 0$ sodass $\ointerval{t-\epsilon}{t+\epsilon} \subset I$ und
    \[ \sup_{s \in \ointerval{t-\epsilon}{t+\epsilon}} \abs{\tfrac{\partial f}{\partial t}(\blank, s)} \in L^1(\Omega). \]
  \end{itemize}
  Dann ist die Abbildung
  \[
    g : I \to \R, \qquad
    t \mapsto \Int{\Omega}{}{f(x, t)}{x}
  \]
  wohldefiniert und stetig differenzierbar mit
  \[ \tfrac{\partial g}{\partial t}(t) = \Int{\Omega}{}{\tfrac{\partial f}{\partial t}(x,t)}{x}. \]
\end{prop}

\begin{bem}
  Die Voraussetzungen sind erfüllt, wenn $\Omega$ offen und beschränkt ist, $f(x, \blank) \in \mathcal{C}^1(I)$ für alle $x \in \Omega$ und $f, \tfrac{\partial f}{\partial t} \in \mathcal{C}(\overline{\Omega} \times I)$.
\end{bem}

\begin{nota}
  Bezeichne mit $\mathcal{L}^n$ das Lebesgue-Maß auf dem $\R^n$. Für messbare Teilmengen $A \subset \R^n$ schreibe $\abs{A} \coloneqq \mathcal{L}^n(A)$.
\end{nota}

\begin{bsp}
  Zwischen dem Volumen von Kugeln und Sphären im $\R^n$ bestehen folgende Zusammenhänge:
  \[
    \abs{B_r(0)} = r^n \cdot \abs{B_1(0)}
    \quad \text{und} \quad
    \abs{B_r(0)} = \tfrac{r}{n} \cdot \enspace\Int{\mathclap{\partial B_r(0)}}{}{\enspace1}{S}
  \]
\end{bsp}
\vspace{-12pt}
\begin{nota}
  $\omega_n \coloneqq \mathcal{L}^n(B_1(0)) = \frac{\pi^{\tfrac{n}{2}}}{\Gamma(\tfrac{n}{2} + 1)}$
\end{nota}

\begin{samepage}

\begin{nota}
  Sei $f : \Omega / M \to \R$ integrierbar für $\Omega \subset \R^n$ messbar mit $\mathcal{L}^k(\Omega) \in \ointerval{0}{\infty}$ bzw. $M \subset \R^n$ eine $k$-dimensionale Untermannigfaltigkeit mit $\Int{M}{}{1}{S} \in \ointerval{0}{\infty}$
  \[
    \MVInt{\Omega}{}{f(x)}{x} \coloneqq \tfrac{1}{\abs{\Omega}} \Int{\Omega}{}{f(x)}{x}
    \quad \text{bzw.} \quad
    \MVInt{M}{}{f(x)}{x} \coloneqq \tfrac{1}{\abs{M}} \Int{M}{}{f(x)}{x}
  \]
  heißen \emph{Mittelwerte} von $f$ auf $\Omega$ bzw. $M$.
\end{nota}


\subsection{Funktionenräume}

% Ausgelassen: Funktionenräume $k$-fach stetiger Funktionen / mit kompaktem Träger / stetig auf den Rand fortsetzbar, etc.

\begin{defn}
  Eine Funktion $f : S \to \R$, $S \subset \R^n$ heißt (lokal) \emph{Hölder-stetig} in $x_0 \in S$ zum Exponenten $\alpha \in \ocinterval{0}{1}$ mit Hölderkonstante $C_{x_0} \in \R_{\geq 0}$, falls für alle $x \in S$ (bzw. $x \in K$ für ein $K \Subset S$) gilt:
  \[ \abs{f(x) - f(x_0)} \leq C_{x_0} \abs{x - x_0}^\alpha \]
\end{defn}

\end{samepage}

\begin{defn}
  Die \emph{Hölder-Seminorm} von $f : S \to \R$ ist
  \[ [f]_{\mathcal{C}^{0,\alpha}(S)} \coloneqq \sup_{x,x_0 \in S} \frac{\abs{f(x_0) - f(x)}}{\abs{x-x_0}^\alpha}. \]
\end{defn}

\begin{defn}[\emph{Hölder-Räume}]
  Sei $\Omega \subset \R^n$ offen, $\alpha \in \ocinterval{0}{1}$.
  \begin{itemize}
    \item $\mathcal{C}^{0,\alpha}(\Omega) \coloneqq \Set{ f \in \mathcal{C}(\Omega) }{ [f]_{\mathcal{C}^{0,\alpha}(K)} < \infty \text{ für alle komp. } K \Subset S }$
    \item $\mathcal{C}^{0,\alpha}(\overline{\Omega}) \coloneqq \Set{f \in \mathcal{C}(\overline{\Omega})}{[f]_{\mathcal{C}^{0,\alpha}(\overline{\Omega})} < \infty}$.
    \item $\mathcal{C}^{k,\alpha}(\Omega) \coloneqq \{ f \in \mathcal{C}^k(\Omega) \mid [D^\beta f]_{\mathcal{C}^{0,\alpha}(K)} < \infty$\\für alle kompakten $K \Subset \Omega$ und Multiindizes $\beta$ mit $\abs{\beta} = k \}$
    \item $\mathcal{C}^{k,\alpha}(\overline{\Omega}) \coloneqq \{ f \in \mathcal{C}^k(\overline{\Omega}) \mid$ alle Ableitungen bis zur Ordnung $k$ von $f$ sind beschränkt und $[D^\beta f]_{\mathcal{C}^{0,\alpha}(\overline{\Omega})} < \infty$ für alle Multiindizes $\beta$ mit $\abs{\beta} = k \}$
  \end{itemize}
\end{defn}

% Ausgelassen: Bemerkung über unterschiedliche Konventionen

\begin{bem}
  Es gelten die Inklusionen $\mathcal{C}(\Omega) \supsetneq \mathcal{C}^{0,\alpha}(\Omega) \supsetneq \mathcal{C}^1(\Omega)$, aber i.\,A. $\mathcal{C}^1(\overline{\Omega}) \not\subset \mathcal{C}^{0,1}(\overline{\Omega})$.
\end{bem}

\begin{bem}
  Die Räume $\mathcal{C}^k(\overline{\Omega})$ und $\mathcal{C}^{k,\alpha}$ sind Banachräume bzgl.
  \begin{align*}
    \norm{f}_{\mathcal{C}^k(\overline{\Omega})} &\coloneqq \sum_{0 \leq \abs{\beta} \leq k} \sup_{\overline{\Omega}} \abs{D^\beta f},\\
    \norm{f}_{\mathcal{C}^{k,\alpha}(\overline{\Omega})} &\coloneqq \norm{f}_{\mathcal{C}^{k}(\overline{\Omega})} + \sum_{\abs{\beta} = k} [D^\beta f]_{\mathcal{C}^{0,\alpha}(\overline{\Omega})}.
  \end{align*}
\end{bem}

\begin{defn}
  Sei $A \subseteq \R^n$ messbar und $p \in \cinterval{1}{\infty}$. Für $f : A \to I$ messbar sei
  \[
    \norm{f}_{L^p(A)} \coloneqq \begin{cases}
      \left( \Int{A}{}{\abs{f}^p} \right)^{1/p} & \text{falls } p < \infty,\\
      \esssup_A \abs{f} & \text{falls } p = \infty.
    \end{cases}
  \]
  Der \emph{Lebesgue-Raum} $L^p(A)$ ist der Raum aller Äquivalenzklassen von fast-überall übereinstimmenden Funktionen, für die $\norm{\blank}_{L^p(A)}$ endlich ist. Der Raum $L_{\text{loc}}^p(A)$ ist der Raum aller Funktionen $A \to \R^n$, die für alle offenen $O \Subset A$ zu $L^p(O)$ gehören.
\end{defn}

\begin{bem}
  $L^p(A)$ ist ein Banachraum mit der Norm $\norm{\blank}_{L^p(A)}$.
\end{bem}


\subsection{Glättungen}

\begin{defn}
  Ein \emph{Glättungskern} auf $\R^n$ ist eine nicht-negative, radialsym- metrische Funktion $\eta \in \mathcal{C}_0^\infty(B_1(0))$ mit $\Int{\R^n}{}{\eta}{x} = 1$.
\end{defn}

\begin{defn}
  Der \emph{Standardglättungskern} ist die Funktion
  \[ \eta(x) \coloneqq C \cdot \exp \left(\tfrac{1}{\abs{x}^2 - 1}\right) \cdot \ind_{B_1(0)}(x) \]
  mit Normierungskonstante $C$. Für $\epsilon > 0$ ist der dazugehörige skalierte Glättungskern gegeben durch
  \[ \eta_{\epsilon}(x) \coloneqq \epsilon^{-n} \eta(x/\eta). \]
  Alle Glättungskern-Eigenschaften bleiben bei Skalierung erhalten.
\end{defn}

\begin{nota}
  $\Omega_{\epsilon} \coloneqq \Set{ x \in \Omega }{ \dist(x, \partial \Omega) > \epsilon }$
\end{nota}

\begin{defn}
  Sei $\Omega \subset \R^n$ offen, $\epsilon > 0$. Für $f \in L_{\text{loc}}^1$ heißt die Funktion
  \[
    f_\epsilon : \Omega_{\epsilon} \to \R, \quad
    x \mapsto \eta_\epsilon * f(x) \coloneqq \Int{\mathclap{B_{\epsilon}(x)}}{}{\enspace\eta_{\epsilon}(x-y) f(y)}{y}
    \quad \text{\emph{$\epsilon$-Glättung} von $f$}
  \]
\end{defn}

\begin{satz}[Eigenschaften von Glättungen]
  Sei $\Omega \subset \R^n$ offen, $\epsilon > 0$ und $f \in L_{\text{loc}}^1(\Omega)$. Dann gilt
  \begin{itemize}
    \item Regularität: $f_{\epsilon} \in \mathcal{C}^\infty(\Omega_\epsilon)$ mit $D^{\alpha} f_\epsilon = (D^\alpha \eta_\epsilon) * f$ für beliebige Multiindizes $\alpha \in \N^n$.
    \item Ist $D_i f$ stetig auf $\Omega$, so gilt $D_i (f_\epsilon) = (D_i f)_\epsilon$ auf $\Omega_\epsilon$. % Vertauschbarkeit mit Ableitungen
    % Nächsten zwei Punkte: Erhaltung von Normen
    \item Falls $f \in \mathcal{C}^\alpha(\Omega)$ für ein $\alpha \in \ocinterval{0}{1}$, so gilt $f_\epsilon \in \mathcal{C}^\alpha(\Omega_\epsilon)$ mit derselben Hölderkonstante.
    \item Falls $f \in L^p(\Omega)$ für $p \in \cinterval{0}{\infty}$, so gilt $\norm{f_\epsilon}_{L^p(\Omega_\epsilon)} \leq \norm{f}_{L^p(\Omega)}$.
    % Nächste drei Punkte: Approximation
    \item $f_\epsilon \xrightarrow{\epsilon \to 0} f$ fast-überall in $\Omega$.
    \item Falls $f \in \mathcal{C}(\Omega)$, so konvergiert $f_\epsilon$ gleichmäßig gegen $f$ für $\epsilon \to 0$ auf kompakten Teilmengen von $\Omega$,
    \item Falls $f \in L_{\text{loc}}^p(\Omega)$ für $p \in \cointerval{1}{\infty}$, so gilt $f_\epsilon \xrightarrow{\epsilon \to 0} f$ in $L_{\text{loc}}^p(\Omega)$.
    \item Abschätzung der Approximationsgüte: Ist $Du \in L^p(\Omega)$, so gilt
    \[ \norm{f - f_\epsilon}_{L^p(\Omega_\epsilon)} \leq \epsilon \cdot \norm{Df}_{L^p{\Omega}}. \]
  \end{itemize}
\end{satz}


\subsection{Hausdorff-Maß}

\begin{defn}
  Sei $A \subset \R^n$, $k \in \cointerval{0}{\infty}$, $\delta > 0$. Das \emph{approximierende Maß} $H_\delta^k$ von $A$ ist definiert als
  \[
    \HM_\delta^k(A) \coloneqq \inf \Set{ \sum_{i=1}^\infty \omega_k r_i^k }{ A \subset \bigcup_{i=1}^\infty \overline{B_{r_i}(x_i)}, r_i < \delta }
  \]
\end{defn}

\begin{bem}
  $\HM_\delta^k(A)$ ist monoton fallend in $\delta$.
\end{bem}

\begin{defn}
  Das \emph{$k$-dimensionale Hausdorff-Maß} $\HM^k$ von $A$ ist
  \[ \HM^k(A) \coloneqq \lim_{\delta \to 0} \HM_\delta^k(A). \]
\end{defn}

\begin{prop}
  \begin{itemize}
    \item Für $\delta > 0$ ist $\HM_\delta^k$ ein Maß auf $\R^n$.
    \item $\HM^k$ ist ein Maß auf $\R^n$
    \item Bewegungsinvarianz: $\HM^k(x + T(A)) = \HM^k(A)$ für $x {\in} \R^n$, $T {\in} O(n)$.
    \item Ist $f : A \to \R^m$ Lipschitz-stetig mit Konstante $L_f$, so gilt
    \[ \HM^k(f(A)) \leq L_f^k \HM^k(A). \]
    \item Skalierungsverhalten: $\HM^k(\lambda A) = \lambda^k \HM^k(A)$
    \item Spezialfälle: $\HM^0$ ist ein Zählmaß, $\HM^n = \mathcal{L}^n$ und $\HM^k \equiv 0$.
  \end{itemize}
\end{prop}

\begin{lem}
  Sei $A \subset \R^n$ und $0 \leq k < k' < \infty$.
  \begin{itemize}
    \item Ist $\HM^k(A) < \infty$, so gilt $\HM^{k'}(A) = 0$.
    \item Ist $\HM^{k'}(A) > 0$, so gilt $\HM^k(A) = \infty$.
  \end{itemize}
\end{lem}

\begin{defn}
  Die \emph{Hausdorff-Dimension} von $A \subset \R^n$ ist
  \begin{align*}
    \dim_H(A) &\coloneqq \inf \Set{k \in \R_{\geq 0}}{\HM^k(A) = 0}\\
    &= \sup \Set{k \in \R_{\geq 0}}{\HM^k(A) = \infty}.
  \end{align*}
\end{defn}

\begin{samepage}

\begin{prop}
  Für die Cantor-Menge $C \subset \R$ gilt $\dim_H(C) = \frac{\log 2}{\log 3}$.
\end{prop}


% Kapitel 2.
\section{Laplace- und Poisson-Gleichung}

\end{samepage}

\begin{defn}
  Die \emph{Laplace-} bzw. \emph{Poisson-Gleichung} ist die Gleichung
  \[ \Delta u = 0 \quad \text{bzw.} \quad \Delta u = f \qquad \text{auf $\Omega \subset \R^n$}. \]
\end{defn}

% Kapitel 2.2. Harmonische Funktionen

\iffalse
  Physikalische Motivation/Herleitung
  % siehe auch: Skript Schmidt 

  Beschreibung von Gleichgewichtszuständen von physikalischen Zuständen $u$ (Temperatur / Konzentration)

  Gleichgewicht: "`Nettofluss"' durch den Rand $V \subset \Omega$:
  Jedes (glatt berandete) Testvolumen verschwindet.

  $F \coloneqq Flussdichte von $u

  $0 = \Int{\partial V}{}{F \circ \nu}{S} \overset{\text{Gauss}}{=} \Int{V}{}{\dive F}{x}$

  Es folgt: $\dive F = 0$, da $V$ beliebig

  Typischerweise $F = -a Du$ ($a > 0$).

  Es folgt: $-a \Delta u = 0$

  In der Physik: Diffusion von Feldern, 1. Ficksche Gesetz, Wärmeleitung, Fouriersches Gesetz

  Definition von harmonischen Funktionen und der Fundamentallösung
\fi

\begin{defn}
  Sei $\Omega \subset \R^n$ offen, $u \in \mathcal{C}^2(\Omega)$. Man nennt $u$
  \begin{itemize}
    \item \emph{harmonisch}, falls $\Delta u = 0$ in $\Omega$ gilt.
    \item \emph{subharmonisch}, falls $\Delta u \geq 0$ in $\Omega$ gilt.
    \item \emph{superharmonisch}, falls $\Delta u \leq 0$ in $\Omega$ gilt.
  \end{itemize}
\end{defn}

\begin{bspe}
  \begin{itemize}
    \item Affine Funktionen sind harmonisch.
    \item Sei $A \in \R^{n \times n}$. Definiere $u(x) \coloneqq x \cdot Ax$. Dann gilt $\Delta u = \spur A$, also $\Delta u = 0 \iff \spur A = 0$.
    % Ausgelassen: Harmonische Funktionen aus Ansatz mit Trennung der Variablen
    \item Real- und Imaginärteil von holomorphen Fktn. sind harmonisch.
  \end{itemize}
\end{bspe}

\iffalse
  % Herleitung der Fundamentallösung: siehe Evans, PDE, Abschnitt 2.2.1 a) oder Schmidt, Abschnitt 2.1.3
  Konstruktion von rotationssymmetrischen harmonischen Funktionen, \dh{}
  \[ \Delta u = 0 \]
  mit $u(x) = v(r)$ mit $v : \R \to \R$ und $r = \norm{x} = (x_1^2 + ... + x_n^2)^{1/2}$.
  Beachte (für $i \in \{ 1, ..., n \}, x \not= 0$):
  \begin{itemize}
    \item $D_i r = \frac{x_i}{(x_1^2 + ... + x_n^2)^{1/2}}$, also $\abs{Dr}^2 = \sum_{i=1}^n (D_i r)^2 = 1$
    \item $D_i D_i r = D_i (D_i r) = \frac{1}{r} - \frac{x_i x_i}{r^3}$
    \item $\Delta r = \sum_{i=1}^n (\frac{1}{r} - \frac{x_i^2}{r^3}) = \frac{n-1}{r}$
    \item $\Delta u = \sum_{i=1}^n D_i (v^i(r) D_i r) = v''(r) \sum_{i=1}^n (D_i r)^2 + v'(r) \sum_{i=1}^n D_i D_i r = v''(r) + \frac{n-1}{r} v'(r) = 0$
  \end{itemize}

  Für $v' \not= 0$ kann man diese ODE explizit lösen

  \[ (\log \abs{v'(r)})' = \frac{v''(r)}{v'(r)} = \frac{1-n}{r} = (1-n)(\log r)' = (\log r^{1-n})' \]

  Also (Integrieren, Exponentialfunktion anwenden): $v'(r) = c r^{1-n}$ für $c \in \R$.
  Somit $v(r) = c_1 \cdot \log r + c_2$, wenn $n = 2$
  Somit $v(r) = c_1 \cdot r^{2-n} + c_2$, wenn $n \geq 3$
\fi

\begin{defn}
  Die Funktion $\Phi : \R^n \setminus \{ 0 \} \to \R$, definiert durch
  \[
    \Phi(x) \coloneqq \begin{cases}
      - (2\pi)^{-1} \log \abs{x}, & \text{wenn $n = 2$}\\
      (n (n{-}2) \, \omega_n)^{-1} \abs{x}^{2-n}, & \text{wenn $n \geq 3$}
    \end{cases}
  \]
  heißt \emph{Fundamentallösung} der Laplacegleichung.
\end{defn}

\begin{bem}
  \begin{itemize}
    \item $\Phi$ ist radialsymmetrisch, \dh{} für alle $x_1, x_2 \in \R^n \setminus \{ 0 \}$ mit $\norm{x_1} = \norm{x_2}$ gilt $\Phi(x_1) = \Phi(x_2)$.
    \item $\Phi$, $\abs{D\Phi} \in L^1(B_R(0))$ für alle $R > 0$ aber $\abs{D^2 \phi} \not\in L^1(B_1(0))$.
    \item Die Konstanten wurden so gewählt, dass gilt:
    \[ - \Int{\mathclap{\partial B_r(0)}}{}{D \Phi \cdot \nu}{\HM^{n-1}} = 1 \quad \text{für alle $r > 0$}. \]
  \end{itemize}
\end{bem}

% 2.1. (Schmidt: Lemma 2.3)
\begin{lem}
  Sei $\Omega \subset \R^n$ offen, $B_R(x_0) \subset \Omega$, $u \in \mathcal{C}^2(\Omega)$. Für
  \[
    \phi : \ointerval{0}{R} \to \R, \quad
    r \mapsto \enspace \MVInt{\mathclap{\partial B_r(x_0)}}{}{\enspace u}{\HM^{n-1}}
    \qquad \text{gilt dann}
  \]
  \begin{itemize}
    \miniitem{0.40 \linewidth}{$\lim_{r \to 0} \phi(r) = u(x_0)$}
    \miniitem{0.48 \linewidth}{$\phi'(r) = \tfrac{r}{n} \enspace\MVInt{\mathclap{B_r(x_0)}}{}{\enspace\Delta u(x)}{x}$}
  \end{itemize}
\end{lem}

% Vorlesung vom 15.4.2014

% 2.2.
\begin{kor}[Mittelwertseigenschaft]
  Sei $\Omega \subset \R^n$ offen, $B_r(x_0) \Subset \Omega$ und $u \in \mathcal{C}^2(\Omega)$. Dann gilt:
  \[
    0 = \Delta u \enspace\implies\enspace
    u(x_0) = \enspace\MVInt{\mathclap{\partial B_r(x_0)}}{}{\enspace u}{\HM^{n-1}}
    \quad \text{und} \quad
    u(x_0) = \enspace\MVInt{\mathclap{B_r(x_0)}}{}{\enspace u}{\HM^{n-1}}
  \]
  In diesen Gleichungen darf man $=$ durch $\leq$, $<$, $\geq$ oder $>$ ersetzen.
\end{kor}

% 2.3.
\begin{satz}
  Sei $\Omega \subset \R^n$ offen. Dann sind äquivalent:
  \begin{itemize}
    \item $u$ ist harmonisch, \dh{} es gilt $\Delta u = 0$ in $\Omega$.
    \item $u$ erfüllt die sphärische Mittelwertseigenschaft, \dh{} es gilt
    \[
      u(x_0) = \enspace\MVInt{\mathclap{\partial B_r(x_0)}}{}{\enspace u}{\HM^{n-1}}
      \qquad \text{für alle Kugeln $B_r(x_0) \Subset \Omega$.}
    \]
    \item $u$ erfüllt die Mittelwertseigenschaft auf Kugeln, \dh{} es gilt
    \[
      u(x_0) = \enspace\MVInt{\mathclap{B_r(x_0)}}{}{\enspace u}{\HM^{n-1}}
      \qquad \text{für alle Kugeln $B_r(x_0) \Subset \Omega$.}
    \]
  \end{itemize}
\end{satz}

\begin{bem}
  Die Äquivalenz gilt auch unter den schwächeren Voraussetzungen $u \in \mathcal{C}(\Omega)$ oder $u \in L^1(\Omega)$.
\end{bem}

% 2.4.
\begin{satz}
  Sei $\Omega \subset \R^n$ offen, beschränkt und $u \in \mathcal{C}^2(\Omega) \cap \mathcal{C}^0(\overline{\Omega})$ subharmonisch in $\Omega$, \dh{} $\Delta u \geq 0$ in $\Omega$. Dann gilt
  \begin{itemize}
    \item Das \emph{schwache Maximumsprinzip}: $\max_{\overline{\Omega}} u = \max_{\partial \Omega} u$
    \item Das \emph{starke Maximumsprinzip}: Ist $\Omega$ zusammenhängend und existiert $x_0 \in \Omega$ mit $u(x_0) = \max_{\overline{\Omega}} u$, so ist $u$ konstant.
  \end{itemize}
  % Entsprechende Maximumsprinzipien für superharmonische Funktionen
\end{satz}

% Ausgelassen: Definition "`zusammenhängend"'

\begin{bem}
  Sei $\Omega \subset \R^n$ beschränkt, offen, zusammenhängend und $u \in \mathcal{C}^2(\Omega) \cap \mathcal{C}(\overline{\Omega})$ harmonisch. Dann gilt
  \[ \min_{\partial \Omega} u < \max_{\partial \Omega} u \enspace\implies\enspace \min_{\partial \Omega} u < u < \max_{\partial \Omega} u \text{ auf $\Omega$}. \]
\end{bem}

\begin{kor}[Eindeutigkeit]
  Sei $\Omega \subset \R^n$ offen, beschränkt und $u, v \in \mathcal{C}^2(\Omega) \cap \mathcal{C}(\overline{\Omega})$. Dann ist $u = v$, falls gilt:
  \[
    \left\{ \begin{array}{ll}
      \Delta u = \Delta v & \text{in $\Omega$}\\
      u = v & \text{auf $\partial \Omega$}
    \end{array} \right.
  \]
\end{kor}

\begin{bem}[Stetige Abhängigkeit von Randwerten]
  Gilt lediglich $\Delta u = \Delta v$ in $\Omega$, aber nicht $u = v$ auf $\partial \Omega$, so gilt immerhin
  \[ \max_{\overline{\Omega}} \abs{u - v} = \max_{\partial \Omega} \abs{u - v}. \]
\end{bem}

% Ausgelassen: Bemerkung zur stetigen Abhängigkeit von den Randwerten

% 2.6.
\begin{satz}[Harnack-Ungleichung]
  Sei $\Omega \subset \R^n$ offen, $V \Subset \Omega$ offen, zusammenhängend. Dann gibt es eine Konstante $c = c(\Omega, V)$, sodass
  \[
    \sup_{V} u \leq c \cdot \inf_{V} u
    \qquad \text{für alle harmonischen Fktn. $u : \Omega \to \R_{\geq 0}$.}
  \]
  % Insbesondere sind alle Funktionswerte vergleichbar.
\end{satz}

% Vorlesung vom 24.4.2014

% Thema: Regularität

% 2.8.
\begin{satz}
  Sei $\Omega \subset \R^n$ offen und erfülle $u \in \mathcal{C}(\Omega)$ die Mittelwert- Eigenschaft auf Sphären, \dh{}
  \[
    u(x_0) = \enspace\MVInt{\mathclap{\partial B_r(x_0)}}{}{\enspace u}{\HM^{n-1}}
    \quad \text{für alle Kugeln $B_r(x_0) \Subset \Omega$.}
  \]
  Dann gilt $u(x) = u_\epsilon(x)$ für alle $x \in \Omega$ und $\epsilon < \dist(x, \partial \Omega)$.\\
  Insbesondere ist $u \in \mathcal{C}^\infty(\Omega)$ und harmonisch.
\end{satz}
% Achtung: Keine Aussage des Satzes über Stetigkeit der Randwerte!

\begin{kor}
  Obiger Satz gilt auch, wenn $u \in \mathcal{C}(\Omega)$ die Mittelwert-Eigenschaft auf Kugeln erfüllt, \dh{}
  \[
    u(x_0) = \enspace\MVInt{\mathclap{B_r(x_0)}}{}{\enspace u}{\HM^{n-1}}
    \quad \text{für alle Kugeln $B_r(x_0) \Subset \Omega$.}
  \]
\end{kor}

\begin{defn}
  Eine Folge von Funktionen $(f_n)_{n \in \N}$ auf einem topologischen Raum $X$ \emph{konvergiert lokal gleichmäßig} gegen $f : X \to \R$, falls es zu jedem Punkt $x \in X$ eine Umgebung $U_x$ von $x$ gibt, sodass $f_n$ auf $U_x$ gleichmäßig gegen $f$ konvergiert.
\end{defn}

\begin{kor}[Konvergenzsatz von Weierstraß]
  Sei $\Omega \subset \R^n$ offen, zusammenhängend und $(u_k)_{k \in \N}$ eine Folge harmonischer Funktionen auf $\Omega$, die lokal gleichmäßig gegen eine Funktion $u$ konvergiert.\\
  Dann ist $u$ harmonisch auf $\Omega$.
\end{kor}

\begin{kor}[Harnackscher Konvergenzsatz]
  Sei $\Omega \subset \R^n$ offen und $(u_k)_{k \in \N}$ eine monoton wachsende Folge harmonischer Funktionen auf $\Omega$. Gibt es ein $x_0 \in \Omega$, sodass $(u_k(x_0))_{k \in \N}$ beschränkt (und damit konvergent) ist, so konvergiert $(u_k)$ lokal gleichmäßig gegen eine harmonische Funktion auf $\Omega$.
\end{kor}

\begin{satz}[von Hermann Weyl]
  Sei $\Omega \subset \R^n$ offen und $u \in L_{\text{loc}}^1(\Omega)$ mit
  \[
    \Int{\Omega}{}{u \cdot \Delta \phi}{x} = 0
    \quad \text{für alle $\phi \in \mathcal{C}_0^\infty(\Omega)$.}
  \]
  Dann gibt es eine harmonische Funktion $\tilde{u} : \Omega \to \R$ mit $u(x) = \tilde{u}(x)$ für fast alle $x \in \Omega$.
\end{satz}

\begin{satz}[Innere Abschätzung für Ableitungen harmonischer Fktn]\mbox{}\\
  Sei $\Omega \subset \R^n$ offen und $u \in \mathcal{C}^2(\Omega)$ harmonisch. Dann gilt für jeden Multiindex $\alpha$ mit $\abs{\alpha} = k \in \N_0$ und jede Kugel $B_r(x_0) \Subset \Omega$:
  \[
    \abs{D^\alpha u(x_0)} \leq C(n, k) r^{-n-k} \norm{u}_{L^1(B_r(x_0))}
    \quad \text{mit } C(n, k) \coloneqq \tfrac{(2^{n+1} nk)^k}{\omega_n}.
  \]
\end{satz}

\begin{satz}[Liouville]
  Sei $u \in \mathcal{C}^2(\Omega)$ harmonisch.
  \begin{itemize}
    \item Ist $u$ beschränkt, so ist $u$ konstant.
    \item Gilt $\limsup_{\abs{x} \to \infty} \tfrac{\abs{u(x)}}{\abs{x}^{k+1}} = 0$, so ist $u$ ein Polynom, dessen Grad $\leq k$ ist.
  \end{itemize}
\end{satz}

\begin{defn}
  Sei $\Omega \subset \R^n$ offen. Eine Funktion $f : \Omega \to \R$ heißt \emph{analytisch} in $x \in \Omega$, falls $f$ sich lokal durch ihre Taylorreihe darstellen lässt, also ein $r \in \ointerval{0}{\dist(x, \partial \Omega)}$ existiert mit
  \[
    f(y) = \sum_{\alpha \in \N^n} \tfrac{1}{\alpha!} D^\alpha f(x) (y-x)^\alpha
    \qquad \text{für alle $y \in B_r(x)$.}
  \]
\end{defn}

\begin{satz}
  Sei $\Omega \subset \R^n$ offen und $u \in \mathcal{C}^2(\Omega)$. Wenn $u$ harmonisch ist, dann auch analytisch.
\end{satz}

% Vorlesung vom 29.4.2014

\begin{prob}
  Sei $\Omega \subset \R^n$ offen, (beschränkt), regulär und $f : \Omega \to \R$ und $g : \partial \Omega \to \R$ stetig. Gesucht ist $u : \overline{\Omega} \to \R$ mit
  \[
    (2.1) \left\{ \begin{array}{rlll}
      - \Delta u &=& f &\quad \text{in $\Omega \subset \R^n$,}\\
      u &=& g &\quad \text{auf $\partial \Omega$.}
    \end{array} \right.
  \]
\end{prob}

% Bekannt: Lösungen dazu sind eindeutig

% 2.15.
\begin{satz}[Greensche Darstellungsformel]
  Sei $\Omega \subset \R^n$ beschränkt, offen mit $\mathcal{C}^1$-Rand und $h \in \mathcal{C}^2(\Omega) \cap \mathcal{C}^1(\overline{\Omega})$ mit $\Delta h {\in} L^1(\Omega)$. Es gilt für $x {\in} \Omega$:
  \begin{align*}
    h(x) = - \Int{\Omega}{}{\Phi(x-y) \Delta h(y)}{y} &+ \Int{\mathclap{\partial \Omega}}{}{\Phi(x {-} y) \cdot Dh(y) \cdot \nu}{S(y)}\\
    &- \Int{\mathclap{\partial \Omega}}{}{h(y) D_y \Phi(x{-}y) \cdot \nu}{S(y)}
  \end{align*}
\end{satz}

\begin{bem}
  Für Randpunkte $x \in \partial \Omega$ gilt:
  \begin{align*}
    \tfrac{1}{2} h(x) = - \Int{\Omega}{}{\Phi(x-y) \Delta h(y)}{y} &+ \Int{\mathclap{\partial \Omega}}{}{\Phi(x {-} y) \cdot Dh(y) \cdot \nu}{S(y)}\\
    &- \Int{\mathclap{\partial \Omega}}{}{h(y) D_y \Phi(x{-}y) \cdot \nu}{S(y)}
  \end{align*}
\end{bem}

% Ausgelassen: Physikalische Interpretation, Merkregel

\begin{kor}[Darstellungsformel für Lsgn in $\R^n$]
  Sei $f {\in} \mathcal{C}_0^2(\R^n)$, setze
  \[ u : \R^n \to \R, \qquad x \mapsto (\Phi * f)(x) \coloneqq \Int{\R^n}{}{\Phi(x-y) f(y)}{y}. \]
  Dann gilt: $u \in \mathcal{C}^2(\R^n)$ und $- \Delta u = f$ in $\R^n$.
\end{kor}

\begin{bem}
  \begin{itemize}
    \item Für $n = 2$ ist die Lösung potentiell unbeschränkt.
    \item Für $n \geq 3$ ist diese Lsg beschränkt und erfüllt $\enspace\lim_{\mathclap{\abs{x} \to \infty}} u(x) = 0$.
  \end{itemize}
\end{bem}

\begin{prop}
  Jede andere beschränkte Lösung von $- \Delta u = f$ auf $\R^n$ unterscheidet sich nur durch eine additive Konstante.
\end{prop}

\begin{defn}
  Sei $\Omega \subset \R^n$ offen. Eine \emph{Greensche Funktion} für $\Omega$ ist eine Funktion
  $G : \Set{ (x, y) \in \Omega \times \Omega }{ x \not= y } \to \R$, falls für alle $x \in \Omega$ gilt:
  \begin{itemize}
    \item Die \emph{Korrektorfunktion} $y \mapsto G(x, y) - \Phi(x - y)$ ist von der Klasse $\mathcal{C}^2(\Omega) \cap \mathcal{C}^1(\overline{\Omega})$ und ist harmonisch in $\Omega$. % (für $x=y$ forsetzbar)
    \item Die Funktion $G(x, \blank)$ hat Nullrandwerte auf $\partial \Omega$, \dh{} es gilt $\lim_{y \to y_0} G(x, y) = 0$ für alle $y_0 \in \partial \Omega \cup \{ \infty \}$.
  \end{itemize}
\end{defn}

% Ausgelassen: Bemerkung zum Neumann-RWP und der Greenschen Funktion zweiter Art (mit verschwindender Normalenableitung)

\begin{bem}
  \begin{itemize}
    \item Ist $\Omega$ beschränkt, so ist die Greensche Funktion eindeutig.
    \item Die Funktion $G(x, \blank)$ ist in $\mathcal{C}^2(\Omega \setminus \{ x \}) \cap \mathcal{C}(\overline{\Omega} \setminus \{ x \})$ und hat die gleiche Singularität wie $y \mapsto \Phi(x - y)$.
  \end{itemize}
\end{bem}

\begin{satz}
  Sei $\Omega \subset \R^n$ offen, beschränkt, mit $\mathcal{C}^1$-Rand $\partial \Omega$. Ist $u \in \mathcal{C}^2(\Omega) \cap \mathcal{C}^1(\overline{\Omega})$ eine Lösung von (2.1) und ist $G$ die Greensche Funktion für $\Omega$ (falls existent), dann gilt für alle $x \in \Omega$:
  \[ u(x) = \Int{\Omega}{}{G(x,y) f(y)}{y} - \Int{\partial \Omega}{}{g(y) \cdot D_y G(x,y) \cdot \nu}{S(y)}. \]
\end{satz}

% Vorlesung vom 6.5.2014

% 2.18
\begin{lem}
  Sei $\Omega \subset \R^n$ offen, $G$ die Greensche Funktion für $\Omega$ und $B_r(x) \Subset \Omega$. Für $f \in \mathcal{C}^2(\Omega) \cap \mathcal{C}^1(\overline{\Omega})$ gilt dann:
  \[ \lim_{\epsilon \to 0} \quad\enspace \Int{\mathclap{\partial B_{\epsilon}(x)}}{}{\left( G(x, y) Df(y) - f(y) D_y G(x, y) \right) \cdot \nu(y) }{\HM^{n-1}(y)} = f(x). \]
\end{lem}

% 2.19
\begin{satz} % Symmetrie der Greenschen Funktion
  Ist $G$ die Greensche Funktion zu $\Omega \subset \R^n$ offen, beschränkt mit $\mathcal{C}^1$-Rand $\partial \Omega$, so gilt $G(x, y) = G(y, x)$ für alle $x, y \in \Omega$ mit $x \not= y$.
\end{satz}

\begin{kor}
  Sei $G$ die Greensche Funktion zu $\Omega \subset \R^n$ offen, beschränkt mit $\mathcal{C}^1$-Rand $\partial \Omega$, so ist die Funktion $x \mapsto G(x, y)$ harmonisch auf $\Omega \setminus \{ y \}$.
\end{kor}

\begin{defn}
  Sei $B_r(a) \subset \R^n$ eine Kugel, $x \in \R^n \setminus \partial B_r(a)$. Dann heißt
  \[
    x^* \coloneqq a + r^2 \frac{x - a}{\norm{x - a}^2} \in \R^n \setminus \partial B_r(a) \quad
    \text{\emph{Spiegelungspunkt} von $x$.}
  \]
\end{defn}

\begin{bem}
  \begin{minipage}{0.12 \linewidth}Es gilt:\end{minipage}
  \begin{minipage}{0.63 \linewidth}
    \begin{itemize}
      \miniitem{0.65 \linewidth}{$\norm{x - a} \cdot \norm{x^* - a} = r^2$}
      \miniitem{0.3 \linewidth}{$(x^*)^* = x$}
    \end{itemize}
  \end{minipage}
  \begin{itemize}
    \item $\fa{y \in \partial B_r(a)} \norm{x^* - y}^2 = r^2 \norm{x - a}^{-2} \norm{y - x}^2$.
  \end{itemize}
\end{bem}

\begin{nota}
  %Sei $g : B_r(a) \times B_r(a) \setminus \Set{ (x, x) }{ x \in B_r(a) }$ def. durch
  Für $B_r(a) \subset \R$ sei $g : B_r(a) \times B_r(a) \to \R$ definiert durch
  \[ g(x, y) \coloneqq \begin{cases}
      - \Phi\left( \tfrac{\abs{x-a}}{r} (y - x^*) \right) & \text{für $x \in B_r(a) \setminus \{ a \}$}\\
      - \Phi(r e_1) & \text{für $x = a$.}
    \end{cases}
  \]
\end{nota}

\begin{prop}
  Für die Funktion $g$ gilt:
  \begin{itemize}
    \item $g(x, y) - \Phi(x - y) = 0$ für alle $y \in \partial B_r(a)$ und $x \in B_r(a)$.
    \item $y \mapsto g(x, y)$ ist glatt und harmonisch in $B_r(a)$ für alle $x \in B_r(a)$.
  \end{itemize}
\end{prop}

\begin{kor}
  Die Greensche Funktion für $B_r(a) \subset \R^n$ lautet
  \begin{align*}
    G_{B_r(a)}(x, y) &\coloneqq \Phi(x - y) + g(x, y)\\
    &= \begin{cases}
      \Phi(x - y) - \Phi\left(\tfrac{\abs{x-a}}{r} (y - x^*)\right) & \text{für $x \in B_r(a) \setminus \{ a \}$}\\
      \Phi(a - y) - \Phi(r e_1) & \text{für $x = a$.}
    \end{cases}
  \end{align*}
\end{kor}

\begin{defn}
  Der \emph{Poisson-Kern für die Kugel} $B_r(a) \subset \R^n$ ist
  \[ K_{B_r(a)}(x, y) \coloneqq \tfrac{1}{n \omega_n} \cdot \tfrac{r^2 - \abs{x - a}^2}{r \abs{x - y}^n}. \]
\end{defn}

\begin{satz}[Poisson-Integralformel für Kugeln]
  Sei $B_r(a) \subset \R^n$ und $g : \partial B_r(a) \to \R$ stetig.
  \begin{itemize}
    \item Für $u {\in} \mathcal{C}^2(B_r(a)) \cap \mathcal{C}(\overline{B_r(a)})$ harmonisch mit $u {=} g$ auf $\partial B_r(a)$ gilt
    \[
      u(x) = \Int{\mathclap{\partial B_r(a)}}{}{K_{B_r(a)}(x,y) g(y)}{\HM^{n-1}(y)}
      \quad \text{für alle $x \in B_r(a)$.} \tag{2.8}
    \]
    \item Umgekehrt definiert (2.8) die eindeutige harmonische Funktion $u \in \mathcal{C}^2(B_r(a)) \cap \mathcal{C}^1(\overline{B_r(a)})$ mit $u = g$ auf $\partial B_r(a)$.
  \end{itemize}
\end{satz}

% Ausgelassen: Bemerkung
% * Punkt 1 ist für $a=x$ die MW-Eigenschaft undermöglicht Herleitung einer speziellen Harnack-Ungleichung
% * Punkt 2 ist Satz mit expliziter Lösungsformel und wird als Satz von Schwarz bezeichnet.

% Vorlesung vom 8.5.2014

\begin{nota}
  $\R^n_+ \coloneqq \Set{ (x_1, ..., x_n) \in \R^n }{ x_n > 0 }$ heißt \emph{Halbraum}.
\end{nota}

\begin{defn}
  Für $x = (x_1, ..., x_n) \in \R^n$ heißt $\overline{x} \coloneqq (x_1, ..., x_{n-1}, - x_n)$ \emph{Spiegelpunkt} von $x$ bzgl. $\partial \R^n_+$.
\end{defn}

\begin{satz}
  Die Greensche Funktion für $\R^n_+$ lautet
  \[
    G_{\R^n_+}(x, y) \coloneqq \Phi(x - y) - \Phi(x^* - y) \quad
    \text{für $x, y \in \R^n_+$ mit $x \not= y$.}
  \]
\end{satz}

\begin{defn}
  Der \emph{Poisson-Kern für den Halbraum} $\R^n_+$ ist definiert als
  \[
    K_{\R^n_+}(x, y) \coloneqq \tfrac{1}{n \omega_n} \cdot \tfrac{2 x_n}{\abs{x-y}^n} \quad
    \text{für $x \in \R^n_+$ und $y \in \partial \R^n_+$.}
  \]
\end{defn}

\begin{satz}[Poisson-Integralformel für den Halbraum]
  Sei $g \in \mathcal{C}(\R^{n-1}) \cap L^\infty(\R^{n-1})$. Dann definiert
  \[
    u(x) \coloneqq \Int{\partial \R^n_+}{}{K_{\R^n_+}(x, y) g(y)}{\HM^{n-1}(y)} \quad
    \text{für $x \in \R^n_+$}
  \]
  eine beschränkte, harmonische Funktion $u \in \mathcal{C}(\R^n_+) \cap \mathcal{C}^1(\overline{\R^n_+})$ mit $u = g$ auf $\partial \R^n_+$.
\end{satz}

\begin{nota}
  Für $\Omega \subset \R^n$ schreibe
  \begin{itemize}
    \item $\Omega^+ \coloneqq \Set{(x_1, ..., x_n) \in \Omega}{x_n > 0} = \Omega \cap \R^n_+$
    \item $\Omega^0 \coloneqq \Set{(x_1, ..., x_n) \in \Omega}{x_n = 0} = \Omega \cap \partial \R^n_+$
    \item $\Omega^- \coloneqq \Set{(x_1, ..., x_n) \in \Omega}{x_n < 0} = \Omega \setminus \overline{\R^n_+}$
  \end{itemize}
\end{nota}

\begin{satz}
  Sei $\Omega \subset \R^n$ offen und symmetrisch bzgl. $\partial \R^n_+$, \dh{} für alle $x \in \R^n$ gilt $x \in \Omega \iff \overline{x} \in \Omega$.
  \begin{itemize}
    \item Ungerades Spiegelungsprinzip: Ist $u \in \mathcal{C}^2(\Omega^+) \cap \mathcal{C}(\Omega^+ \cup \Omega^0)$ harmonisch auf $\Omega^+$ mit $u = 0$ auf $\Omega^0$, so ist die ungerade Fortsetzung
    \[
      \overline{u}(x) \coloneqq \begin{cases}
        u(x) & \text{für $x \in \Omega^+ \cup \Omega^0$,}\\
        -u(\overline{x}) = -u(x_1, ..., -x_n) & \text{für $x \in \Omega^-$}
      \end{cases}
    \]
    harmonisch auf $\Omega$.
    \item Gerades Spiegelungsprinzip: Ist $u \in \mathcal{C}^2(\Omega^+) \cap \mathcal{C}^1(\Omega^+ \cup \Omega^0)$ mit $D_n u = 0$ auf $\Omega^0$, so ist die gerade Fortsetzung
    \[
      \overline{u}(x) \coloneqq \begin{cases}
        u(x) & \text{für $x \in \Omega^+ \cup \Omega^0$,}\\
        u(\overline{x}) = u(x_1, ..., x_{n-1}, -x_n) & \text{für $x \in \Omega^-$}
      \end{cases}
    \]
    harmonisch auf $\Omega$.
  \end{itemize}
\end{satz}

\begin{bem}
  Aus dem Spiegelungsprinzip und dem Satz von Liouville folgt die Eindeutigkeit beschränkter harmonischer Funktionen auf dem Halbraum mit vorgegebenen Randwerten.
\end{bem}

% Kapitel 2.4. Die Perron-Methode

\begin{prob}[Dirichlet-RWP]
  Für $\Omega \subset \R^n$ und $g \in \mathcal{C}(\partial \Omega)$ ist $u \in \mathcal{C}^2(\Omega) \cap \mathcal{C}(\overline{\Omega})$ gesucht mit
  \[
    (2.9) \left\{ \begin{array}{rlll}
      \Delta u &=& 0 &\text{ in $\Omega$,}\\
      u &=& g &\text{ in $\partial \Omega$.}
    \end{array} \right.
  \]
\end{prob}

\begin{defn}
  Sei $\Omega \subset \R^n$ offen. Eine Funktion $u \in \mathcal{C}(\Omega)$ heißt \emph{$\mathcal{C}^0$-sub- harmonisch}, falls die MW-Ungleichung auf Sphären erfüllt ist, \dh{}
  \[
    u(x_0) \leq \enspace \MVInt{\mathclap{\partial B_r(x_0)}}{}{\enspace u}{\HM^{n-1}} \quad
    \text{für alle $x_0 \in \Omega$ und $r \in \ointerval{0}{\dist(x_0, \partial \Omega)}$.}
  \]
  Die Funktion heißt \emph{$\mathcal{C}^0$-superharmonisch}, falls $-u$ $\mathcal{C}^0$-subharmonisch ist und \emph{$\mathcal{C}^0$-harmonisch}, falls $u$ sowohl subharmonisch als auch harmonisch ist.
\end{defn}

\begin{nota}
  \begin{minipage}[t]{0.8 \linewidth}
    \vspace{-15pt}
    \begin{align*}
      H^-(\Omega) &\coloneqq \Set{u \in \mathcal{C}(\Omega)}{ \text{$u$ ist subharmonisch auf $\Omega$} }\\
      H^+(\Omega) &\coloneqq \Set{u \in \mathcal{C}(\Omega)}{ \text{$u$ ist superharmonisch auf $\Omega$} }\\
      H^0(\Omega) &\coloneqq \Set{u \in \mathcal{C}(\Omega)}{ \text{$u$ ist harmonisch auf $\Omega$} }
    \end{align*}
  \end{minipage}
\end{nota}

\begin{bem}
  $\mathcal{C}^0$-harmonische Funktionen sind glatt und harmonisch.
\end{bem}

\begin{lem}
  Sei $\Omega \subset \R^n$ offen. Für $u \in \mathcal{C}(\Omega)$ sind äquivalent:
  \begin{itemize}
    \item $u$ ist $\mathcal{C}^0$-subharmonisch auf $\Omega$.
    \item $u$ erfüllt die MW-Ungleichung auf Kugeln, \dh{} für alle $x_0 \in \Omega$ gilt
    \[
      u(x_0) \leq \enspace\MVInt{\mathclap{B_r(x_0)}}{}{\enspace u(x)}{x} \quad
      \text{für alle $r \in \ointerval{0}{\dist(x_0, \partial \Omega)}$.}
    \]
    \item $u$ erfüllt die MW-Ungleichung auf kleinen Kugeln, \dh{} für alle $x_0 \in \Omega$ gibt es ein $R(x_0) \in \ointerval{0}{\dist(x_0, \partial \Omega)}$ mit
    \[
      u(x_0) \leq \enspace\MVInt{\mathclap{B_r(x_0)}}{}{\enspace u(x)}{x} \quad
      \text{für alle $r \in \ointerval{0}{R(x_0)}$.}
    \]
    \item Für alle Kugeln $B_r(x_0) \Subset \Omega$ gilt das Vergleichsprinzip mit harmonischen Funktionen, \dh{} für jede harmonische Funktion $h \in \mathcal{C}(\overline{B_r(x_0)})$ gilt: $u \leq h \text{ auf $\partial B_r(x_0)$} \implies u \leq h \text{ in $B_r(x_0)$.}$
  \end{itemize}
\end{lem}

\begin{defn}
  Sei $\Omega \subset \R^n$ offen und $B_r(x_0) \Subset \Omega$. Der \emph{Perron-Projektor} $P_{x_0,r} : \mathcal{C}(\Omega) \to \mathcal{C}(\Omega)$ ist definiert durch
  \[
    (P_{x_0,r} u)(x) \coloneqq \begin{cases}
      \enspace u(x), & x \in \Omega \setminus B_r(x_0)\\
      \qquad \Int{\mathclap{\partial B_r(x_0)}}{}{\enspace K_{B_r(x_0)}(x,y) u(y)}{\HM^{n-1}(y)}, & x \in B_r(x_0).
    \end{cases}
  \]
\end{defn}

\begin{bem}
  Die Funktion $P_{x_0,r}(u)$ wird \emph{harmonische Fortsetzung} von $u$ auf $B_r(x_0)$ genannt.
\end{bem}

\begin{lem}
  Sei $\Omega \subset \R^n$ offen. Dann gilt:
  \begin{itemize}
    \item Sind $v \in H^-(\Omega)$ und $w \in H^+(\Omega)$, so gilt $v - w \in H^-(\Omega)$.
    \item Sind $v_1, v_2 \in H^-(\Omega)$, $w_1, w_2 \in H^+(\Omega)$ und $\lambda_1, \lambda_2 \geq 0$, so ist
    \begin{align*}
      \{ \max(v_1, v_2), \lambda_1 v_1 + \lambda_2 v_2 \} &\subset H^-(\Omega),\\
      \{ \min(w_1, w_2), \lambda_1 w_1 + \lambda_2 w_2 \} &\subset H^+(\Omega).
    \end{align*}
    \item Sind $v \in H^-(\Omega), w \in H^+(\Omega)$ und $B_r(x_0) \Subset \Omega$, so gelten
    \begin{align*}
      P_{x_0,r} v \geq v \text{ in $\Omega$} \quad &\text{und} \quad P_{x_0,r} v \in H^-(\Omega),\\
      P_{x_0,r} w \leq w \text{ in $\Omega$} \quad &\text{und} \quad P_{x_0,r} w \in H^+(\Omega).
    \end{align*}
  \end{itemize}
\end{lem}

\begin{defn}
  Sei $\Omega \subset \R^n$ offen, beschränkt und $g \in \mathcal{C}(\partial \Omega)$. Dann heißt $u \in \mathcal{C}(\overline{\Omega})$ \emph{Sublösung} des Dirichletproblems (2.9), falls $u \leq g$ auf $\partial \Omega$ gilt und \emph{Superlösung} von (2.9), falls $u \geq g$ auf $\partial \Omega$ gilt.
\end{defn}

\begin{nota}
  \begin{minipage}[t]{0.7 \linewidth}
    \vspace{-15pt}
    \begin{align*}
      H_g^-(\Omega) &\coloneqq \Set{ u \in \mathcal{C}(\overline{\Omega}) \cap H^-(\Omega) }{ u \leq g \text{ auf $\partial \Omega$} },\\
      H_g^+(\Omega) &\coloneqq \Set{ u \in \mathcal{C}(\overline{\Omega}) \cap H^+(\Omega) }{ u \geq g \text{ auf $\partial \Omega$} },\\
      u^-(x) &\coloneqq \sup\,\Set{ v(x) }{ v \in H_g^-(\Omega) },\\
      u^+(x) &\coloneqq \sup\,\Set{ v(x) }{ v \in H_g^+(\Omega) }.
    \end{align*}
  \end{minipage}
\end{nota}

\begin{methode}[Perron]
  Zeige zunächst, das $u^-$ und $u^+$ harmonisch sind. Zeige dann unter gewissen Regularitätsvoraussetzungen an $\Omega$, dass $u^- = u^+$ gilt und die vorgegebenen Randwerte annimmt.
\end{methode}

\begin{lem}
  Sei $\Omega \subset \R^n$ offen, beschränkt, $g \in \mathcal{C}(\partial \Omega)$. Dann sind $u^-$ und $u^+$ wohldefiniert und es gilt
  \[ \inf_{\partial \Omega} g \leq u^- \leq u^+ \leq \sup_{\partial \Omega} g. \]
\end{lem}

\begin{bem}
  Falls eine Lösung $u$ des Dirichlet-Problems existiert, so gilt $u = u^- = u^+$.
\end{bem}

% Vorlesung vom 13.5.2014

\begin{satz}
  Sei $\Omega \subset \R^n$ offen, beschränkt und $g \in \mathcal{C}(\partial \Omega)$. Dann sind $u^-$ und $u^+$ harmonisch in $\Omega$.
\end{satz}

\begin{defn}
  Sei $\Omega \subset \R^n$ offen, beschränkt, $x_0 \in \partial \Omega$. Eine Funktion $b : \overline{\Omega} \to \R_0^+$ heißt (obere) \emph{Barriere} zu $\Omega$ in $x_0$, falls
  \begin{itemize}
    \item $b \in \mathcal{C}(\overline{\Omega}) \cap H^+(\Omega)$
    \item $b(x_0) = 0$ und $b(x) > 0$ für alle $x \in \overline{\Omega} \setminus \{ x_0 \}$
  \end{itemize}
  Falls eine solche Barriere existiert, so heißt $x_0$ \emph{regulärer Randpunkt}.
  %Der Rand $\partial \Omega$ heißt \emph{regulär}, falls jeder Randpunkt $x_0 \in \partial \Omega$ regulär ist.
\end{defn}

\begin{bem}
  Eine Funktion $b : \overline{\Omega} \to \R_0^-$ heißt untere Barriere, falls $(-b) : \overline{\Omega} \to \R_0^+$ eine obere Barriere ist.
\end{bem}

\begin{defn}
  Eine \emph{lokale Barriere} zu $\Omega$ in $x_0 \in \partial \Omega$ ist eine Barriere $\tilde{b} : B_r(x_0) \cap \Omega \to \R^+_0$ zu $B_r(x_0) \cap \Omega$ in $x_0$.
\end{defn}

\begin{bem}
  Aus jeder lokalen Barriere lässt sich eine globale Barriere konstruieren.
\end{bem}

% 2.27
\begin{lem}
  Sei $\Omega \subset \R^n$ offen, beschränkt. Falls das Dirichlet-Problem (2.9) für jede Funktion $g \in \mathcal{C}(\partial \Omega)$ eine Lösung $u \in \mathcal{C}^2(\Omega) \cap \mathcal{C}(\overline{\Omega})$ besitzt, so sind alle Randpunkte $x_0 \in \partial \Omega$ regulär.
\end{lem}

% 2.28
\begin{lem}
  Sei $\Omega \subset \R^n$ offen, beschränkt, $x_0 \in \partial \Omega$. Ist $x_0$ regulär, dann gilt für jede stetige Funktion $g \in \mathcal{C}(\partial \Omega)$:
  \[ \lim_{\Omega \ni x \to x_0} u^-(x) = g(x_0) = \lim_{\Omega \ni x \to x_0} u^+(x). \]
\end{lem}

% 2.29
\begin{satz}[Perron]
  Sei $\Omega \subset \R^n$ offen, beschränkt. Dann sind äquivalent:
  \begin{itemize}
    \item Der Rand $\partial \Omega$ ist regulär.
    \item Das Dirichlet-Problem (2.9) besitzt eine eindeutige Lösung $u \in \mathcal{C}^2(\Omega) \cap \mathcal{C}(\overline{\Omega})$ für alle $g \in \mathcal{C}(\partial \Omega)$.
  \end{itemize}
\end{satz}

\begin{defn}
  Eine offene Menge $\Omega \subset \R^n$ erfüllt die \emph{äußere Kugel- bedingung} in $x_0 \in \partial \Omega$ an $\Omega$, falls ein Ball $B_r(a) \subset \R^n \setminus \Omega$ mit $\overline{\Omega} \cap \overline{B_r(a)} = \{ x_0 \}$ existiert.
\end{defn}

\begin{lem}
  Sei $\Omega \subset \R^n$ offen. Erfüllt $\Omega$ die äußere Kugelbedingung in $x_0 \in \partial \Omega$ an $\Omega$, dann ist $x_0$ ein regulärer Randpunkt.
\end{lem}

\begin{bem}
  Beschränkte Gebiete mit $\mathcal{C}^2$-Rand und strikt konvexe Gebiete erfüllen die äußere Kugelbedingung.
\end{bem}

% Ausgelassen: Bemerkung Kegelbedingung
% Ausgelassen: Bemerkung zur Balayage-Methode
% Ausgelassen: Überblick, weiteres Vorgehen

% Vorlesung vom 15.5.2014

% Kapitel 2.5. Das Newtonpotential und Lösung der Poisson-Gleichung

\begin{problem}
  Seien $\Omega \subset \R^n$, $f : \Omega \to \R$, $g : \partial \Omega \to \R$ gegeben. Gesucht ist eine Lösung $u \in \mathcal{C}^2(\Omega) \cap \mathcal{C}(\overline{\Omega})$ der Poisson-Gleichung
  \[
    (2.10) \left\{ \begin{array}{rlll}
      - \Delta u &=& f & \text{in $\Omega$}\\
      u &=& g & \text{auf $\partial \Omega$.}
    \end{array} \right.
  \]
\end{problem}

% 2.31.
\begin{prop}
  Sei $\Omega \subset \R^n$ beschränkt, offen und $\partial \Omega$ regulär. Sind $f \in \mathcal{C}^2_0(\R^n)$, $g \in \mathcal{C}(\partial \Omega)$, so besitzt (2.10) eine eindeutige Lösung $u \in \mathcal{C}^2(\Omega) \cap \mathcal{C}(\overline{\Omega})$.
\end{prop}

\begin{bem}
  Die Behauptung folgt leicht aus den vorherigen Sätzen, aber die Voraussetzung $f \in \mathcal{C}^2_0(\Omega)$ ist zu stark.
\end{bem}

\begin{defn}
  Sei $\Omega \subset \R^n$ offen und beschränkt. Das \emph{Newton-Potential} $N_f : \Omega \to \R$ einer Funktion $f \in L^\infty(\Omega)$ ist die Faltung
  \[ N_f(x) \coloneqq \Int{\Omega}{}{\Phi(x-y) f(y)}{y}. \]
\end{defn}

\begin{bem}
  Die Fktn $f$ wird außerhalb von $\Omega$ durch $0$ fortgesetzt.
\end{bem}

% 2.32.
\begin{lem}
  Sei $\Omega \subset \R^n$ offen, beschränkt und $f \in L^\infty(\Omega)$. Dann gilt $N_f \in \mathcal{C}^1(\Omega)$ mit $D N_f(x) = \Int{\Omega}{}{D_x \Phi(x-y) f(y)}{y}$ für alle $x \in \Omega$.
\end{lem}

% 2.33.
\begin{kor}
  Sei $\Omega \subset \R^n$ offen, beschränkt und $f \in L^\infty(\Omega)$. Dann gilt $N_f \in \mathcal{C}^{1,\alpha}(\Omega)$ für alle $\alpha \in \ointerval{0}{1}$.
\end{kor}

% Vorlesung vom 20.5.2014

% 2.34.
\begin{satz}[Hölder] % Dissertation, 1882
  Sei $\Omega \subset \R^n$ offen, beschr. und $f \in \mathcal{C}^{0,\alpha}(\Omega) \cap L^\infty(\Omega)$ für ein bel. $\alpha \in \ointerval{0}{1}$. Dann gilt $N_f \in \mathcal{C}^2(\Omega)$ mit $- \Delta N_f = f$ in $\Omega$ und für jede Kugel $B$ mit $\Omega \Subset B$ gilt die Darstellung
  \begin{align*}
    D_i D_j N_f(x) = &\Int{B}{}{D_{x_i} D_{x_j} \Phi(x-y) (f(y) - f(x))}{y}\\
    & - f(x) \cdot \Int{\partial B}{}{D_{x_j} \Phi(x-y) \nu_i(y)}{\HM^{n-1}(y)}.
  \end{align*}
\end{satz}

% 2.35.
\begin{satz}
  Sei $\Omega \subset \R^n$ offen, beschränkt und $\partial \Omega$ regulär. Sind $f \in \mathcal{C}^{0,\alpha}(\Omega) \cap L^\infty(\Omega)$ für ein $\alpha \in \ointerval{0}{1}$ und $g \in \mathcal{C}(\partial \Omega)$, so besitzt (2.10) eine eindeutige Lösung $u \in \mathcal{C}^2(\Omega) \cap \mathcal{C}(\overline{\Omega})$.
\end{satz}

\begin{bem}
  Für das Newtonpotential überträgt sich die Regularität von $f$ wie folgt: Ist $\alpha \in \ointerval{0}{1}$ und $f \in \mathcal{C}^{k,\alpha}(\overline{B_{2R}})$, so ist $N_f \in \mathcal{C}^{k+2,\alpha}(\overline{B_R})$ für alle $R > 0$. Im Allgemeinen gilt:
  \[
    f {\in} L^\infty \not\Rightarrow N_f \in \mathcal{C}^2, \quad
    f {\in} \mathcal{C}^k \not\Rightarrow N_f \in \mathcal{C}^{k+2}, \quad
    f {\in} \mathcal{C}^{k,1} \not\Rightarrow N_f \in \mathcal{C}^{k+2,1}
  \]
\end{bem}

% Vorlesung vom 22.5.2014

% Ausgelassen: Optimalität der Voraussetzung, Hebung von Punktsingularitäten

\begin{problem}
  Gesucht sind Lösungen des Eigenwertproblems mit Dirichlet-Randwert $0$ für den Laplaceoperator, also Eigenwerte $\lambda \in \R$ und Eigenfunktionen $u_\lambda \in \mathcal{C}^2(\Omega)$ mit
  \[
    \left\{ \begin{array}{rlll}
      - \Delta u_\lambda &=& \lambda u_\lambda & \text{in $\Omega$}\\
      u_\lambda &=& 0 & \text{auf $\partial \Omega$.}
    \end{array} \right.
  \]
\end{problem}

\begin{bemn}
  \begin{itemize}
    \item Es sind nur nichtnegative EWe möglich.
    \item Es gibt höchstens abzählbar viele EWe.
    \item Die zugehörigen Eigenräume sind endlich-dimensional.
    \item Eigenfunktionen zu unterschiedlichen EWen sind orthogonal bzgl.
    \[ \langle u, v \rangle_{L^2} \coloneqq \Int{\Omega}{}{u \cdot v}{x}. \]
    \item Eigenfunktionen sind glatt.
  \end{itemize}
\end{bemn}

\section{Wärmeleitungsgleichung}

\begin{nota}
  Sei $I \subset \R$ ein Intervall, $\Omega \subset \R^n$. Wir suchen $u : I \times \Omega \to \R, (t, x) \mapsto u(t, x)$ und schreiben $u_t \coloneqq \tfrac{\partial u}{\partial t}$ für die Zeitableitung, $D u(t, x) \coloneqq D_x u(t, x)$ für die Ortsableitung und $\Delta u(t,x) \coloneqq \Delta_x u(t,x)$.
\end{nota}

\begin{defn}
  Sei $\Omega \subset \R^n$ und $T > 0$.
  \begin{itemize}
    \item Der \emph{parabolische Zylinder} ist $\Omega_T \coloneqq \ocinterval{0}{T} \times \Omega \subset \R^{n+1}$.
    \item Der \emph{parabolische Rand} von $\Omega_T$ ist
    \[ \partial_p \Omega_T \coloneqq (\{ 0 \} \times \Omega) \cup (\cinterval{0}{T} \times \partial \Omega) \subset \partial \Omega_T. \]
  \end{itemize}
\end{defn}

\begin{nota}
  Wir benötigen untersch. Regularitäten in Ort und Zeit:
  \[ \mathcal{C}_1^2(\Omega_T) \coloneqq \Set{f \in \mathcal{C}^1(\Omega_T)}{f \text{ ist zweimal stetig nach $x$ ableitbar}} \]
\end{nota}

\begin{prob}
  Wärmeleitungsgleichung (WLG): $u_t - \Delta u = 0$
\end{prob}

% Ausgelassen: Physikalische Motivation

\begin{defn}
  Sei $\Omega \subset \R^n$ offen, $T > 0$ und $u \in \mathcal{C}^2_1(\Omega_T)$. Dann heißt $u$
  \begin{align*}
    \left. \begin{array}{l}
      \text{subkalorisch}\\
      \text{\emph{kalorisch}}\\
      \text{superkalorisch}\\
    \end{array} \right\}
    \text{ falls }
    \left\{ \begin{array}{l}
      u_t - \Delta u \leq 0\\
      u_t - \Delta u = 0\\
      u_t - \Delta u \geq 0
    \end{array} \right\}
    \text{ in $\Omega_T$ gilt.}
  \end{align*}
\end{defn}

\begin{bem}[Invarianzen der Lösungseigenschaft]
  Wenn $u \in \mathcal{C}^2_t(\Omega_T)$ kalorisch, dann auch (auf geeigneten Zylindern):
  \begin{alignat*}{3}
    u_{0,x_0}(t,x) &\coloneqq u(t, x-x_0) &&\text{ für } x_0 \in \R^n\\
    u_{\Tau, 0}(t,x) &\coloneqq u(t-\Tau, x) &&\text{ für } \Tau \in \R\\
    u_R(t,x) &\coloneqq u(t, Rx) &&\text{ für } R \in SO(n)\\
    u_\lambda(t,x) &\coloneqq \lambda^{n} u(\lambda^2 t, \lambda x) &&\text{ für } \lambda > 0
  \end{alignat*}
  Es gilt $\Int{\R^n}{}{u_\lambda(t, y)}{y} = \Int{\R^n}{}{u(\lambda^2 t, y)}{y}$.
\end{bem}

\begin{bspe}
  \begin{itemize}
    \item Sei $v$ harmonisch auf $\Omega \subset \R^n$. Dann ist $(t, x) \mapsto v(x)$ kalorisch auf $\Omega_T$.
    \item Für $n=1$ und $c_1, c_2, a \in \R$ sind kalorisch:
    \begin{align*}
      (t, x) &\mapsto \exp(a^2 t) \cdot (c_1 \sinh(a x) + c_2 \cosh(a x))\\
      (t, x) &\mapsto \exp(-a^2 t) \cdot (c_1 \sin(a x) + c_2 \cos(a x))\\
      (t, x) &\mapsto c_1 x + c_2
    \end{align*}
  \end{itemize}
\end{bspe}

% Vorlesung vom 27.5.2014

% Ausgelassen: Herleitung der Fundamentallösung (invariant und Rotationen und Dilationen)

\begin{defn}
  Die \emph{Fundamentallösung der WLG} ist die Funktion
  \[ \Psi : \ointerval{0}{\infty} \times \R^n \to \R, \quad (t, x) \mapsto \frac{1}{(4 \pi t)^{n/2}} \exp\left(- \frac{\abs{x}^2}{4t}\right). \]
\end{defn}

\begin{bemn}
  \begin{itemize}
    \item $\Psi_t(t, x) - \Delta_x \Psi(t, x) = 0$ für alle $t > 0$ und $x \in \R^n$.
    \miniitem{0.40 \linewidth}{$\Psi(t, 0) \xrightarrow{t \to 0} \infty$,}
    \miniitem{0.48 \linewidth}{$\Psi(t, x) \xrightarrow{t \to 0} 0$ bei $x \not= 0$ fest.}
    \item Für alle $t \in \R$ gilt $\Int{\R^n}{}{\Psi(t, x)}{x} = 1$.
    \item Raum- und Zeitableitungen lassen sich explizit berechnen.
    \item $\norm{\Psi(t, \blank)}_{L^\infty(\R^n)} \xrightarrow{t \to \infty} 0,$
    \item $\norm{D \Psi(t, \blank)}_{L^1(\R^n)} + \norm{\Psi_t(t, \blank)}_{L^1(\R^n)} \xrightarrow{t \to \infty} 0,$
    % TODO: Ableitung (irgendwelche Polynomfunktionen) verstehen!
    % \[ \partial_t^k D_x^{\alpha} \Psi(t, x) = P_{\alpha,t}(x, t^{-1/2}) e^{-\abs{x}^2 / 4t} \]
    % Ausgelassen: Eigenschaften legen Fundamentallösung eindeutig fest
    \item $\Psi(t, \blank) * \Psi(s, \blank) = \Psi(t+s, \blank)$
  \end{itemize}
\end{bemn}

\begin{bem}
  Für die Fundamentallösung der Laplace-Gleichung gilt:
  \[
    \fa{x_0 \in \R^n} B_1(x_0) = \Set{ x \in \R^n }{ \Phi(x_0 - x) > \Phi(r e_1) }
  \] % "`Superniveaumenge"'
\end{bem}

\begin{defn}
  Die \emph{Wärmeleitungskugel} um $(t_0, x_0) \in \R \times \R^n$ mit $r > 0$ ist
  \begin{align*}
    W_r(t_0, x_0) &\coloneqq \Set{(t, x) \in \R^{n+1}}{t < t_0 \text{ und } \Psi(t_0-t,x_0-x) > r^{-n}}\\
    &\subseteq \ointerval{-\infty}{t_0} \times \R^n
  \end{align*}
\end{defn}

\begin{nota}
  $W_r \coloneqq W_r(0, 0)$
\end{nota}

\begin{nota}
  Für $r > 0$ setze
  \[
    b_r : \R^{-} \times \R^n \to \R, \quad
    (t, x) \mapsto \tfrac{\abs{x}^2}{4t} + n \log r - \tfrac{n}{2} \log(-4 \pi t).
  \]
\end{nota}

\begin{bemn}
  \begin{itemize}
    \item Monotonie: Für $r \leq \tilde{r}$ gilt $W_r(x_0) \subseteq W_{\tilde{r}}(x_0)$
    \item Translationsinvarianz: $W_r(t_0, x_0) = (t_0, x_0) + W_r(0, 0)$
    \item Parabolische Reskalierung: $(t, x) \in W_r \iff (r^{-2}t, r^{-1}x) \in W_1$
    \item Explizite Darstellung: Für $r > 0$ gilt
    \begin{align*}
      W_r(0, 0) &= \Set{(t, x) \in \R^{-} \times \R^n}{b_r(t, x) > 0}\\
      \partial W_r(0, 0) &= \{ (0, 0) \} \cup \Set{(t, x) \in \R^{-} \times \R^n}{b_r(t,x) = 0}\\
      W_r(0, 0) &= \{ (t, x) \in \R^{-} \times \R^n \mid - \tfrac{r^2}{4 \pi} < t < 0 \text{ und }\\
      & \qquad \abs{x}^2 < 4t (-n \log r + \tfrac{n}{2} \log (-4 \pi t)) \}
    \end{align*}
    \item Es gilt $\Int{W_r}{}{\tfrac{\abs{x}^2}{t^2}}{(t,x)} = 4 r^n$
  \end{itemize}
\end{bemn}

% 3.1.
\begin{lem}
  Sei $R > 0$ und $u \in \mathcal{C}^2_1(W_R)$. Für
  \[
    \phi : \ointerval{0}{R} \to \R, \quad
    r \mapsto \frac{1}{4 r^n} \enspace \Int{W_r}{}{u(t,x) \frac{\abs{x}^2}{t^2}}{(t,x)}
    \qquad \text{gilt dann}
  \]
  \begin{itemize}
    \item $\lim_{r \to 0} \phi(r) = u(0,0)$
    \item $\phi'(r) = \frac{n}{r^{n+1}} \enspace \Int{W_r}{}{\left( -u_t(t,x) + \Delta u(t,x) \right) \cdot b_r(t, x)}{(t,x)}$
  \end{itemize}
\end{lem}

% 3.2.
\begin{satz}[MWE]
  Sei $\Omega \subset \R^n$, $T > 0$, $u \in \mathcal{C}^2_1(\Omega_T)$ und $W_r(t_0, x_0) \Subset \Omega_T$. Dann gilt:
  \begin{align*}
    u_t - \Delta u &= 0 \text{ in } \Omega_T\\
    \implies u(t_0, x_0) &= \frac{1}{4 r^n} \qquad \Int{\mathclap{W_r(t_0, x_0)}}{}{\quad u(t,x) \cdot \frac{\abs{x-x_0}^2}{(t-t_0)^2}}{(t,x)}
  \end{align*}
  In diesen Gleichungen darf man $=$ durch $\leq$, $<$, $\geq$ oder $>$ ersetzen.
\end{satz}

% 3.3.
\begin{kor}
  Für $\Omega \subset \R^n$ offen, $T {>} 0$ und $u \in \mathcal{C}_1^2(\Omega_T)$ sind äquivalent:
  \begin{itemize}
    \item $u$ ist kalorisch
    \item $u$ erfüllt die MWE auf WL-Kugeln, \dh{} f.\,a. $W_r(t_0, x_0) \Subset \Omega_T$ gilt
    \[
      u(t_0, x_0) = \frac{1}{4 r^n} \qquad \Int{\mathclap{W_r(t_0, x_0)}}{}{\quad u(t,x) \cdot \frac{\abs{x-x_0}^2}{(t-t_0)^2}}{(t,x)}
    \]
  \end{itemize}
\end{kor}

% 3.3. (versehentliche Verdoppelung)
\begin{satz}
  Sei $\Omega \subset \R^n$ beschränkt, offen, $T > 0$ und $u \in \mathcal{C}^2_1(\Omega_T) \cap \mathcal{C}(\overline{\Omega_T})$ subkalorisch. Dann gilt:
  \begin{itemize}
    \item Das \emph{schwache Maximumsprinzip}: $\max_{\overline{\Omega_T}} u = \max_{\partial_p \Omega_T} u$
    \item Das \emph{starke Maximumsprinzip}: Ist $\Omega$ zshgd und gibt es $(t_0, x_0) \in \Omega_T$ mit $u(t_0, x_0) = \max_{\overline{\Omega_T}} u$, so ist $u$ konstant auf $\Omega_{t_0}$.
  \end{itemize}
\end{satz}

\begin{bem}
  Es gelten auch entsprechende Minimumsprinzipien für superkalorische Funktionen.
\end{bem}

\begin{bem}["`unendliche Ausbreitungsgeschwindigkeit"']\mbox{}\\
  Sei $\Omega \subset \R^n$ beschränkt, offen, zusammenhängend, $T > 0$ und $u \in \mathcal{C}_1^2(\Omega_T) \cap \mathcal{C}(\overline{\Omega_T})$ eine kalorische Funktion, die auf $\cinterval{0}{T} \times \partial \Omega$ verschwindet. Dann gilt
  \[ \min_{\{ 0 \} \times \partial \Omega} u < \max_{\{ 0 \} \times \partial \Omega} u \enspace\implies\enspace \min_{\partial \{ 0 \} \times \Omega} u < u < \max_{\{ 0 \} \times \partial \Omega} u \text{ in $\Omega_T$}. \]
\end{bem}

% 3.4.
\begin{kor}[Eindeutigkeit]
  Sei $\Omega \subset \R^n$ offen, beschränkt, $T > 0$, $u, v \in \mathcal{C}^2_1(\Omega_T) \cap \mathcal{C}(\overline{\Omega_T})$ mit
  \[
    \left\{ \begin{array}{ll}
      u_t - \Delta u = v_t - \Delta v & \text{in } \Omega_T\\
      u = v & \text{auf } \partial_p \Omega_T
    \end{array} \right.
  \]
  Dann gilt $u \equiv v$ in $\Omega_T$.
\end{kor}

\begin{prop}
  Sei $\Omega \subset \R^n$ beschränkt, offen mit regulären Rand- punkten und $f \in \mathcal{C}^{0,\alpha}(\Omega) \cap L^\infty(\Omega)$ für ein $\alpha \in \ointerval{0}{1}$ und $g \in \mathcal{C}(\partial \Omega)$. Dann gilt für jede Lösung $u \in \mathcal{C}_1^2(\Omega \times \ointerval{0}{\infty}) \cap \mathcal{C}(\Omega \times \cointerval{0}{\infty})$ von
  \[
    \left\{ \begin{array}{ll}
      u_t - \Delta u = f & \text{in } \ointerval{0}{\infty} \times \Omega\\
      u = g & \text{auf } \partial_p \Omega_T
    \end{array} \right.
  \]
  mit beliebigen Anfangswerten auf $\{ 0 \} \times \Omega$:
  \[
    \lim_{t \to \infty} u(t, \blank) = v,
  \]
  wobei $v \in \mathcal{C}^2(\Omega) \cap \mathcal{C}(\overline{\Omega})$ die eindeutige Lösung der folgender Poisson-Gleichung ist:
  \[
    \left\{ \begin{array}{ll}
      \Delta u = f & \text{in } \Omega,\\
      u = g & \text{auf } \partial \Omega.
    \end{array} \right.
  \]
\end{prop}

% Kapitel 3.2. Darstellungsformeln für Lösungen im Ganzraumfall

% Skript, 3.17
\begin{satz}
  Sei $g \in L^p(\R^n)$ für ein $p \in \cinterval{1}{\infty}$. Dann ist die Funktion $u : \ointerval{0}{\infty} \times \R^n \to \R$ definiert durch
  \[ u(t, x) \coloneqq (\Psi(t, \blank) * g)(x) = \Int{\R^n}{}{\Psi(t, x-y) \cdot g(y)}{y} \]
  in $\mathcal{C}^\infty(\ointerval{0}{\infty} \times \R^n)$ und erfüllt
  \[
    \left\{ \begin{array}{rlll}
      u_t - \Delta u &=& 0 & \text{in $\ointerval{0}{\infty} \times \R^n$},\\
      u &=& g & \text{auf $\{ 0 \} \times \R^n$}.
    \end{array} \right.
  \]
\end{satz}

% Skript, 3.18
\begin{satz}
  Für die Funktion $u$ aus dem vorherigen Satz gilt:
  \begin{itemize}
    \item Ist $g \in \mathcal{C}(\R^n) \cap L^\infty(\R^n)$, so konvergiert
    \[
      u(t, x) \xrightarrow{t \to 0, x \to x_0} g(x_0)
      \quad \text{für alle $x_0 \in \R^n$.}
    \]
    Die Konvergenz ist gleichmäßig auf kompakten Mengen.
    \item Ist $g \in L^p(\R^n)$ für ein $p \in \cinterval{1}{\infty}$, so ist $\norm{u(t, \blank)}_{L^p(\R^n)} \leq \norm{g}_{L^p(\R^n)}$ für alle $t > 0$. Ist $p < \infty$, so gilt
    \[ \norm{u(t, \blank) - g}_{L^p(\R^n)} \xrightarrow{t \to 0} 0. \]
  \end{itemize}
\end{satz}

% Übungsaufgabe
\begin{prop}
  Es gilt außerdem
  \[
    \Int{\R^n}{}{u(t, x)}{x} = \Int{\R^n}{}{g(x)}{x}
    \quad \text{für alle $t > 0$.}
  \]
\end{prop}

% Skript, 3.20
\begin{satz}
  Sei $f \in \mathcal{C}_1^2(\cinterval{0}{\infty} \times \R^n)$ mit $\supp(f) \Subset \cinterval{0}{\infty} \times \R^n$. Dann ist die Funktion $u : \ointerval{0}{\infty} \times \R^n \to \R$ definiert durch
  \[ u(t, x) \coloneqq \Int{0}{t}{\Int{\R^n}{}{\Psi(t-s,x-y) \cdot f(s,y)}{y}}{s} \]
  in $\mathcal{C}_1^2(\ointerval{0}{\infty} \times \R^n) \cap \mathcal{C}(\cointerval{0}{\infty} \times \R^n)$ und erfüllt die inhomogene WLG
  \[
    \left\{ \begin{array}{rlll}
      u_t - \Delta u &=& f & \text{in $\ointerval{0}{\infty} \times \R^n$},\\
      u &=& 0 & \text{auf $\{ 0 \} \times \R^n$}.
    \end{array} \right.
  \]
\end{satz}

% Skript, 3.22
\begin{satz}[Allgemeine Lösungsformel]
  Sei $g \in \mathcal{C}(\R^n) \cap L^\infty(\R^n)$ und $f \in \mathcal{C}_1^2(\cointerval{0}{\infty} \times \R^n)$ mit $\supp(f) \Subset \cointerval{0}{\infty} \times \R^n$. Dann ist die Funktion $u : \ointerval{0}{\infty} \times \R^n \to \R$ definiert durch
  \[
    u(t, x) \coloneqq \Int{\R^n}{}{\Psi(t, x-y) \cdot g(y)}{y} + \Int{0}{t}{\Int{\R^n}{}{\Psi(t-s,x-y) \cdot f(s, y)}{y}}{s}
  \]
  in $\mathcal{C}_1^2(\ointerval{0}{\infty} \times \R^n) \cap \mathcal{C}(\cointerval{0}{\infty} \times \R^n)$ und erfüllt
  \[
    \left\{ \begin{array}{rlll}
      u_t - \Delta u &=& f & \text{in $\ointerval{0}{\infty} \times \R^n$},\\
      u &=& g & \text{auf $\{ 0 \} \times \R^n$}.
    \end{array} \right.
  \]
\end{satz}


% Vorlesung vom 5.6.2014

\section{Wellengleichung}

% Ausgelassen: Physikalische Motivation

% Kapitel 4.1. Darstellungsformeln für Lösungen im Ganzraumfall

% Kapitel 4.1.1. Die d'Alembertsche Formel für $n = 1$

\begin{nota}
  $u_{tt} \coloneqq \partial_t^2 u$
\end{nota}

\begin{prob}
  Seien $g \in \mathcal{C}^2(\R^n)$, $h \in \mathcal{C}^1(\R^n)$, $f \in \mathcal{C}^1(\R^n)$. Gesucht ist eine Lösung der homogenen \emph{Wellengleichung} (WG)
  \[
    (4.1) \left\{ \begin{array}{ll}
      u_{tt} - \Delta u = 0 & \text{in } \ointerval{0}{\infty} \times \R^n\\
      u = g, u_t = h & \text{auf } \{ 0 \} \times \R^n
    \end{array} \right.
  \]
\end{prob}

\begin{satz}
  Seien $b \in \R^n$, $\tilde{f} \in \mathcal{C}^1(\cointerval{0}{\infty} \times \R^n)$ und $\tilde{g} \in \mathcal{C}(\cointerval{0}{\infty} \times \R^n)$. Dann besitzt das AWP für die Transportgleichung
  \[
    \left\{ \begin{array}{rlll}
      \partial_t u + b \cdot Du &=& \tilde{f} & \text{in $\ointerval{0}{\infty} \times \R^n$},\\
      u &=& \tilde{g} & \text{auf $\{ 0 \} \times \R^n$}.
    \end{array} \right.
  \]
  eine eindeutige Lösung $u \in \mathcal{C}^1(\ointerval{0}{\infty} \times \R^n) \cap \mathcal{C}(\cointerval{0}{\infty} \times \R^n)$, die gegeben ist durch
  \[ u(t, x) \coloneqq \tilde{g}(x - tb) + \Int{0}{t}{\tilde{f}(s, x + (s - t) b)}{s}. \]
\end{satz}

\begin{satz}
  Seien $g \in \mathcal{C}^2(\R)$, $h \in \mathcal{C}^1(\R)$ und
  \[
    u : \cointerval{0}{\infty} \times \R \to \R \quad
    (t, x) \mapsto \tfrac{1}{2} \left( g(x+t) + g(x-t) \right) + \tfrac{1}{2} \Int{x-t}{x+t}{h(s)}{s}.
  \]
  Dann ist $u \in \mathcal{C}^2(\cointerval{0}{\infty} \times \R)$ die eindt. Lsg von (4.1) auf $\ointerval{0}{\infty} \times \R$.
\end{satz}

\begin{bem}
  \begin{itemize}
    \item Es gibt keinen Regularisierungseffekt, \dh{} im Allgemeinen ist $u$ nur in $\mathcal{C}^2$, nicht besser.
    \item "`Endliche Ausbreitungsgeschwindigkeit"': Ist $\supp(g) \cup \supp(f) \subset \cinterval{x_0 - r}{x_0 + r}$ für $x_0 \in \R$, $r > 0$, so gilt $\supp(t, \blank) \subset \cinterval{x_0 - (t + r)}{x_0 + t + r}$.
  \end{itemize}
\end{bem}

\begin{kor}
  Seien $g \in \mathcal{C}^2(\cointerval{0}{\infty})$ mit $\lim_{x \to 0} g''(x) = g(0) = 0$ und $h \in \mathcal{C}(\cointerval{0}{\infty})$ mit $h(0) = 0$. Dann ist die Funktion $u : \cointerval{0}{\infty} \times \cointerval{0}{\infty} \to \R$, definiert durch
  \[
    u(t, x) \coloneqq \begin{cases}
      \tfrac{1}{2} \left( g(x + t) + g(x - t) \right) + \tfrac{1}{2} \Int{x-t}{x+t}{h(s)}{s} & \text{für } x \geq t \geq 0,\\
      \tfrac{1}{2} \left( g(x + t) - g(t - x) \right) + \tfrac{1}{2} \Int{t-x}{x+t}{h(s)}{s} & \text{für } t \geq x \geq 0
    \end{cases}
  \]
  die $\mathcal{C}^2$-Lösung des AWP der homogenen Wellengleichung
  \[
    \left\{ \begin{array}{rlll}
      u_{tt} - \Delta u = 0 & \text{in } \ointerval{0}{\infty} \times \ointerval{0}{\infty},\\
      u = g, u_t = h & \text{auf } \{ 0 \} \times \ointerval{0}{\infty},\\
      u = 0 & \text{auf } \ointerval{0}{\infty} \times \{ 0 \}.
    \end{array} \right.
  \]
\end{kor}

% Kapitel 4.1.2. Die Methode der sphärischen Mittelwerte für $n \geq 2$

\begin{nota}
  Für $u : \cointerval{0}{\infty} \times \R^n \to \R$ und $g, h : \R^n \to \R$ definiere
  \[
    U_x(t, r) \coloneqq \quad \MVInt{\mathclap{\partial B_r(x)}}{}{\enspace u(t, y)}{\HM^{n-1}(y)},
  \]
  \[
    G_x(r) \coloneqq \quad \MVInt{\mathclap{\partial B_r(x)}}{}{\enspace g(y)}{\HM^{n-1}(y)},
    \qquad
    H_x(r) \coloneqq \quad \MVInt{\mathclap{\partial B_r(x)}}{}{\enspace h(y)}{\HM^{n-1}(y)}
  \]
\end{nota}

\begin{lem}[Euler-Lagrange-Darboux-Gleichung]\mbox{}\\
  Sei $u \in \mathcal{C}^m(\cointerval{0}{\infty} \times \R^n)$ mit $m \geq 2$ eine Lsg von (4.1). Dann gilt $U_x \in \mathcal{C}^m(\cointerval{0}{\infty} \times \cointerval{0}{\infty})$ für alle $x \in \R^n$ und $U_x$ erfüllt das AWP
  \[
    \left\{ \begin{array}{ll}
      \partial_t^2 U_x - \partial_r^2 U_x - \tfrac{n-1}{r} \partial_r U_x = 0 & \text{in } \ointerval{0}{\infty} \times \ointerval{0}{\infty},\\
      U_x = G_x, \partial_t U_x = H_x & \text{auf } \{ 0 \} \times \R^n.
    \end{array} \right.
  \]
\end{lem}

% Vorlesung vom 12.6.2014

% Unterkapitel: Spezialfall $n=3$ und die Kirchhoffsche Formel

\begin{nota}
  $\widetilde{U}_x(t, r) \coloneqq r U_x(t, r)$, $\widetilde{G}_x(r) \coloneqq r G_x(r)$, $\widetilde{H}_x(r) \coloneqq r H_x(r)$
\end{nota}

% 4.3.
\begin{lem}
  Sei $u \in \mathcal{C}^m(\cointerval{0}{\infty} \times \R^3)$ für $m \geq 3$ eine Lösung der Wellengleichung (4.1). Dann gilt
  \[
    \left\{ \begin{array}{ll}
      \partial_t^2 \widetilde{U}_x - \partial_r^2 \widetilde{U}_x = 0 & \text{in } \ointerval{0}{\infty} \times \ointerval{0}{\infty} \\
      \widetilde{U}_x(0, r) = \widetilde{G}_x(r), \partial_r \widetilde{U}_x(0, r) = \widetilde{H}_x(r) & \text{in } \{ 0 \} \times \ointerval{0}{\infty} \\
      \widetilde{U}_x(t, 0) = 0 & \text{für } t > 0
    \end{array} \right.
  \]
\end{lem}

\begin{bem}
  Mit der Darstellungsformel für $\tilde{U}_x : \cointerval{0}{\infty} \times \cointerval{0}{\infty} \to \R$ folgt die \emph{Kirchhoffsche Formel} für $u : \ointerval{0}{\infty} \times \R^3 \to \R$:
  \begin{align*}
    u(t, x) &= \lim_{r \to 0} U_x(t, r) = \lim_{r \to 0} \frac{\widetilde{U}_x(t, r)}{r} \\
    &= \partial_t \left( t \quad \MVInt{\mathclap{\partial B_t(x)}}{}{\enspace g(y)}{S(y)} \right) + t \quad \MVInt{\mathclap{\partial B_t(x)}}{}{\enspace h(y)}{S(y)} \\
    &= \quad \MVInt{\mathclap{\partial B_t(x)}}{}{\enspace \left( g(y) + D g(y) \cdot (y-x) + t h(y) \right)}{S(y)}
  \end{align*}
\end{bem}

% 4.4.
\begin{satz}
  Seien $g \in \mathcal{C}^3(\R^3)$, $h \in \mathcal{C}^2(\R^3)$ und $u : \ointerval{0}{\infty} \times \R^3 \to \R$ definiert durch die Kirchhoffsche Formel. Es gilt
  \begin{itemize}
    \item $u \in \mathcal{C}^2(\cointerval{0}{\infty} \times \R^3)$
    \item $u$ ist die eindeutige Lösung der dreidimensionalen Wellengleichung mit Anfangswerten wie in (4.1)
  \end{itemize}
\end{satz}

\begin{bemn}
  \begin{itemize}
    \item Regularitätsverlust: $u$ nur in $\mathcal{C}^2$, obwohl $g \in \mathcal{C}^3$
    \item $u(t, x)$ hängt nur von den Werten von $g$ und $h$ auf $\partial B_t(x)$ ab.
    \item Somit gilt das \emph{Huygensche Prinzip}: Störungen der Anfangsdaten in der Umgebung eines Punktes ändern für große Zeiten die Lösung in dieser Umgebung nicht.
    \item Für ungerade Dimensionen $n$ funktioniert die Strategie der Transformation auf eine Lösung der eindim. Wellengleichung mit
    \[ \widetilde{U}_x(t, r) \coloneqq (r^{-1} \partial_r)^{(n-3)/2} (r^{n-2} U_x(t, r)). \]
    % Ausgelassen: Regularitätsanforderung $\mathcal{C}^{(n+3)/2}$ von $g$, $\mathcal{C}^{(n+1)/2}$ von $h$ für $\mathcal{C}^2$-Lösung
  \end{itemize}
\end{bemn}

\begin{satz}
  Sei $n \geq 3$ ungerade, $g \in \mathcal{C}^{\tfrac{n+3}{2}}(\R^n)$, $h \in \mathcal{C}^{\tfrac{n+1}{2}}$ und $u : \ointerval{0}{\infty} \times \R^n \to \R$ definiert durch
  \begin{align*}
    u(t, x) \coloneqq & \left( \prod_{k=1}^{(n-1)/2} (2k - 1) \right)^{-1} \left[ \partial_t \left( t^{-1} \partial_t \right)^{\tfrac{n-3}{2}} \left( t^{n-2} \enspace \MVInt{\mathclap{\partial B_t(x)}}{}{\enspace g}{\HM^{n-1}} \right) \right. \\
    & + \left. \left( t^{-1} \partial_t \right)^{\tfrac{n-3}{2}} \left( t^{n-2} \enspace \MVInt{\mathclap{\partial B_t(x)}}{}{\enspace h}{\HM^{n-1}} \right) \right]
  \end{align*}
  Dann ist $u \in \mathcal{C}^2(\cointerval{0}{\infty}) \times \R^n$ und die eindeutige Lösung der homogenen WG auf $\cointerval{0}{\infty} \times \R^n$ mit $u = g$ und $\partial_t u = h$ auf $\{ 0 \} \times \R^n$.
\end{satz}

% Alternativer Zugang für ungerade Dimensionen und Zusammenhang WG/WLG
\begin{bem}(Zusammenhang von WG und WLG)
  Sei $\Psi_{(n=1)}$ die Fundamentallösung der WLG in $\ointerval{0}{\infty} \times \R^1$. Sei $u$ eine glatte, beschränkte Lsg der Wellengleichung (4.1) mit $h \equiv 0$. Sei
  \[
    v(t, x) \coloneqq \Int{-\infty}{\infty}{\Psi_{(n=1)}(t, s) \cdot \overline{u}(s, x)}{s}
    \quad \text{mit} \enspace \overline{u}(s, x) \coloneqq u(\abs{s}, x).
  \]
  Dann erfüllt $v$ die homogene WLG
  \[
    \left\{ \begin{array}{rlll}
      u_t - \Delta u &=& 0 & \text{in $\ointerval{0}{\infty} \times \R^n$},\\
      u &=& g & \text{auf $\{ 0 \} \times \R^n$}.
    \end{array} \right.
  \]
\end{bem}
% TODO: Weitere Folgerungen?

% Überschrift: Spezialfall $n=2$ und die Poissonsche Formel

\begin{bem}
  Für ungerade Dimensionen $n$ können wir Lsgen der WG herleiten. Sei nun $u : \cointerval{0}{\infty} \times \R^n \to \R$ eine Lösung von (4.1) mit $n$ gerade. Dann definiert
  \[
    \overline{u} : \cointerval{0}{\infty} \times \R^{n+1} \to \R, \quad (t, (x, \underline{\,\,\,})) \mapsto u(t, x)
  \]
  eine Lsg der homogenen WG mit Startwerten $\overline{u} = \overline{g}$, $\partial_t \overline{u} = \overline{h}$ auf $\{ 0 \} \times \R^{n+1}$ (wobei $\overline{g}$ und $\overline{h}$ analog definiert zu $\overline{u}$ sind). Mit der Darstellung von $\overline{u}$ und passenden Transformationen ergibt sich eine Formel für $u$. Speziell für $n=2$:
\end{bem}

% 4.5.
\begin{satz}
  Seien $g \in \mathcal{C}^3(\R^2)$, $h \in \mathcal{C}^2(\R^2)$ und $u : \ointerval{0}{\infty} \times \R^2 \to \R$ definiert durch
  \[
    u(t, x) \coloneqq \tfrac{1}{2} \partial_t \left( t^2 \enspace \MVInt{\mathclap{B_t(x)}}{}{\enspace \tfrac{g(y)}{\sqrt{t^2 - \abs{y-x}^2}}}{y} \right) + \tfrac{1}{2} t^2 \enspace \MVInt{\mathclap{B_t(x)}}{}{\enspace \tfrac{g(y)}{\sqrt{t^2 - \abs{y-x}^2}}}{y}
  \]
  Dann ist $u \in \mathcal{C}^2(\cointerval{0}{\infty} \times \R^2)$ die eindt. Lsg von (4.1) auf $\ointerval{0}{\infty} \times \R^2$.
\end{satz}

% Ausgelassen: Allgemeine Lösung für gerade Dimensionen

\begin{bem}
  Bei geraden Dimensionen hängt $u(t, x)$ von allen Anfangsdaten in $B_t(x)$ ab, \dh{} es gilt kein Huygensches Prinzip.
\end{bem}

\begin{satz}
  Sei $n$ gerade, $g \in \mathcal{C}^{\tfrac{n+4}{2}}(\R^n)$, $h \in \mathcal{C}^{\tfrac{n+2}{2}}(\R^n)$ und $u : \ointerval{0}{\infty} \times \R^n \to \R$ definiert durch
  \begin{align*}
    u(t, x) \coloneqq & \left( \prod_{k=1}^{n/2} 2k \right)^{-1} \left[ \partial_t \left( t^{-1} \partial_t \right)^{\tfrac{n-2}{2}} \left( t^n \enspace \MVInt{\mathclap{B_t(x)}}{}{\enspace \frac{g(y)}{(t^2 - \abs{x-y}^2)^{1/2}}}{y} \right) \right. \\
    & \left. + \left( t^{-1} \partial_t \right)^{\tfrac{n-2}{2}} \left( t^n \enspace \MVInt{\mathclap{B_t(x)}}{}{ \enspace \frac{h(y)}{(t^2 - \abs{x-y}^2)^{1/2}}}{y} \right) \right]
  \end{align*}
  Dann ist $u \in \mathcal{C}^2(\cointerval{0}{\infty}) \times \R^n$ und die eindeutige Lösung der homogenen WG auf $\cointerval{0}{\infty} \times \R^n$ mit $u = g$ und $\partial_t u = h$ auf $\{ 0 \} \times \R^n$.
\end{satz}
% Ausgelassen: Es wird $g \in \mathcal{C}^{\tfrac{n+4}{2}}$, $h \in \mathcal{C}^{\tfrac{n+2}{2}}$ für $u \in \mathcal{C}^2$ benötigt

% Kapitel 4.1.3. Die inhomogene Gleichung

\begin{satz}
  Sei $f \in \mathcal{C}^{\lfloor n/2 \rfloor - 1}(\ointerval{0}{\infty} \times \R^n)$. Sei $v_s : \ointerval{s}{\infty} \times \R^n \to \R$ für $s \in \ointerval{0}{\infty}$ die Lösung von
  % TODO: Muss hier 0 statt $g$ stehen?
  \[
    \left\{ \begin{array}{rlll}
      \partial_t^2 v_s - \Delta v_s = 0 & \text{in } \ointerval{s}{\infty} \times \R^n, \\
      v_s = g, \enspace \partial_t v_s = f & \text{auf } \{ s \} \times \R^n.
    \end{array} \right.
  \]
  Setze $u(t, x) \coloneqq \Int{0}{t}{v_s(t, x)}{s}$. Dann ist $u \in \mathcal{C}^2(\cointerval{0}{\infty} \times \R^n)$ und $u$ löst die inhomogene Wellengleichung mit Nullanfangsdaten
  \[
    \left\{ \begin{array}{ll}
      \partial_t^2 u - \Delta u = f & \text{in } \ointerval{0}{\infty} \times \R^n, \\
      u = 0, \partial_t u = 0 & \text{auf } \{ 0 \} \times \R^n.
    \end{array} \right.
  \]
\end{satz}

\begin{bem}
  Aus den Darstellungsformeln für $n=1,3$ ergibt sich:
  \begin{align*}
    u_{n=1}(t, x) &= \tfrac{1}{2} \enspace \Int{0}{t}{\enspace \Int{x - \tilde{s}}{x + \tilde{s}}{f(t - \tilde{s}, y)}{y}}{\tilde{s}} \\
    u_{n=3}(t, x) &= \tfrac{1}{4 \pi} \quad \Int{\mathclap{\partial B_t(x)}}{}{\enspace \frac{f(t - \abs{y-x}, y)}{\abs{y-x}}}{y}
  \end{align*}
\end{bem}

% Ausgelassen: Bemerkung: "`Endliche Ausbreitungsgeschwindigkeit"' der Inhomogenität

% Kapitel 4.2. Die Wellengleichung auf allgemeineren Gebieten

\begin{lem}
  Sei $\Omega \subset \R^n$ offen, beschränkt mit $\mathcal{C}^1$-Rand $\partial \Omega$, $T > 0$ und $u \in \mathcal{C}^2(\overline{\Omega_T})$ eine Lösung der homogenen Wellengleichung $\partial_t^2 u - \Delta u = 0$ in $\Omega_T$ mit $u = 0$ auf $\cinterval{0}{T} \times \partial \Omega$. Sei
  \[
    e(t) \coloneqq \tfrac{1}{2} \enspace \Int{\Omega}{}{(\partial_t u(t,x))^2 + \abs{D u(t,x)}^2}{x} \quad \text{für } t \in \cinterval{0}{T}.
  \]
  Dann ist $e$ konstant auf $\cinterval{0}{T}$.
\end{lem}

\begin{satz}
  Sei $\Omega \subset \R^n$ offen, beschränkt mit $\mathcal{C}^1$-Rand $\partial \Omega$, $T > 0$, $g \in \mathcal{C}^2(\partial_p \Omega_T)$, $h \in \mathcal{C}^1(\Omega)$ und $f \in \mathcal{C}(\overline{\Omega})$. Seien $u, v$ Lösungen von
  \[
    \left\{ \begin{array}{ll}
      \partial_t^2 u - \Delta u = f & \text{in } \Omega_T, \\
      u = g & \text{auf } \partial_p \Omega_T, \\
      \partial_t u = h & \text{auf } \{ 0 \} \times \Omega.
    \end{array} \right.
  \]
  Dann gilt $u \equiv v$.
\end{satz}

\begin{defn}
  Sei $(t_0, x_0) \in \ointerval{0}{\infty} \times \R^n$. Dann heißt
  \[
    C(t_0, x_0) \coloneqq \Set{(t, x) \in \ointerval{0}{t_0} \times \R^n}{ \abs{x-x_0} < t_0 - t }
  \]
  \emph{Vergangenheitskegel} mit Spitze $(t_0, x_0)$.
\end{defn}

\begin{samepage}

\begin{satz}
  Sei $(t_0, x_0) \in \ointerval{0}{\infty} \times \R^n$ und löse $u \in \mathcal{C}^2(\overline{C(t_0, x_0)})$ die homogene Wellengleichung in $C(t_0, x_0)$ mit Anfangsbedingung $u = 0$, $\partial_t u = 0$ auf $\{ 0 \} \times B_{t_0}(x_0)$. Dann gilt $u \equiv 0$ in $C(t_0, x_0)$.
\end{satz}

% Vorlesung vom 17. und 24.06.2014

\section{PDGLn erster Ordnung}

\end{samepage}

\begin{prob}
  Sei $\Omega \subset \R^n$ offen, mit hinreichend regulären Rand und $g : \partial \Omega \to \R$ und $E : \overline{\Omega} \times \R \times \R^n \to \R$ hinreichend regulär (zumindest einmal stetig diff'bar). Gesucht ist eine Lösung $u$ von
  \[
    (5.1) \left\{ \begin{array}{llll}
      E(x, u(x), D u(x)) & = & 0 & \text{in } \Omega, \\
      u & = & g & \text{auf } \partial \Omega.
    \end{array} \right.
  \]
\end{prob}

\begin{nota}
  Bezeichnungen der Argumente: $E(x, v, z)$
\end{nota}

% Ausgelassen:
% * Satz 5.1 von Picard-Lindelöf
% * Satz 5.2 über implizite Funktionen
% * Satz 5.3 von der lokalen Umkehrbarkeit

% Kapitel 5.2. Lokale Existenz mittels Charakteristiken

\begin{satz}[Charakteristische GLn]
  Sei $E \in \mathcal{C}^1(\Omega \times \R \times \R^n)$ und $u \in \mathcal{C}^2(\Omega)$ eine Lösung von $E(x, u(x), D u(x)) = 0$ in $\Omega$. Sei $I \subset \R$ ein Intervall, $\gamma \in \mathcal{C}^1(I, \Omega)$, $v \coloneqq u \circ \gamma$ und $z \coloneqq D u \circ \gamma$.\\
  Ist $\gamma$ eine Lösung der DGL $\gamma' = D_z E(\gamma, v, z)$ in $I$, so gilt in $I$:
  \[
    v' = D_z(\gamma, v, z) \cdot z, \qquad
    z' = - D_x(\gamma, v, z) - D_v E(\gamma, v, z) \cdot z.
  \]
\end{satz}

\begin{defn}
  Die Fktn $\gamma$, $v$ und $z$ heißen \emph{Charakteristiken} von (5.1).
\end{defn}

\begin{nota}
  Sei im Folgenden $\Omega \coloneqq \R^n_{+} = \R^{n-1} \times \R_{+}$.
\end{nota}

\begin{defn}
  Ein Vektor $(x_0, v_0, z_0) \in \partial \Omega \times \R \times \R^n$ heißt \emph{zulässig} für das AWP (5.1), falls folgende Kompatibilitätsbedingungen erfüllt sind:
  \[
    g(x_0) = v_0, \quad
    (z_0)_i = D_i g(x_0) \text{ für } i \in \{ 1, ..., n{-}1 \}, \quad
    E(x_0, v_0, z_0) = 0.
  \]
\end{defn}

\begin{bem}
  Alle Werte bis auf $(z_0)_n$ sind eindt. durch $x_0$ bestimmt.
\end{bem}

\begin{defn}
  Der Vektor $(x_0, v_0, z_0)$ heißt \emph{nichtcharakteristisch}, falls $D_{z_n} E(x_0, v_0, z_0) \not= 0$.
\end{defn}

\begin{lem}
  Sei $(x_0, v_0, z_0)$ ein zulässiger, nichtcharakteristischer Vektor für das AWP (5.1). Dann existiert $\delta > 0$, sodass für alle $y \in B_{\delta}(x_0) \cap \partial \R^n_{+}$ eine eindeutige Lösung $\overline{z}(y)$ gibt für
  \[
    \overline{z_i}(y) = D_i g(y) \text{ für } i \in \{ 1, ..., n{-}1 \}
    \quad \text{und} \quad
    E(y, g(y), \overline{z}(y)) = 0.
  \]
\end{lem}

\begin{bem}
  Sei $(x_0, v_0, z_0) \in \partial \Omega \times \R \times \R^n$ ein zulässiger, nichtcharakteristischer Vektor. Dann existiert nach dem Satz von Picard-Lindelöf eine lokale Lösung $(\gamma(x_0, \blank), v(x_0, \blank), z(x_0, \blank))$ des Systems mit Anfangswerten
  \[
    \left\{ \begin{array}{rl}
      \gamma'(x_0, s) & = D_z E(\gamma, v, z), \\
      v'(x_0, s) & = D_z E(\gamma, v, z) \cdot z(x_0, s), \\
      z'(x_0, s) & = - D_x E(\gamma, v, z) - D_v E(\gamma, v, z) \cdot z(x_0, s), \\
      \gamma(x_0, 0) = x_0, & v(x_0, 0) = g(x_0), \enspace\,\, z(x_0, 0) = z_0
    \end{array} \right.
  \]
\end{bem}

% Ausgelassen: Homogene Transportgleichung

\begin{bsp}
  Die charakteristischen Gleichungen der allgemeinen quasilinearen Gleichung $a(\blank, u) \cdot D u + E_0(\blank, u) = 0$ sind
  \[ \gamma' = a(\gamma, v), \quad v' = a(\gamma, v) \cdot z = - E_0(\gamma, v). \]
  Die Gleichung für $z'$ wird nicht benötigt.
\end{bsp}

\begin{lem}
  Sei $(x_0, v_0, z_0)$ ein zulässiger, nichtcharakteristischer Vektor für das AWP (5.1). Dann existiert ein offenes Intervall $I \subset \R$ mit $0 \in I$ und Umgebungen $W \subset \partial \R^n_{+}$ und $V \subset \R^n_+$ von $x_0$, sodass für alle $x \in V$ eindeutige $s = s(x) \in I \cap \ointerval{0}{\infty}$ und $y = y(x) \in W$ existieren mit $x = \gamma(y, s)$ und die Abbildung $x \mapsto (y(x), s(x))$ von der Klasse $\mathcal{C}^2$ ist.
\end{lem}

\begin{defn}
  Seien $V, \Omega \subset \R^n$ offen mit $V \subset \Omega$. Eine Funktion $u \in \mathcal{C}^1(V) \cap \mathcal{C}^0(V \cup (\partial \Omega \cap \overline{V}))$ heißt \emph{lokale Lösung} von (5.1) in $V$, falls gilt:
  \[
    \left\{ \begin{array}{rlll}
      E(x, u(x), D u(x)) & = & 0 & \text{in } V, \\
      u & = & g & \text{auf } \partial \Omega \cap \overline{V}.
    \end{array} \right.
  \]
\end{defn}

\begin{satz}
  Sei $(x_0, v_0, z_0)$ ein zulässiger, nichtcharakteristischer Vektor für (5.1) und $V$ die Umgebung von $x_0$ in $\R^n_{+}$ aus dem letzten Lemma. Dann ist
  \[ u : V \to \R, \quad x \mapsto v(y(x), s(x)) \]
  wohldefiniert, von der Klasse $\mathcal{C}^1$ und eine lokale Lsg von (5.1) in $V$.
\end{satz}

% Vorlesung vom 26.6.2014

% Ausgelassen: Zwei Beispiele zur Bedeutung der Bedingung $D_{z_n} E(x_0, y_0, z_0) \not= 0$

% Kapitel 5.3. Skalare Erhaltungsgleichungen

\begin{prob}
  Die \emph{Erhaltungsgleichung} mit $F \in \mathcal{C}(\R, \R^n)$ ist
  \[
    \partial t u + \dive_x F(u) = \partial_t u + F'(u) \cdot Du = 0
    \quad \text{in } \ointerval{0}{\infty} \times \R^n.
  \]
\end{prob}

\begin{bsp}
  Mit $n=1$ und $F(u) \coloneqq \tfrac{u^2}{2}$ ergibt sich \emph{Burgers Gleichung}
  \[
    \partial t u + \dive_x \left( \frac{u^2}{2} \right) = \partial_t u + u \partial_x u = 0
    \quad \text{in } \ointerval{0}{\infty} \times \R.
  \]
  Die (wichtigen) charakteristischen Gleichungen sind
  \[
    \gamma' = (1, F'(v)) = (1, v), \quad
    v' = (1, F'(v)) \cdot z = 0.
  \]
  Die Charakteristiken sind also gerade Wege, auf denen $u$ konstant ist und deren Richtung von den Anfangswerten abhängt. Damit hängt die Lösbarkeit von den Anfangswerten ab. Es kann vorkommen, dass sich verschiedene Charakteristiken in einem Punkt treffen oder Punkte von keiner Charakteristik getroffen wird.
\end{bsp}

\begin{defn}
  Seien $F \in \mathcal{C}^1(\R, \R^n)$ und $g \in L^\infty(\R)$. Eine Funktion $u \in L^\infty(\ointerval{0}{\infty} \times \R)$ heißt \emph{schwache Lösung} des AWPs
  \[
    (5.10) \left\{ \begin{array}{rlll}
      \partial_t u + \dive_x(F(u)) & = & 0 & \text{in } \ointerval{0}{\infty} \times \R, \\
      u & = & g & \text{auf } \{ 0 \} \times \R,
    \end{array} \right.
  \]
  falls für alle Funktionen $v \in \mathcal{C}_0^\infty(\cointerval{0}{\infty} \times \R)$ gilt:
  \[
    \Int{0}{\infty}{\Int{\R}{}{\left( u \partial_t v + F(u) \cdot D v \right)}{x}}{t} + \Int{\R}{}{g(x) v(0, x)}{x} = 0.
  \]
\end{defn}

\begin{lem}
  Ist $u \in L^\infty(\ointerval{0}{\infty} \times \R)$ eine schwache Lösung von (5.10) mit $u \in \mathcal{C}^1(O)$ für eine offene Menge $O \subset \cointerval{0}{\infty} \times \R$, so gilt
  \[ \partial_t u + \dive F(u) = 0 \enspace \text{in } O. \]
\end{lem}

\begin{lem}[Rankine-Hugeniot-Bedingung]
  Ist $u \in L^\infty(\ointerval{0}{\infty} \times \R)$ eine schwache Lösung von (5.10) und $O \subset \ointerval{0}{\infty} \times \R$ eine offene Menge, die durch eine $\mathcal{C}^1$-Kurve $C$ in zwei disjunkte, offene Mengen $O_l$ und $O_r$ geteilt wird. Sei $u_l \in \mathcal{C}^1(O_l) \cap \mathcal{C}^0(O_l \cup C)$ mit $u = u_l$ in $O_l$ und $u_r \in \mathcal{C}^1(O_r) \cap \mathcal{C}^0(U_r \cup C)$ mit $u = u_r$ in $O_r$. Dann gilt
  \[
    \begin{pmatrix}
      u_l - u_r \\
      F(u_l) - F(u_r)
    \end{pmatrix} \cdot \nu = 0
    \quad \text{in } C,
  \]
  wobei $\nu$ den äußeren Normalvektor an $O_l$ bezeichnet.
\end{lem}

% Ausgelassen: Beispiele für schwache Lösungen von Burger's Gleichung


\section{Elliptische PDGLn zweiter Ordnung}

\begin{prob}
  Gesucht ist eine Lösung der PDGL in Divergenzform
  \begin{align*}
    Lu(x) \coloneqq & \sum_{1 \leq i, j \leq n} D_j (a_{ij}(x) D_{i} u(x)) + \sum_{1 \leq i \leq n} b_i(x) D_i u(x) \\
      & + c(x) u(x) = f(x) \tag{6.1}
  \end{align*}
  in einem beschränkten Gebiet $\Omega \subset \R^n$. Dabei sollen die Koeffizientenfunktionen $(a_{ij})_{1 \leq i, j \leq n}$, $(b_i)_{1 \leq i \leq n}$ und $c$ beschränkt und $f \in L^2(\Omega)$ sein.
\end{prob}

\begin{defn}
  Der Differentialoperator $L$ heißt \emph{gleichmäßig elliptisch}, falls eine Konstante $\theta > 0$ existiert mit
  \[ \fa{x \in \Omega, \xi \in \R^n} a(x) \xi \cdot \xi \coloneqq \sum_{1 \leq i, j \leq n} a_{ij}(x) \xi_i \xi_j \geq \theta \abs{\xi}. \]
\end{defn}

% Kapitel 6.1. Funktionalanalytische Vorbereitungen

% Ausgelassen: Definitionen von metrischer Raum, Cauchy-Folge, Konvergenz, Vollständigkeit, Normen, normierter Raum, Banachraum, Skalarprodukt

\begin{defn}
  Ein \emph{Hilbertraum} ist ein vollständiger normierter Vektorraum, dessen Norm von einem Skalarprodukt induziert wird.
\end{defn}

\begin{defn}
  Seien $(X, \norm{\blank}_X)$ und $(Y, \norm{\blank}_Y)$ zwei normierte Räume. Setze
  \begin{align*}
    \mathcal{L}(X, Y) & \coloneqq \Set{ T : X \to Y }{ T \text{ linear und stetig.} } \\
    \norm{T}_{\mathcal{L}(X, Y)} & \coloneqq \sup_{x \in X \setminus \{ 0 \}} \frac{\norm{Tx}_Y}{\norm{x}_X}.
  \end{align*}
\end{defn}

\begin{defn}
  Sei $(X, \norm{\blank}_X)$ ein normierter Raum. Der Raum $X^{*} \coloneqq \mathcal{L}(X, \R)$ heißt \emph{Dualraum}, seine Elemente \emph{beschränkte lineare Funktionale} auf $X$.
\end{defn}

\begin{satz}[Projektionssatz]
  Sei $H$ ein reeller Hilbertraum und $K \subset H$ nichtleer, abgeschlossen und konvex. Dann gibt es eine eindeutige Abbildung $P : H \to K$ mit
  \[ \fa{x \in H} \norm{x - P(x)}_H = \dist(x, K) \coloneqq \inf_{y \in K} \norm{x - y}_H. \]
\end{satz}

\begin{bem}
  Alternative Charakterisierungen der Projektion $P$:
  \begin{itemize}
    \item $\fa{x \in H, y \in K} \scp{x - P(x)}{y - P(x)}_H \leq 0$
    \item Falls $K \subset H$ ein Unterraum: $\fa{x \in H, y \in K} \scp{x - P(x)}{y}_H = 0$.
  \end{itemize}
\end{bem}

\begin{satz}[Darstellungssatz von Riesz]
  Sei $(H, \scp{\blank}{\blank}_H)$ ein reeller Hilbertraum. Dann ist die Abbildung
  \[
    J : H \to H^*, \quad
    x \mapsto y \mapsto \scp{x}{y}_H
  \]
  ein linearer isometrischer Isomorphismus, \dh{} es gilt $\norm{J(x)}_{H^*} = \norm{x}_H$ und $J$ ist bijektiv.
\end{satz}

\begin{satz}[Lax-Milgram]
  Sei $H$ ein reeller Hilbertraum und $B : H \times H \to \R$ eine Bilinearform, für die gilt:
  \begin{itemize}
    \item Beschränktheit: $\ex{L > 0} \fa{x, y \in H} B(x, y) \leq L \cdot \norm{x}_H \cdot \norm{y}_H$
    \item Koerzivität: $\ex{\theta > 0} B(x, x) \geq \theta \cdot \norm{x}_H^2$
  \end{itemize}
  Dann existiert eine lineare Bijektion $\Lambda : H^* \to H$, sodass
  \[ \fa{x^* \in H^*} x^* = B(\Lambda(x^*), \blank). \]
  Außerdem sind $\Lambda$ wie auch $\Lambda^{-1}$ beschränkt.
\end{satz}

% Kapitel 6.2. Schwache Ableitungen und Sobolevräume

\begin{defn}
  Sei $f \in L^1_{\text{loc}}(\Omega)$ und $\beta \in \N_0^n$ ein Multiindex. Eine Funktion $g_\beta \in L^1_{\text{loc}}(\Omega)$ heißt \emph{schwache Ableitung} von $f$ nach $\beta$, falls gilt:
  \[
    \fa{\phi \in \mathcal{C}_0^\infty(\Omega)}
    \Int{\Omega}{}{f D^\beta \phi}{x} = (-1)^{\abs{\beta}} \Int{\Omega}{}{g_\beta \cdot \phi}{x}.
  \]
  Falls für alle Multiindizes $\beta$ mit $\abs{\beta} \leq k$ eine schwache Ableitung existiert, so heißt $f$ \emph{schwach differenzierbar} bis zur Ordnung $k$.
\end{defn}

\begin{bem}
  \begin{itemize}
    \item Die schwache Ableitung ist eindeutig in $L^1_{\text{loc}}(\Omega)$.
    \item Falls eine starke Ableitung existiert, so stimmt sie mit der schwachen Ableitung überein.
    \item Alternative, äquivalente Definition: $\overline{D}^\beta f$ heißt schwache Ableitung von $f$ nach $\beta$, falls eine Folge $(f_k)_{k \in \N}$ existiert mit
    \[
      f_k \xrightarrow{k \to \infty} f \text{ stark in } L^1_{\text{loc}}(\Omega)
      \quad \text{und} \quad
      D^\beta f \xrightarrow{k \to \infty} \overline{D}^\beta f \text{ in } L^1_{\text{loc}}(\Omega).
    \]
  \end{itemize}
\end{bem}

\begin{nota}
  $D^\beta f \coloneqq g_\beta$ (wegen Eindeutigkeit)
\end{nota}

\begin{bsp}
  Die schwache Ableitung der Betragsfktn. ist die Signumsfktn. Die Signumsfunktion besitzt keine schwache Ableitung.
\end{bsp}

% Ausgelassen: Schwache Ableitung von $f(x) = \abs{x}^{- \gamma}$.

\begin{defn}
  Sei $k \in \N$ und $p \in \cinterval{1}{\infty}$. Der \emph{Sobolevraum} $W^{k,p}_{(\text{loc})}(\Omega)$ ist
  \begin{align*}
    W^{k,p}_{(\text{loc})}(\Omega) & \coloneqq \Set{ f : \Omega \to \R }{ \text{$D^\beta$ ex. in $L^p_{(\text{loc})}(\Omega)$ f.\,a. $\beta \in \N_0^n$ mit $\abs{\beta} \leq k$} } \\
    \norm{f}_{W^{k,p}(\Omega)} & \coloneqq \begin{cases}
      \left( \sum_{\abs{\beta} \leq k} \norm{D^\beta f}^p_{L^p(\Omega)} \right)^{1/p}, & \text{für } p < \infty \\
      \sum_{\abs{\beta} \leq k} \norm{D^\beta u}_{L^\infty(\Omega)}, & \text{für } p = \infty.
    \end{cases}
  \end{align*}
\end{defn}

\begin{bem}
  Alternativ können die Sobolevräume definiert werden durch Vervollständigung aller glatten Funktionen unter der $W^{k,p}(\Omega)$-Norm als
  \begin{align*}
    & H^{k,p}(\Omega) \coloneqq \overline{\mathcal{C}^\infty(\Omega) \cap W^{k,p}(\Omega)}^{\norm{\blank}_{W^{k,p}(\Omega)}} \\
    & = \Set{ f \in W^{k,p}(\Omega) }{ \ex{(f_m)_{m \in \N} \text{ in } \mathcal{C}^\infty(\Omega) \cap W^{k,p}(\Omega)} f_m \to f \text{ in } W^{k,p}(\Omega)}.
  \end{align*}
\end{bem}

% Ausgelassen: Kriterien für schwache Ableitungen in $L^p$

% 6.19
\begin{satz}
  Sei $f \in W^{k,p}(\Omega)$ für $k \in \N$, $p \in \cinterval{1}{\infty}$ und $\alpha \in \N_0^n$ mit $\abs{\alpha} \leq k$. Dann gelten:
  \begin{itemize}
    \item $D^\alpha f \in W^{k - \abs{\alpha},p}(\Omega)$ mit $D^\beta (D^\alpha f) = D^{\alpha + \beta} f$ für $\beta \in \N_0^n$ mit $\abs{\beta} \leq k - \abs{\alpha}$.
    \item $W^{k,p}(\Omega)$ ist ein Vektorraum und $D^\alpha : W^{k,p}(\Omega) \to W^{k - \abs{\alpha},p}(\Omega)$ linear.
    \item Ist $\Omega' \subset \Omega$ offen, so gilt $f|_{\Omega'} \in W^{k,p}(\Omega')$.
    \item Ist $\eta \in \mathcal{C}_0^\infty(\Omega)$, so gilt $\eta f \in W^{k,p}(\Omega)$ mit Leibnizformel
    \[
      D^\alpha(\eta f) = \sum_{\beta \leq \alpha} \binom{\alpha}{\beta} D^\beta \eta D^{\alpha - \beta} f
      \quad \text{mit} \quad
      \binom{\alpha}{\beta} \coloneqq \frac{\alpha!}{\beta! (\alpha - \beta)!}.
    \]
  \end{itemize}
\end{satz}

% 6.20
\begin{satz}
  Die Sobolevräume $W^{k,p}(\Omega)$ sind vollständig, also Banachräume für $k \in \N$ und $p \in \cinterval{1}{\infty}$.
\end{satz}

\begin{bem}
  $W^{k,2}(\Omega)$ ist ein Hilbertraum mit Skalarprodukt
  \[
    \scp{f}{g}_{W^{k,p}(\Omega)} \coloneqq \sum_{\abs{\beta} \leq k} \Int{\Omega}{}{D^\beta f D^\beta g}{x}.
  \]
\end{bem}

% 6.22
\begin{satz}[Eigenschaften von Glättungen]
  Sei $\Omega \subset \R^n$ offen, $\epsilon > 0$ und $f \in W^{k,p}_{\text{loc}}(\Omega)$ für $k \in \N_0$ und $p \in \cointerval{1}{\infty}$. Dann gilt:
  \begin{itemize}
    \item Vertauschbarkeit mit schwachen Ableitungen: Für alle Multiindizes $\beta \in \N_0^n$ mit $\abs{\beta} \leq k$:
    \[
      D^\beta (f_{\epsilon}) = (D^\beta f)_{\epsilon}
      \quad \text{in } \Omega_{\epsilon} \coloneqq \Set{x \in \Omega}{ \dist(x, \partial \Omega) > \epsilon }.
    \]
    \item Erhaltung der Norm: $\norm{f_\epsilon}_{W^{k,p}(\Omega_\epsilon)} \leq \norm{f}_{W^{k,p}(\Omega)}$.
    \item Approximation: $f_\epsilon \xrightarrow{\epsilon \to 0} f$ in $W^{k,p}_{\text{loc}}(\Omega)$.
  \end{itemize}
\end{satz}

% 6.23.
\begin{satz}
  Sei $\Omega \subset \R^n$ offen und beschränkt, $f \in W^{k,p}_{\text{loc}}(\Omega)$ für $k \in \N$ und $p \in \cointerval{1}{\infty}$. Dann gibt es eine Folge $(f_m)_{m \in \N}$ in $\mathcal{C}^\infty(\Omega) \cap W^{k,p}(\Omega)$ mit $f_m \to f$ in $W^{k,p}(\Omega)$.
\end{satz}

\begin{kor}
  $\mathcal{C}^\infty(\Omega) \cap W^{k,p}(\Omega)$ liegt dicht in $W^{k,p}(\Omega)$.
\end{kor}

% Ausgelassen: Bemerkung über Verbesserung des vorhergehenden Satzes bei glattem Rand

% Ausgelassen: Ausblick: Einbettungssätze, Fortsetzungssätze, Spursätze über Approximation mit glatten Funktionen

\begin{defn}
  Sei $k \in \N$ und $p \in \cointerval{1}{\infty}$. Wir setzen
  \[
    W^{k,p}_0(\Omega) \coloneqq \overline{\mathcal{C}_0^\infty(\Omega) \cap W^{k,p}(\Omega)}^{\norm{\blank}_{W^{k,p}(\Omega)}}.
  \]
\end{defn}

\begin{satz}[Poincaré-Ungleichung]
  Sei $\Omega \subset \R^n$ offen, die in einem Streifen der Breite $d > 0$ zwischen zwei parallelen Hyperebenen enthalten ist. Ist $p \in \cointerval{1}{\infty}$, so gilt für alle $f \in W_0^{1,p}(\Omega)$
  \[
    \Int{\Omega}{}{\abs{f}^p}{x} \leq \tfrac{d^p}{p} \Int{\Omega}{}{\abs{D f}^p}{x}.
  \]
\end{satz}

\begin{bem}
  Die Poincaré-Ungleichung gilt auch für Funktionen $f \in W^{1,p}(\Omega)$ mit verschwindendem Mittelwert auf $\Omega$ und $\Omega$ beschränkt und zusammenhängend.
\end{bem}

\begin{bem}
  Als abgeschlossener Unterraum von $W^{1,p}(\Omega)$ ist $W^{1,p}_0(\Omega)$ vollständig und nach der Poincaré-Ungleichung definiert
  \[ \norm{f}_{W_0^{1,p}(\Omega)} \coloneqq \norm{Df}_{L^p(\Omega, \R^n)} \]
  eine zu $\norm{\blank}_{W^{1,p}(\Omega)}$ äquivalente Norm auf $W^{1,p}_0(\Omega)$.
\end{bem}

% Kapitel 6.3. Existenz schwacher Lösungen

\begin{defn}
  Eine Fktn $u \in W^{1,2}(\Omega)$ heißt \emph{schwache Lsg} von (6.1), falls
  \begin{align*}
    & \Int{\Omega}{}{a(x) Du(x) \cdot D \phi(x)}{x} + \Int{\Omega}{}{b(x) \cdot Du(x) \phi(x)}{x} \\
    & + \Int{\Omega}{}{c(x) u(x) \phi(x)}{x} = \Int{\Omega}{}{f(x) \phi(x)}{x}
  \end{align*}
  für alle $\phi \in \mathcal{C}_0^\infty(\Omega)$ gilt. Verwendete Notation dabei:
  \[
    a Du \cdot D \phi \coloneqq \sum_{1 \leq i, j \leq n} a_{ij} D_i u D_j \phi
    \quad \text{und} \quad
    b \cdot D_u \phi \coloneqq \sum_{1 \leq i \leq n} b_i D_i u \phi.
  \]
\end{defn}

% Ausgelassen: Bemerkung, dass Gleichheit für größere Klasse von Test-Funktionen $W_0^{1,2}(\Omega)$ gilt

\begin{defn}
  Ist $L$ der lineare Operator aus der partiellen DGL (6.1), so ist die mit $L$ assoziierte Bilinearform $B_L : W_0^{1,2}(\Omega) \times W_0^{1,2}(\Omega) \to \R$ gegeben durch
  \[
    B_L(v, w) \coloneqq \Int{\Omega}{}{a(x) Dv(x) \cdot Dw(x)}{x} + \Int{\Omega}{}{b(x) \cdot Dv(x) w(x)}{x} + \Int{\Omega}{}{c(x) v(x) w(x)}{x}.
  \]
\end{defn}

% 6.34
\begin{lem}
  Sei $L$ ein gleichmäßig elliptischer Differentialoperator mit Elliptizitätskonstante $\theta$. Dann existieren Konstanten $C_1, C_2 > 0$ und $C_3 \geq 0$ (in Abhängigkeit von $a$, $b$, $c$, $n$, $\Omega$, $\theta$) mit
  \begin{align*}
    B_L(v,v) & \geq C_1 \norm{v}^2_{W_0^{1,2}(\Omega)} - C_3 \norm{v}^2_{L^2(\Omega)}, \\
    \abs{B_L(v,w)} & \leq C_2 \norm{v}_{W_0^{1,2}(\Omega)} \norm{w}_{W_0^{1,2}(\Omega)}.
  \end{align*}
\end{lem}

% 6.35.
\begin{satz}
  Sei $L$ ein glm. elliptischer Differentialoperator. Dann existiert eine Konstante $\gamma \geq 0$, sodass für alle $\mu \geq \gamma$ das Randwertproblem
  \[
    \left\{ \begin{array}{rlll}
      Lu + \mu u & = & f & \text{in } \Omega \\
      u & = & 0 & \text{auf } \partial \Omega
    \end{array} \right.
  \]
  für jedes $f \in L^2(\Omega)$ eine eindeutige schwache Lösung $u \in W_0^{1,2}(\Omega)$ besitzt. Im Fall $b = 0$, $c \geq 0$ kann $\gamma = 0$ gewählt werden.
\end{satz}

% Ausgelassen: Bemerkungen

\end{document}